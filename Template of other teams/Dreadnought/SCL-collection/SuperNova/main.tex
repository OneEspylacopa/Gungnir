\documentclass[a4paper,10pt]{book}
\usepackage{amsmath}
\usepackage{amssymb}
\usepackage{fontspec}
\usepackage{listings} 
\usepackage{harpoon}
\usepackage[left=1.5cm, right=1.5cm]{geometry}
\usepackage[BoldFont]{xeCJK}
\oddsidemargin -0.1 true cm
\if@twoside
	\evensidemargin -0.1 true cm
\fi
\setlength{\parindent}{0em}
\setCJKmainfont{Microsoft YaHei}
\lstset{
	language=C++,
	numbers=left,
	tabsize=4,
	breaklines=tr,
	extendedchars=false
}

\title{\Large{SuperNova} \\ [2ex] \LARGE{Standard Code Library} \\[2ex] \begin{normalsize} version 3.141 \end{normalsize}}
\date{October, 2013}

\begin{document}
\maketitle

\tableofcontents

\newpage

\chapter{二维几何}
	\section{Naive Tips}
		\begin{enumerate}
	\item 注意舍入方式(0.5的舍入方向),防止输出-0
	\item 几何题注意多测试不对称数据
	\item 整数几何注意避免出界
	\item 符点几何注意EPS的使用
	\item 公式化简后再代入
	\item atan2(0,0)=0, atan2的值域为[-$\pi$, $\pi$]
	\item 使用acos, asin, sqrt 等函数时,注意定义域
\end{enumerate}


	\section{几何公式}
		\subsection*{三角形}
\begin{enumerate}
	\item 半周长
			$P=(a+b+c)/2$
	\item 面积
			$S=aH_a/2=ab\sin(C)/2=\sqrt{P(P-a)(P-b)(P-c)}$
	\item 中线
		$M_a=\sqrt{2(b^2+c^2)-a^2)}/2=\sqrt{b^2+c^2+2bc\cos(A)}/2$
	\item 角平分线 
		$T_a=\sqrt{bc((b+c)^2-a^2)}/(b+c)=2bc\cos(A/2)/(b+c)$
	\item 高线
		$H_a=b\sin(C)=c\sin(B)=\sqrt{b^2-((a^2+b^2-c^2)/(2a))^2}$
	\item 内切圆半径
		\begin{align*}
			r&=S/P=\arcsin(B/2)\sin(C/2)/\sin((B+C)/2)=4R\sin(A/2)\sin(B/2)\sin(C/2)\\
			&=\sqrt{(P-a)(P-b)(P-c)/P}=P\tan(A/2)\tan(B/2)\tan(C/2)
		\end{align*}
	\item 外接圆半径
		$R=abc/(4S)=a/(2\sin(A))=b/(2\sin(B))=c/(2\sin(C))$
\end{enumerate}

\subsection*{四边形}
	$D1,D2$为对角线,$M$对角线中点连线,$A$为对角线夹角
	\begin{enumerate}
		\item $a^2+b^2+c^2+d^2=D1^2+D2^2+4M^2$
		\item $S=D1D2\sin(A)/2$
		\item 圆内接四边形 $ac+bd=D1D2$
		\item 圆内接四边形,$P$为半周长 $S=\sqrt{(P-a)(P-b)(P-c)(P-d)}$
	\end{enumerate}

\subsection*{正n边形}
	$R$为外接圆半径,$r$为内切圆半径
	\begin{enumerate}
		\item 中心角 $A=2\pi/n$
		\item 内角 $C=(n-2)\pi/n$
		\item 边长 $a=2\sqrt{R^2-r^2}=2R\sin(A/2)=2r\tan(A/2)$
		\item 面积 $S=nar/2=nr^2\tan(A/2)=nR^2\sin(A)/2=na^2/(4\tan(A/2))$
	\end{enumerate}

\subsection*{圆}
	\begin{enumerate}
		\item 弧长 $l=rA$
		\item 弦长 $a=2\sqrt{2hr-h^2}=2r\sin(A/2)$
		\item 弓形高 $h=r-\sqrt{r^2-a^2/4}=r(1-\cos(A/2))=\arctan(A/4)/2$
		\item 扇形面积 $S1=rl/2=r^2A/2$
		\item 弓形面积 $S2=(rl-a(r-h))/2=r^2(A-\sin(A))/2$
	\end{enumerate}
	
\subsection*{棱柱}
	\begin{enumerate}
		\item 体积$V=Ah$,$A$为底面积,$h$为高
		\item 侧面积 $S=lp$,$l$为棱长,$p$为直截面周长
		\item 全面积 $T=S+2A$
	\end{enumerate}

\subsection*{棱锥}
	\begin{enumerate}
		\item 体积$V=Ah$,$A$为底面积,$h$为高
		\item 正棱锥侧面积 $S=lp$,$l$为棱长,$p$为直截面周长
		\item 正棱锥全面积 $T=S+2A$
	\end{enumerate}

\subsection*{棱台}
	\begin{enumerate}
		\item 体积 $V=(A1+A2+\sqrt{A1A2})h/3$, $A1,A2$为上下底面积,$h$为高
		\item 正棱台侧面积 $S=(p1+p2)l/2$,$p1,p2$为上下底面周长,$l$为斜高
		\item 正棱台全面积 $T=S+A1+A2$
	\end{enumerate}

\subsection*{圆柱}
	\begin{enumerate}
		\item 侧面积 $S=2\pi rh$
		\item 全面积 $T=2\pi r(h+r)$
		\item 体积 $V=\pi r^2h$
	\end{enumerate}

\subsection*{圆锥}
	\begin{enumerate}
		\item 母线 $l=\sqrt{h^2+r^2}$
		\item 侧面积 $S=\pi rl$
		\item 全面积 $T=\pi r(l+r)$
		\item 体积 $V=\pi r^2h/3$
	\end{enumerate}

\subsection*{圆台}
	\begin{enumerate}
		\item 母线 $l=\sqrt{h^2+(r1-r2)^2}$
		\item 侧面积 $S=\pi(r1+r2)l$
		\item 全面积 $T=\pi r1(l+r1)+\pi r2(l+r2)$
		\item 体积 $V=\pi(r1^2+r2^2+r1r2)h/3$
	\end{enumerate}

\subsection*{球}
	\begin{enumerate}
		\item 全面积 $T=4\pi r^2$
		\item 体积 $V=4\pi r^3/3$
	\end{enumerate}

\subsection*{球台}
	\begin{enumerate}
		\item 侧面积 $S=2\pi rh$
		\item 全面积 $T=\pi(2rh+r1^2+r2^2)$
		\item 体积 $V=\pi h(3(r1^2+r2^2)+h^2)/6$
	\end{enumerate}

\subsection*{球扇形}
	\begin{enumerate}
		\item 全面积 $T=\pi r(2h+r0)$,$h$为球冠高,$r0$为球冠底面半径
		\item 体积 $V=2\pi r^2h/3$
	\end{enumerate}


	\section{点类}
		\begin{lstlisting} 
#include <cmath>
#include <cstdio>
#include <vector>
#include <cstring>
#include <iostream>
#include <algorithm>
#define foreach(e,x) for(__typeof(x.begin()) e=x.begin();e!=x.end();++e)
using namespace std;

const double PI = acos(-1.);
const double EPS = 1e-8;
inline int sign(double a) {
	return a < -EPS ? -1 : a > EPS;
}

struct Point {
	double x, y;
	Point() {
	}
	Point(double _x, double _y) :
			x(_x), y(_y) {
	}
	Point operator+(const Point&p) const {
		return Point(x + p.x, y + p.y);
	}
	Point operator-(const Point&p) const {
		return Point(x - p.x, y - p.y);
	}
	Point operator*(double d) const {
		return Point(x * d, y * d);
	}
	Point operator/(double d) const {
		return Point(x / d, y / d);
	}
	bool operator<(const Point&p) const {
		int c = sign(x - p.x);
		if (c)
			return c == -1;
		return sign(y - p.y) == -1;
	}
	double dot(const Point&p) const {
		return x * p.x + y * p.y;
	}
	double det(const Point&p) const {
		return x * p.y - y * p.x;
	}
	double alpha() const {
		return atan2(y, x);
	}
	double distTo(const Point&p) const {
		double dx = x - p.x, dy = y - p.y;
		return hypot(dx, dy);
	}
	double alphaTo(const Point&p) const {
		double dx = x - p.x, dy = y - p.y;
		return atan2(dy, dx);
	}
	//clockwise
	Point rotAlpha(const double &alpha, const Point &o = Point(0, 0)) const {
		double nx = cos(alpha) * (x - o.x) + sin(alpha) * (y - o.y);
		double ny = -sin(alpha) * (x - o.x) + cos(alpha) * (y - o.y);
		return Point(nx, ny) + o;
	}
	Point rot90() const {
		return Point(-y, x);
	}
	Point unit() {
		return *this / abs();
	}
	void read() {
		scanf("%lf%lf", &x, &y);
	}
	double abs() {
		return hypot(x, y);
	}
	double abs2() {
		return x * x + y * y;
	}
	void write() {
		cout << "(" << x << "," << y << ")" << endl;
	}
};

#define cross(p1,p2,p3) ((p2.x-p1.x)*(p3.y-p1.y)-(p3.x-p1.x)*(p2.y-p1.y))
#define crossOp(p1,p2,p3) sign(cross(p1,p2,p3))

Point isSS(Point p1, Point p2, Point q1, Point q2) {
	double a1 = cross(q1,q2,p1), a2 = -cross(q1,q2,p2);
	return (p1 * a2 + p2 * a1) / (a1 + a2);
}

double minDiff(double a, double b) // a, b in[0, 2 * PI)
{
	return min(abs(a - b), 2 * PI - abs(a - b));
}
\end{lstlisting} 

	\section{基本操作}
		顺时针或逆时针传入一个凸多边形,返回被半平面$\overrightharp{q1q2}$逆时针方向切割掉之后的凸多边形
\begin{lstlisting}
vector<Point> convexCut(const vector<Point>&ps, Point q1, Point q2) {
	vector<Point> qs;
	int n = ps.size();
	for (int i = 0; i < n; ++i) {
		Point p1 = ps[i], p2 = ps[(i + 1) % n];
		int d1 = crossOp(q1,q2,p1), d2 = crossOp(q1,q2,p2);
		if (d1 >= 0)
			qs.push_back(p1);
		if (d1 * d2 < 0)
			qs.push_back(isSS(p1, p2, q1, q2));
	}
	return qs;
}
\end{lstlisting}
返回ps的有向面积
\begin{lstlisting}
double calcArea(const vector<Point>&ps) {
	int n = ps.size();
	double ret = 0;
	for (int i = 0; i < n; ++i) {
		ret += ps[i].det(ps[(i + 1) % n]);
	}
	return ret / 2;
}
\end{lstlisting}
返回点集ps组成的凸包
\begin{lstlisting}
vector<Point> convexHull(vector<Point> ps) {
	int n = ps.size();
	if (n <= 1)
		return ps;
	sort(ps.begin(), ps.end());
	vector<Point> qs;
	for (int i = 0; i < n; qs.push_back(ps[i++])) {
		while (qs.size() > 1 && crossOp(qs[qs.size()-2],qs.back(),ps[i]) <= 0)
			qs.pop_back();
	}
	for (int i = n - 2, t = qs.size(); i >= 0; qs.push_back(ps[i--])) {
		while ((int)qs.size() > t && crossOp(qs[(int)qs.size()-2],qs.back(),ps[i]) <= 0)
			qs.pop_back();
	}
	qs.pop_back();
	return qs;
}
\end{lstlisting}
返回凸包ps的直径
\begin{lstlisting}
double convexDiameter(const vector<Point>&ps) {
	int n = ps.size();
	int is = 0, js = 0;
	for (int i = 1; i < n; ++i) {
		if (ps[i].x > ps[is].x)
			is = i;
		if (ps[i].x < ps[js].x)
			js = i;
	}
	double maxd = ps[is].distTo(ps[js]);
	int i = is, j = js;
	do {
		if ((ps[(i + 1) % n] - ps[i]).det(ps[(j + 1) % n] - ps[j]) >= 0)
			(++j) %= n;
		else
			(++i) %= n;
		maxd = max(maxd, ps[i].distTo(ps[j]));
	} while (i != is || j != js);
	return maxd;
}
\end{lstlisting}
判断点p在线段q1q2上,端点重合返回true
\begin{lstlisting}
int onSegment(Point p, Point q1, Point q2)
{
	return crossOp(q1, q2, p) == 0 && sign((p - q1).dot(p - q2)) <= 0;
}
\end{lstlisting}
判断线段p1p2和q1q2是否严格相交,重合或端点相交返回false
\begin{lstlisting}
int isIntersect(Point p1, Point p2, Point q1, Point q2)
{
	return crossOp(p1, p2, q1) * crossOp(p1, p2, q2) < 0 && crossOp(q1, q2, p1) * cross(q1, q2, p2) < 0;
}
\end{lstlisting}
判断直线p1p2和q1q2是否平行
\begin{lstlisting}
int isParallel(Point p1, Point p2, Point q1, Point q2)
{
	return sign((p2 - p1).det(q2 - q1)) == 0;
}
\end{lstlisting}
返回点p到直线uv的距离
\begin{lstlisting}
double distPointToLine(Point p, Point u, Point v)
{
	return abs((u - p).det(v - p)) / u.distTo(v);
}
\end{lstlisting}
判断点q是否在简单多边形p内部,边界返回false
\begin{lstlisting}
int insidePolygon(Point q, vector<Point> &p)
{
	int n = p.size();
	for(int i = 0; i < n; ++ i) {
		if (onSegment(q, p[i], p[(i + 1) % n])) return false;
	}
	Point q2;
	double offsite = LIM;
	for( ; ; ) {
		int flag = true;
		int rnd = rand() % 10000;
		q2.x = cos(rnd) * offsite;
		q2.y = sin(rnd) * offsite;
		for(int i = 0; i < n; ++ i) {
			if (onSegment(p[i], q, q2)) {
				flag = false;
				break;
			}
		}
		if (flag) break;
	}
	int cnt = 0;
	for(int i = 0; i < n; ++ i) {
		cnt += isIntersect(p[i], p[(i + 1) % n], q, q2);
	}
	return cnt & 1;
}
\end{lstlisting}
判断直线l1l2是否与圆相交,相切返回true
\begin{lstlisting}
int isIntersectLineToCircle(Point c, double r, Point l1, Point l2)
{
	return (distPointToLine(c, l1, l2) - r) <= 0;
}
\end{lstlisting}
判断圆与线段是否有公共点,线段在圆内部返回true
\begin{lstlisting}
int isIntersectSegmentToCircle(Point c, double r, Point p1, Point p2)
{
	if ((distPointToLine(c, p1, p2) - r) > 0) return false;
	if (sign(c.distTo(p1) - r) <= 0 || sign(c.distTo(p2) - r) <= 0) return true;
	Point c2 = (p2 - p1).rot90() + c;
	return crossOp(c, c2, p1) * crossOp(c, c2, p2) <= 0;
}
\end{lstlisting}
判断圆与圆是否相交,外切或内切返回true
\begin{lstlisting}
int isIntersectCircleToCircle(Point c1, double r1, Point c2, double r2)
{
	double dis = c1.distTo(c2);
	return sign(dis - abs(r1 - r2)) >= 0 && sign(dis - (r1 + r2)) <= 0;
}
\end{lstlisting}
求直线与圆的两个交点
\begin{lstlisting}
void intersectionLineToCircle(Point c, double r, Point l1, Point l2, Point& p1, Point& p2) {
	Point c2 = c + (l2 - l1).rot90();
	c2 = isSS(c, c2, l1, l2);
	double t = sqrt(r * r - (c2 - c).abs2());
	p1 = c2 + (l2 - l1).unit() * t;
	p2 = c2 - (l2 - l1).unit() * t;
}
\end{lstlisting}
求圆与圆的两个交点
\begin{lstlisting}
void intersectionCircleToCircle(Point c1, double r1, Point c2, double r2, Point &p1, Point &p2) {
	double t = (1 + (r1 * r1 - r2 * r2) / (c1 - c2).abs2()) / 2;
	Point u = c1 + (c2 - c1) * t;
	Point v = u + (c2 - c1).rot90();
	intersectionLineToCircle(c1, r1, u, v, p1, p2);
}
\end{lstlisting}

	\section{球面}
		计算圆心角lat表示纬度,$-90\leq w\leq 90$,lng表示经度 \\
返回两点所在大圆劣弧对应圆心角,$0\leq angle \leq \pi$
\begin{lstlisting}
double angle(double lng1,double lat1,double lng2,double lat2) {
	double dlng = abs(lng1 - lng2) * PI / 180; 
	while(dlng >= PI + PI) dlng -= PI + PI;
	if (dlng > PI) dlng = PI + PI - dlng;
	lat1 *= PI / 180, lat2 *= PI / 180;
	return acos(cos(lat1) * cos(lat2) * cos(dlng) + sin(lat1) * sin(lat2));
}
\end{lstlisting}

计算直线距离,$r$为球半径
\begin{lstlisting}
double line_dist(double r,double lng1,double lat1,double lng2,double lat2) {
	double dlng = abs(lng1 - lng2) * PI / 180; 
	while(dlng >= PI + PI) dlng -= PI + PI;
	if (dlng > PI) dlng = PI + PI - dlng;
	lat1 *= PI / 180, lat2 *= PI / 180;
	return r * sqrt(2 - 2 * (cos(lat1) * cos(lat2) * cos(dlng) + sin(lat1) * sin(lat2)));
}
\end{lstlisting}

计算球面距离,$r$为球半径 
\begin{lstlisting}
inline double sphere_dist(double r,double lng1,double lat1,double lng2,double lat2){
	return r * angle(lng1, lat1, lng2, lat2);
}
\end{lstlisting}

	\section{半平面交}
		\begin{lstlisting}
struct Border {
	Point p1, p2;
	double alpha;
	void setAlpha() {
		alpha = atan2(p2.y - p1.y, p2.x - p1.x);
	}
	void read() {
		p1.read();
		p2.read();
		setAlpha();
	}
};

int n;
const int MAX_N_BORDER = 20000 + 10;
Border border[MAX_N_BORDER];

bool operator<(const Border&a, const Border&b) {
	int c = sign(a.alpha - b.alpha);
	if (c != 0)
		return c == 1;
	return crossOp(b.p1,b.p2,a.p1) >= 0;
}

bool operator==(const Border&a, const Border&b) {
	return sign(a.alpha - b.alpha) == 0;
}

const double LARGE = 10000;

void add(double x, double y, double nx, double ny) {
	border[n].p1 = Point(x, y);
	border[n].p2 = Point(nx, ny);
	border[n].setAlpha();
	n++;
}

Point isBorder(const Border&a, const Border&b) {
	return isSS(a.p1, a.p2, b.p1, b.p2);
}

Border que[MAX_N_BORDER];
int qh, qt;

bool check(const Border&a, const Border&b, const Border&me) {
	Point is = isBorder(a, b);
	return crossOp(me.p1,me.p2,is) > 0;
}

void convexIntersection() {
	qh = qt = 0;
	sort(border, border + n);
	n = unique(border, border + n) - border;
	for (int i = 0; i < n; ++i) {
		Border cur = border[i];
		while (qh + 1 < qt && !check(que[qt - 2], que[qt - 1], cur))
			--qt;
		while (qh + 1 < qt && !check(que[qh], que[qh + 1], cur))
			++qh;
		que[qt++] = cur;
	}
	while (qh + 1 < qt && !check(que[qt - 2], que[qt - 1], que[qh]))
		--qt;
	while (qh + 1 < qt && !check(que[qh], que[qh + 1], que[qt - 1]))
		++qh;
}

void calcArea() {
	static Point ps[MAX_N_BORDER];
	int cnt = 0;

	if (qt - qh <= 2) {
		puts("0.0");
		return;
	}

	for (int i = qh; i < qt; ++i) {
		int next = i + 1 == qt ? qh : i + 1;
		ps[cnt++] = isBorder(que[i], que[next]);
	}

	double area = 0;
	for (int i = 0; i < cnt; ++i) {
		area += ps[i].det(ps[(i + 1) % cnt]);
	}
	area /= 2;
	area = fabsl(area);
	cout.setf(ios::fixed);
	cout.precision(1);
	cout << area << endl;
}

void halfPlaneIntersection()
{
	cin >> n;
	for (int i = 0; i < n; ++i) {
		border[i].read();
	}
	add(0, 0, LARGE, 0);
	add(LARGE, 0, LARGE, LARGE);
	add(LARGE, LARGE, 0, LARGE);
	add(0, LARGE, 0, 0);

	convexIntersection();
	calcArea();
}
\end{lstlisting}

	\section{最小圆覆盖}
		\begin{lstlisting}
#include<cmath>
#include<cstdio>
#include<algorithm>
using namespace std;
const double eps=1e-6;
struct couple
{
	double x, y;
	couple(){}
	couple(const double &xx, const double &yy)
	{
		x = xx; y = yy;
	}
} a[100001];
int n;
bool operator < (const couple & a, const couple & b)
{
	return a.x < b.x - eps or (abs(a.x - b.x) < eps and a.y < b.y - eps);
}
bool operator == (const couple & a, const couple & b)
{
	return !(a < b) and !(b < a);
}
inline couple operator - (const couple &a, const couple &b)
{	
	return couple(a.x-b.x, a.y-b.y);
}
inline couple operator + (const couple &a, const couple &b)
{
	return couple(a.x+b.x, a.y+b.y);
}
inline couple operator * (const couple &a, const double &b)
{
	return couple(a.x*b, a.y*b);
}
inline couple operator / (const couple &a, const double &b)
{
	return a*(1/b);
}
inline double operator * (const couple &a, const couple &b)
{
	return a.x*b.y-a.y*b.x;
}
inline double len(const couple &a)
{
	return a.x*a.x+a.y*a.y;
}
inline double di2(const couple &a, const couple &b)
{
	return (a.x-b.x)*(a.x-b.x)+(a.y-b.y)*(a.y-b.y);
}
inline double dis(const couple &a, const couple &b)
{
	return sqrt((a.x-b.x)*(a.x-b.x)+(a.y-b.y)*(a.y-b.y));
}
struct circle
{
	double r; couple c;
} cir;
inline bool inside(const couple & x)
{
	return di2(x, cir.c) < cir.r*cir.r+eps;
}
inline void p2c(int x, int y)
{
	cir.c.x = (a[x].x+a[y].x)/2;
	cir.c.y = (a[x].y+a[y].y)/2;
	cir.r = dis(cir.c, a[x]);
}
inline void p3c(int i, int j, int k)
{
	couple x = a[i], y = a[j], z = a[k];
	cir.r = sqrt(di2(x,y)*di2(y,z)*di2(z,x))/fabs(x*y+y*z+z*x)/2;
	couple t1((x-y).x, (y-z).x), t2((x-y).y, (y-z).y), t3((len(x)-len(y))/2, (len(y)-len(z))/2);
	cir.c = couple(t3*t2, t1*t3)/(t1*t2);
}
inline circle mi()
{
	sort(a + 1, a + 1 + n);
	n = unique(a + 1, a + 1 + n) - a - 1;
	if(n == 1)
	{
		cir.c = a[1];
		cir.r = 0;
		return cir;
	}
	random_shuffle(a + 1, a + 1 + n);
	p2c(1, 2);
	for(int i = 3; i <= n; i++)
		if(!inside(a[i]))
		{
			p2c(1, i);
			for(int j = 2; j < i; j++)
				if(!inside(a[j]))
				{
					p2c(i, j);
					for(int k = 1; k < j; k++)
						if(!inside(a[k]))
							p3c(i,j, k);
				}
		}
	return cir;
}
\end{lstlisting}

	\section{求直线与凸包的交点}
		\subsection{tEJtM}
			\begin{lstlisting}
#include<cstring>
#include<cstdio>
#include<cmath>
#include<algorithm>
using namespace std;
struct couple
{
	double x, y;
	couple(){}
	couple(const double & _x, const double & _y) : x(_x), y(_y) {} 
	void scan() {scanf("%lf%lf", &x, &y); }
} cp1, cp2, x, y;
double operator * (const couple & a, const couple & b) {return a.x * b.y - a.y * b.x;}
couple operator - (const couple & a, const couple & b) {return couple(a.x - b.x, a.y - b.y);}
couple operator + (const couple & a, const couple & b) {return couple(a.x + b.x, a.y + b.y);}
couple operator * (const double & a, const couple & b) {return couple(a * b.x, a * b.y);}
bool les(const couple & a, const couple & b) {return a.x < b.x or a.x == b.x and a.y < b.y;}
bool mor(const couple & a, const couple & b) {return a.x > b.x or a.x == b.x and a.y > b.y;}
int n, m, mxi, mni, t1, t2, c1, c2, mi;
double eps = 1e-12;
int sign(const double & x) {return x > eps?1:x < -eps?-1:0;}
couple cross(const couple &a, const couple &b, const couple &c, const couple &d)
{
	if(sign((b - a) * (d - c)) == 0) return a;
	double lambda = (c - a) * (d - c) / ((b - a) * (d - c));
	return a + lambda * (b - a);
}
double s[50001];
struct convex_polygon
{
	couple a[50000];
	couple & operator [] (int x) {return a[(x % n + n) % n];}
	int get_max(bool (*cmp)(const couple & a, const couple & b)) {int rtn = 0; for(int i = 1; i < n; i++) if(cmp(a[i], a[rtn])) rtn = i; return rtn;}
} a;
int check(int id)
{
	return (sign((y - x) * (a[id - 1] - a[id]))) * sign((y - x) * (a[id + 1] - a[id])) == 0?-sign((y - x) * (a[id + 1] - a[id])):sign((sign((y - x) * (a[id - 1] - a[id]))) - sign((y - x) * (a[id + 1] - a[id])));
}
int check1(int id)
{
	return sign((y - x) * (a[id] - x)) * sign((y - x) * (a[id + 1] - x)) <= 0?0:sign((y - x) * (a[id] - x));
}
int di(int (*check)(int), int le, int ri)
{
	int nor = check(le), mid;
	if(le > ri) ri += n;
	while(le != ri)
	{
		mid = (le + ri) / 2;
		if(0 == check(mid)) return mid;
		else if(nor == check(mid)) le = mid + 1;
		else ri = mid - 1;
	}
	return le;
}
double area(int le, int ri)
{
	le %= n; ri %= n;
	return (le <= ri)?(s[ri] - s[le]):(s[n] - s[le] + s[ri]);
}
int main()
{
	freopen("sgu345.in", "r", stdin);
	scanf("%d", &n);
	for(int i = 1; i <= n; i++)
	{
		a[i].scan();
		if(i >= 3 and fabs((a[i] - a[i - 1]) * (a[i - 1] - a[i - 2])) < eps)
		{
			a[i - 1] = a[i];
			i--; n--;
		}
	}
	s[0] = 0;
	for(int i = 1; i <= n; i++)
	{
		s[i] = s[i - 1] + a[i - 1] * a[i];
	}
	mni = a.get_max(les);
	mxi = a.get_max(mor);
	scanf("%d", &m);
	for(int i = 1; i <= m; i++)
	{
		x.scan(); y.scan();
		if(check(mni)== 0 or check(mxi)== 0)
		{
			if(check(mxi) == 0) mi = mxi; else mi = mni;
			t1 = mi;
			t2 = di(check, mi + 1, mi - 1);
		}else
		{
			t1 = di(check, mni, mxi);
			t2 = di(check, mxi, mni);
		}
		c1 = di(check1, t1, t2);
		c2 = di(check1, t2, t1);
		if(check1(c1) and check1(c2)) {printf("0\n"); continue;}
		cp1 = cross(a[c1], a[c1 + 1], x, y);
		cp2 = cross(a[c2], a[c2 + 1], x, y);
		printf("%.10f\n", min(fabs(area(c1 + 1, c2) + cp1 * a[c1 + 1] + a[c2] * cp2 + cp2 * cp1) / 2, fabs(area(c2 + 1, c1) + a[c1] * cp1 + cp1 * cp2 + cp2 * a[c2 + 1]) / 2));
	}
	fclose(stdin);
	return 0;
}
\end{lstlisting}

		\subsection{Seraphim}
			\begin{lstlisting}
double calc(point a, point b){
	double k=atan2(b.y-a.y , b.x-a.x); if (k<0) k+=2*pi;return k;
}
//= the convex must compare y, then x£¬a[0] is the lower-right point
//======= three is no 3 points in line. a[] is convex 0~n-1
void prepare(point a[] ,double w[],int &n) {
	int i; rep(i,n) a[i+n]=a[i];
	a[2*n]=a[0];
	rep(i,n) { w[i]=calc(a[i],a[i+1]);w[i+n]=w[i];}
}
int find(double k,int n , double w[]){
	if (k<=w[0] || k>w[n-1]) return 0; int l,r,mid; l=0; r=n-1;
	while (l<=r) { mid=(l+r)/2;if (w[mid]>=k) r=mid-1; else l=mid+1;
	}return r+1;
}
int dic(const point &a, const point &b , int l ,int r , point c[]) {
	int s; if (area(a,b,c[l])<0) s=-1; else s=1; int mid;
	while (l<=r) {
		mid=(l+r)/2; if (area(a,b,c[mid])*s <= 0) r=mid-1;
		else l=mid+1;
	}return r+1;
}
point get(const point &a, const point &b, point s1, point s2) {
	double k1,k2; point tmp; k1=area(a,b,s1); k2=area(a,b,s2);
	if (cmp(k1)==0) return s1; if (cmp(k2)==0) return s2;
	tmp=(s1*k2 - s2*k1) / (k2-k1);
	return tmp;
}
bool line_cross_convex(point a, point b ,point c[] , int n, point &cp1, point &cp2 , double w[]) {
	int i,j;
	i=find(calc(a,b),n,w);
	j=find(calc(b,a),n,w);
	double k1,k2;
	k1=area(a,b,c[i]); k2=area(a,b,c[j]);
	if (cmp(k1)*cmp(k2)>0) return false; //no cross
	if (cmp(k1)==0 || cmp(k2)==0) {
		//cross a point or a line in the convex
		if (cmp(k1)==0) {
			if (cmp(area(a,b,c[i+1]))==0) {cp1=c[i]; cp2=c[i+1];}
			else cp1=cp2=c[i];
			return true;
		}
		if (cmp(k2)==0) {
			if (cmp(area(a,b,c[j+1]))==0) {cp1=c[j];cp2=c[j+1];
			}else cp1=cp2=c[j];
		}return true;
	}
	if (i>j) swap(i,j); int x,y;
	x=dic(a,b,i,j,c); y=dic(a,b,j,i+n,c);
	cp1=get(a,b,c[x-1],c[x]); cp2=get(a,b,c[y-1],c[y]);
	return true;
}
\end{lstlisting}

	\section{点到凸包的切线}
	    \begin{lstlisting}
#include<cstring>
#include<cstdio>
#include<algorithm>
using namespace std;
struct couple
{
	long long x, y;
	couple(){}
	couple(const long long  & _x, const long long &_y) {x = _x; y = _y;}
	void scan(){scanf("%lld%lld", &x, &y);}
	void print() {printf("%lld %lld\n", x, y);}
} q1[111111], *q, q2[111111], a[111111], x;
long long ans, ans1, s1[111111], s2[111111], *s;
int n, Q, cl1, cl2, cl, mid, lb, bs[2], frm, to;
couple operator + (const couple & a, const couple & b)
{return couple(a.x + b.x, a.y + b.y);}
couple operator - (const couple & a, const couple & b)
{return couple(a.x - b.x, a.y - b.y);}
long long operator * (const couple & a, const couple & b)
{return a.x * b.y - a.y * b.x;}
bool operator < (const couple & a, const couple & b)
{return a.x < b.x or a.x == b.x and a.y < b.y;}
typedef bool (* func) (const couple & a, const couple & b);
bool lss(const couple & a, const couple & b) {return a < b;}
bool grt(const couple & a, const couple & b) {return b < a;}
void psh(int i)
{
	while(cl > 1 and (a[i] - q[cl]) * (q[cl] - q[cl - 1]) <= 0) cl--;
	q[++cl] = a[i];
}
bool check(int mid)
{
	return (x - q[mid]) * (q[mid + 1] - x) < 0;
}
func cmp;
void calc()
{
	lb = lower_bound(q + 1, q + 1 + cl, x, cmp) - q;
	if(lb == cl + 1 or lb == 1 or (q[lb] - x) * (x - q[lb - 1]) > 0)
	{
		bs[0] = 1; bs[1] = lb - 1;
		while(bs[0] < bs[1] - 1)
		{
			mid = (bs[0] + bs[1]) / 2;
			bs[check(mid)] = mid;
		}
		frm = check(bs[0])?bs[0]:bs[1];
		bs[0] = lb - 1; bs[1] = cl - 1;
		while(bs[0] < bs[1] - 1)
		{
			mid = (bs[0] + bs[1]) / 2;
			bs[!check(mid)] = mid;
		}
		to = check(bs[1])?bs[1]:bs[0];
		if(!frm) ans1 += 0 * (x * q[1]);
		else if(to == cl) ans1 += 0 * (q[cl1] * x);
		else ans1 += q[frm] * x + x * q[to + 1] - s[to] + s[frm - 1];
	}
}
int main()
{
	scanf("%d%d", &n, &Q);
	for(int i = 1; i <= n; i++) a[i].scan();
	sort(a + 1, a + 1 + n);
	q = q1; s = s1;
	cl = 0;
	for(int i = 1; i <= n; i++)
	{
		psh(i);
	}
	s[0] = 0;
	for(int i = 1; i < cl; i++) s[i] = s[i - 1] + q[i] * q[i + 1];
	cl1 = cl;
	q = q2; s = s2;
	cl = 0;
	for(int i = n; i >= 1; i--)
	{
		psh(i);
	}
	s[0] = 0;
	for(int i = 1; i < cl; i++) s[i] = s[i - 1] + q[i] * q[i + 1];
	cl2 = cl;
	ans = s1[cl1 - 1] + s2[cl2 - 1];
	for(int i = 1; i <= Q; i++)
	{
		x.scan();
		ans1 = ans;
		cl = cl1; q = q1; s = s1; cmp = lss;
		calc();
		cl = cl2; q = q2; s = s2; cmp = grt;
		calc();
		ans1 = abs(ans1);
		printf("%lld.%c\n", ans1 / 2, ans1 % 2 == 1?'5':'0');
	}
	fclose(stdin);
	return 0;
}
\end{lstlisting}

	\section{判断圆存在交集$NlogK$}
		传入n个圆,圆心存在cir中,半径存在radius中,nlogk判断是否存在交集
\begin{lstlisting}
int n;
double sx, sy, d;
vector<Point> cir;
vector<double> radius;

int isIntersectCircleToCircle(Point c1, double r1, Point c2, double r2)
{
	double dis = c1.distTo(c2);
	return sign(dis - (r1 + r2)) <= 0;
}

void getRange(double x, Point &c, double r, double &retl, double &retr)
{
	double tmp = sqrt(max(r * r - (c.x - x) * (c.x - x), 0.0));
	retl = c.y - tmp; retr = c.y + tmp;
}

int checkInLine(double x)
{
	double minR = INF, maxL = -INF;
	double tmpl, tmpr;
	for(int i = 0; i < n; ++ i) {
		if (sign(cir[i].x + radius[i] - x) < 0 || sign(cir[i].x - radius[i] - x) > 0) 
			return false;
		getRange(x, cir[i], radius[i], tmpl, tmpr);
		maxL = max(tmpl, maxL);
		minR = min(tmpr, minR);
		if (maxL > minR) return false;
	}
	return true;
}

int shouldGoLeft(double x)
{
	if (checkInLine(x)) return 2;
	int onL = 0, onR = 0;
	for(int i = 0; i < n; ++ i) {
		if (sign(cir[i].x + radius[i] - x) < 0) onL = 1;
		if (sign(cir[i].x - radius[i] - x) > 0) onR = 1;
	}
	if (onL && onR) return -1;
	if (onL) return 1;
	if (onR) return 0;

	double minR = INF, maxL = -INF, tmpl, tmpr;
	int idMinR, idMaxL;

	for(int i = 0; i < n; ++ i) {
		getRange(x, cir[i], radius[i], tmpl, tmpr);
		if (tmpr < minR) {
			minR = tmpr;
			idMinR = i;
		}
		if (tmpl > maxL) {
			maxL = tmpl;
			idMaxL = i;
		}
	}
	if (! isIntersectCircleToCircle(cir[idMinR], radius[idMinR], cir[idMaxL], radius[idMaxL])) 
		return -1;
	Point p1, p2;
	intersectionCircleToCircle(cir[idMinR], radius[idMinR], cir[idMaxL], radius[idMaxL], p1, p2); 
	return (p1.x < x);
}

int hasIntersectionCircles()
{
	double l = -INF, r = INF, mid;
	for(int i = 0; i < 100; ++ i) {
		mid = (l + r) * 0.5;
		int tmp = shouldGoLeft(mid);
		if (tmp < 0) return 0;
		if (tmp == 2) return 1;
		if (tmp) r = mid;
		else l = mid;
	}
	mid = (l + r) * 0.5;
	return checkInLine(mid);
}
\end{lstlisting}

	\section{圆与三角形的交面积}
	    \begin{lstlisting}
#include<cstring>
#include<cstdio>
#include<algorithm>
#include<cmath>
using namespace std;
const double eps = 1e-12, PI = acos(-1.);
int sign(double x)
{
	return x < -eps?-1:(x > eps?1:0);
}
struct triple
{
	double x, y, z;
	triple(){}
	triple(const double & _x, const double & _y, const double & _z) : x(_x), y(_y), z(_z){}
	void scan() {scanf("%lf%lf%lf", &x, &y, &z);}
	void print(char ch) {printf("%lf %lf %lf%c", x, y, z, ch);}
	double sqrlen() const {return x * x + y * y + z * z;}
	double len() const {return sqrt(sqrlen());}
} p1, p2, p3, dir, co;
double sign(triple x)
{
	return sign(x.x) == 0?(sign(x.y) == 0?sign(x.z):sign(x.y)):sign(x.x);
}
triple operator + (const triple & a, const triple & b)
{
	return triple(a.x + b.x, a.y + b.y, a.z + b.z);
}
triple operator - (const triple & a, const triple & b)
{
	return triple(a.x - b.x, a.y - b.y, a.z - b.z);
}
triple operator * (const double & a, const triple & b)
{
	return triple(a * b.x, a * b.y, a * b.z);
}
triple operator * (const triple & a, const triple & b)
{
	return triple(a.y * b.z - a.z * b.y, a.z * b.x - a.x * b.z, a.x * b.y - a.y * b.x);
}
double operator % (const triple & a, const triple & b)
{
	return a.x * b.x + a.y * b.y + a.z * b.z;
}

double t, ans, r;
double fix(double x)
{
	if(x > 1) return 1;
	else if(x < -1) return -1;
	else return x;
}
double calc(triple pa, triple pb)
{
		if(pa.len() < pb.len()) swap(pa, pb);
		if(pb.len() < eps) return 0;
		double a, b, c, B, C, sinB, cosB, sinC, cosC, S, h, theta;
		a = pb.len();
		b = pa.len();
		c = (pb - pa).len();
		cosB = fix(pb % (pb - pa) / a / c);
		sinB = fix((pb * (pb - pa)).len() / a / c);
		B = acos(cosB);
		cosC =	fix(pa % pb / a / b);
		sinC = fix((pa * pb).len() / a / b);
		C = acos(cosC);
		if(a > r)
		{
			S = C / 2 * r * r;
			h = a * b * sinC / c;
			if(h < r and B < PI / 2) S -= (acos(h / r) * r * r - h * sqrt(r * r - h * h));
		}else if(b > r)
		{
			theta = PI - B - asin(fix(sinB / r * a));
			S = .5 * a * r * sin(theta) + (C - theta) / 2 * r * r;
		}else
			S = .5 * sinC * a * b;
		//printf("%lf\n", S);
		return S;
}
int main()
{
	p1.scan();
	p2.scan();
	p3.scan();
	dir = (p2 - p1) * (p3 - p1);
	double t = dir % p1 / (dir % dir);
	co = t * dir;
	//co.print('\n');
	if(co.sqrlen() > 10000) 
	{
		printf("0\n");
		return 0;
	}
	r = sqrt(10000 - co.sqrlen());
	p1 = p1 - co;
	p2 = p2 - co;
	p3 = p3 - co;
	double ans = 0;
	ans += calc(p1, p2) * sign(p1 * p2);
	ans += calc(p2, p3) * sign(p2 * p3);
	ans += calc(p3, p1) * sign(p3 * p1);
	printf("%.10f\n", fabs(ans));
	fclose(stdin);
	return 0;
}
\end{lstlisting}

	\section{圆与圆的交并面积}
		\begin{lstlisting}
#include<cstring>
#include<cstdio>
#include<algorithm>
#include<cmath>
#include<vector>
using namespace std;
double pi = acos(-1.0), eps = 1e-12;
double sqr(const double & a) {return a * a;}
double ans[1111];
int sign(const double & a) {return a > eps?1:a < -eps?-1:0;}
int n, cnt;
struct couple
{
	double x, y;
	couple(){}
	couple(const double & _x, const double & _y) : x(_x), y(_y){}
	void scan() {scanf("%lf%lf", &x, &y);}
	double sqrlen() {return sqr(x) + sqr(y);}
	double len() {return sqrt(sqrlen());}
	couple rev() {return couple(y, -x);}
	couple zoom(const double & d) {double lambda = d / len(); return couple(lambda * x, lambda * y);}
} dvd;
double atan2(const couple & x) {return atan2(x.y, x.x);}
couple operator - (const couple & a, const couple & b)
{
	return couple(a.x - b.x, a.y - b.y);
}
couple operator + (const couple & a, const couple & b)
{
	return couple(a.x + b.x, a.y + b.y);
}
double operator * (const couple & a, const couple & b)
{
	return a.x * b.y - a.y * b.x;
}
couple operator * (const double & a, const couple & b)
{
	return couple(a * b.x, a * b.y);
}
struct circle
{
	double r; couple o;
	circle(){}
	void scan() {o.scan(); scanf("%lf", &r);}
} cir[1111];
struct arc
{
	double theta;
	int delta;
	couple p;
	arc(const double & _theta, const couple & _p, int _d) :theta(_theta), p(_p), delta(_d){}
};
bool operator < (const arc & a, const arc & b) {return a.theta < b.theta;}
vector<arc> vec;
void psh(const double t1, const couple p1, const double t2, const couple p2)
{
	if(t1 < t2)
	{
		vec.push_back(arc(t1, p1, 1));
		vec.push_back(arc(t2, p2, -1));
	}else
	{
		vec.push_back(arc(t1, p1, 1));
		vec.push_back(arc(pi, dvd, -1));
		vec.push_back(arc(-pi, dvd, 1));
		vec.push_back(arc(t2, p2, -1));
	}
}
int main()
{
	freopen("cirut.in", "r", stdin);

	scanf("%d", &n);
	for(int i = 1; i <= n; i++) cir[i].scan(), ans[i] = 0;
	for(int i = 1; i <= n; i++)
	{
		vec.clear();
		dvd = cir[i].o - couple(cir[i].r, 0);
		vec.push_back(arc(-pi, dvd, 1));
		for(int j = 1; j <= n; j++) if(j != i)
		{
			double d = (cir[j].o - cir[i].o).sqrlen();
			if(d <= sqr(cir[j].r - cir[i].r))
			{
				if(cir[i].r < cir[j].r)
					psh(-pi, dvd, pi, dvd);
			}else if(d < sqr(cir[j].r + cir[i].r))
			{
				double lambda = 0.5 * (1 + (sqr(cir[i].r) - sqr(cir[j].r))/d);
				couple cp = cir[i].o + lambda * (cir[j].o - cir[i].o);
				couple frm = cp + (cir[j].o - cir[i].o).rev().zoom(sqrt(sqr(cir[i].r) - (cp - cir[i].o).sqrlen()));
				couple to = cp - (cir[j].o - cir[i].o).rev().zoom(sqrt(sqr(cir[i].r) - (cp - cir[i].o).sqrlen()));
				psh(atan2(frm - cir[i].o), frm, atan2(to - cir[i].o), to);
			}
		}
		sort(vec.begin() + 1, vec.end());
		vec.push_back(arc(pi, dvd, -1));
		cnt = 0;
		for(int j = 0; j + 1 < vec.size(); j++)
		{
			cnt += vec[j].delta;
			double theta = vec[j + 1].theta - vec[j].theta;
			ans[cnt] += sqr(cir[i].r) * (theta - sin(theta)) * 0.5;
			ans[cnt] += vec[j].p * vec[j + 1].p * 0.5;
		}
	}
	ans[n + 1] = 0;
	for(int i = 1; i <= n; i++) printf("[%d] = %.3lf\n", i, ans[i] - ans[i + 1]);
	fclose(stdin);
	return 0;
}
\end{lstlisting}

	\section{Farmland}
		\subsection{Logic\_IU}
			\begin{lstlisting}
#include<cstdio>
#include<cstring>
#include<vector>
#include<cmath>
#include<iostream>
#include<algorithm>

using namespace std;

#define foreach(e, x) for(__typeof(x.begin()) e = x.begin(); e != x.end(); ++ e)

typedef long long LL;
typedef unsigned long long ULL;
typedef pair<int, int> PII;

const int N = 200 + 10;

struct Point
{
	double x, y;
	Point() {}
	Point(double _x, double _y) {
		x = _x; y = _y;
	}
	Point operator - (const Point &that) const {
		return Point(x - that.x, y - that.y);
	}
	double det(const Point &that) const {
		return x * that.y - y * that.x;
	}
	double alpha() {
		return atan2(y, x);
	}
	void read() {
		scanf("%lf%lf", &x, &y);
	}
};

int n, m;
int vis[N][N];
int prev[N][N];
int in[N];
Point point[N], o;
vector<int> adj[N];
vector<int> sqn;

int cmp(const int &u, const int &v)
{
	Point pu = point[u] - o, pv = point[v] - o;
	return pu.alpha() < pv.alpha();
}

double calcArea(vector<int> &ps)
{
	double area = 0;
	for(int i = 0; i < (int)ps.size(); ++ i) {
		int j = i == (int)ps.size() - 1 ? 0 : i + 1;
		area += point[ps[i]].det(point[ps[j]]);
	}
	return area;
}

void dfs(int u, int v)
{
	if (vis[u][v]) return;
	vis[u][v] = true;
	sqn.push_back(u);
	int w = prev[u][v];
	dfs(v, w);
}

void solve()
{
	cin >> n;
	for(int i = 0; i < n; ++ i) {
		adj[i].clear();
	}
	int u, v;
	for(int i = 0; i < n; ++ i) {
		cin >> u; -- u;
		point[u].read();
		int tot; cin >> tot;
		for(int j = 0; j < tot; ++ j) {
			scanf("%d", &v); -- v;
			adj[u].push_back(v);
		}
	}
	cin >> m;

	for(int i = 0; i < n; ++ i) {
		o = point[i];
		sort(adj[i].begin(), adj[i].end(), cmp);
	}

	for(int i = 0; i < n; ++ i) {
		for(int j = 0; j < (int)adj[i].size(); ++ j) {
			vis[i][adj[i][j]] = false;
			int tmp = j == 0 ? adj[i].size() - 1 : j - 1;
			prev[adj[i][j]][i] = adj[i][tmp];
		}
	}

	int ret = 0;
	for(int i = 0; i < n; ++ i) {
		for(int j = 0; j < (int)adj[i].size(); ++ j) {
			int v = adj[i][j];
			if (vis[i][v]) continue;
			sqn.clear();
			dfs(i, v);
			int flag = true;
			if (calcArea(sqn) > 0) {
				memset(in, 0, sizeof in);
				for(int k = 0; k < (int)sqn.size(); ++ k) {
					v = sqn[k];
					if (in[v]) {
						flag = false;
						break;
					}
					in[v] = true;
				}
				if ((int)sqn.size() == m) 
					ret += flag;
			} 
		}
	}
	cout << ret << endl;
}

int main()
{
	int T; cin >> T;
	for(int i = 1; i <= T; ++ i) {
		solve();
	}
	return 0;
}
\end{lstlisting}

		\subsection{tEJtM}
			\begin{lstlisting}
#include<cstring>
#include<cstdio>
#include<cmath>
#include<vector>
#include<algorithm>
#include<map>
using namespace std;
double eps = 1e-12;
struct couple
{
	int x, y;
	couple(){}
	couple(int _x, int _y) : x(_x), y(_y) {}
	void scan() {scanf("%d%d", &x, &y);}
} a[222], c;
int operator * (const couple & x, const couple & y) {return x.x * y.y - x.y * y.x;}
int operator % (const couple & x, const couple & y) {return x.x * y.x + x.y * y.y;}
couple operator - (const couple & x, const couple & y) {return couple(x.x - y.x, x.y - y.y);}
int k, n, x, y, ans, x1, Q, id[222];
bool flag, flag1;
int area;
vector<int> gen;
vector<int> vec[222];
vector<bool> f[222];
map<int, int> mp;
struct Polar
{
	couple o;
	Polar(const couple & _o) : o(_o){}
	bool operator () (const couple & a, const couple & b)
	{
		return atan2((a - o).y, (a - o).x) < atan2((b - o).y, (b - o).x);
	}
	bool operator () (int x, int y)
	{
		return this->operator () (a[x], a[y]);
	}
};
int main()
{
	freopen("01.in", "r", stdin);
	scanf("%d", &Q);
	for(int qq = 1; qq <= Q; qq++)
	{
		scanf("%d", &n);
		mp.clear();
		for(int i = 1; i <= n; i++)
		{
			scanf("%d", &id[i]);
			a[i].scan();
			vec[i].clear();
			f[i].clear();
			scanf("%d", &x);
			for(int j = 0; j < x; j++) {scanf("%d", &y); vec[i].push_back(y); f[i].push_back(true);}
		}
		scanf("%d", &k);
		for(int i = 1; i <= n; i++)	
			sort(vec[i].begin(), vec[i].end(), Polar(a[i]));
		ans = 0;
		for(int i = 1; i <= n; i++) for(int j = 0; j < vec[i].size(); j++) if(f[i][j])
		{
			x = i;
			y = j;
			gen.clear();
			for(;;){
				f[x][y] = false;
				gen.push_back(x);
				x1 = x;
				x = vec[x][y];
				y = lower_bound(vec[x].begin(), vec[x].end(), x1, Polar(a[x])) - vec[x].begin();
				y = (y + 1) % vec[x].size();
				if(x == i and y == j) break;//break when x == i and y == j, not only x == i!
			}
			if(gen.size() != k) continue;
			gen.push_back(gen.front());
			area = 0;
			for(int i = 0; i + 1 < gen.size(); i++) area += a[gen[i]] * a[gen[i + 1]];//printf("%d %d %lf\n", gen[i], gen[i + 1], a[gen[i]] * a[gen[i + 1]]);}
			if(area >= 0) continue;
			flag = true;
			for(int i = 0; i + 2 < gen.size(); i++) {for(int j = i + 1; j + 1 < gen.size(); j++) if(gen[i] == gen[j]) {flag = false; break;} if(flag == false) break;}
			ans += flag;
		}
		printf("%d\n", ans);
	}
	fclose(stdin);
	return 0;
}	
\end{lstlisting}


\chapter{三维几何}
	\section{三维几何Seraphim Version}
	    \subsection{基本操作}
			\begin{lstlisting}
//vlen(point3 P):length of vector; zero(double x):if fabs(x)<eps) return true;
double vlen(point3 p);
//平面法向量
point3 pvec(point3 s1,point3 s2,point3 s3){return det((s1-s2),(s2-s3));}
//check共线
int dots_inline(point3 p1,point3 p2,point3 p3){
    return vlen(det(p1-p2,p2-p3))<eps;}
//check共平面
int dots_onplane(point3 a,point3 b,point3 c,point3 d){
    return zero(dot(pvec(a,b,c),d-a));}
//check在线段上(end point inclusive)
int dot_online_in(point3 p,line3 l)
int dot_online_in(point3 p,point3 l1,point3 l2){return zero(vlen(det(p-l1,p-l2)))&&(l1.x-p.x)*(l2.x-p.x)<eps&&(l1.y-p.y)*(l2.y-p.y)<eps&&(l1.z-p.z)*(l2.z-p.z)<eps;    }
//check在线段上(end point exclusive)
int dot_online_ex(point3 p,line3 l)
int dot_online_ex(point3 p,point3 l1,point3 l2){ return dot_online_in(p,l1,l2)&&(!zero(p.x-l1.x)||!zero(p.y-l1.y)||!zero(p.z-l1.z))&&(!zero(p.x-l2.x)||!zero(p.y-l2.y)||!zero(p.z-l2.z));
}
//check一个点是否在三角形里(inclusive)
int dot_inplane_in(point3 p,plane3 s)
int dot_inplane_in(point3 p,point3 s1,point3 s2,point3 s3){
    return zero(vlen(det(s1-s2,s1-s3))-vlen(det(p-s1,p-s2))-
        vlen(det(p-s2,p-s3))-vlen(det(p-s3,p-s1)));
}
//check一个点是否在三角形里(exclusive)
int dot_inplane_ex(point3 p,plane3 s)
int dot_inplane_ex(point3 p,point3 s1,point3 s2,point3 s3){
    return dot_inplane_in(p,s1,s2,s3)&&vlen(det(p-s1,p-s2))>eps&&
        vlen(det(p-s2,p-s3))>eps&&vlen(det(p-s3,p-s1))>eps;
}
//check if two point and a segment in one plane have the same side
int same_side(point3 p1,point3 p2,point3 l1,point3 l2)
int same_side(point3 p1,point3 p2,line3 l){
    return dot(det(l.a-l.b,p1-l.b),det(l.a-l.b,p2-l.b))>eps;
}
//check if two point and a segment in one plane have the opposite side
int opposite_side(point3 p1,point3 p2,point3 l1,point3 l2)
int opposite_side(point3 p1,point3 p2,line3 l){
    return dot(det(l.a-l.b,p1-l.b), det(l.a-l.b,p2-l.b))<-eps;
}
//check if two point is on the same side of a plane
int same_side(point3 p1,point3 p2,point3 s1,point3 s2,point3 s3)
int same_side(point3 p1,point3 p2,plane3 s){
    return dot(pvec(s),p1-s.a)*dot(pvec(s),p2-s.a)>eps;
}
//check if two point is on the opposite side of a plane
int opposite_side(point3 p1,point3 p2,point3 s1,point3 s2,point3 s3)
int opposite_side(point3 p1,point3 p2,plane3 s){
    return dot(pvec(s),p1-s.a)*dot(pvec(s),p2-s.a)<-eps;
}
//check if two straight line is parallel
int parallel(point3 u1,point3 u2,point3 v1,point3 v2)
int parallel(line3 u,line3 v){    return vlen(det(u.a-u.b,v.a-v.b))<eps; }
//check if two plane is parallel
int parallel(point3 u1,point3 u2,point3 u3,point3 v1,point3 v2,point3 v3)
int parallel(plane3 u,plane3 v){return vlen(det(pvec(u),pvec(v)))<eps;}
//check if a plane and a line is parallel
int parallel(point3 l1,point3 l2,point3 s1,point3 s2,point3 s3)
int parallel(line3 l,plane3 s){ return zero(dot(l.a-l.b,pvec(s))); }
//check if two line is perpendicular
int perpendicular(point3 u1,point3 u2,point3 v1,point3 v2)
int perpendicular(line3 u,line3 v){return zero(dot(u.a-u.b,v.a-v.b)); }
//check if two plane is perpendicular
int perpendicular(point3 u1,point3 u2,point3 u3,point3 v1,point3 v2,point3 v3)
int perpendicular(plane3 u,plane3 v){    return zero(dot(pvec(u),pvec(v))); }
//check if plane and line is perpendicular
int perpendicular(point3 l1,point3 l2,point3 s1,point3 s2,point3 s3)
int perpendicular(line3 l,plane3 s){return vlen(det(l.a-l.b,pvec(s)))<eps;}
//check 两条线段是否有交点(end point inclusive)
int intersect_in(point3 u1,point3 u2,point3 v1,point3 v2)
int intersect_in(line3 u,line3 v){
    if (!dots_onplane(u.a,u.b,v.a,v.b)) return 0;
    if (!dots_inline(u.a,u.b,v.a)||!dots_inline(u.a,u.b,v.b))
        return !same_side(u.a,u.b,v)&&!same_side(v.a,v.b,u);
    return dot_online_in(u.a,v)||dot_online_in(u.b,v)||
dot_online_in(v.a,u)||dot_online_in(v.b,u);
}
//check 两条线段是否有交点(end point exclusive)
int intersect_ex(point3 u1,point3 u2,point3 v1,point3 v2)
int intersect_ex(line3 u,line3 v){
    return dots_onplane(u.a,u.b,v.a,v.b)&&opposite_side(u.a,u.b,v)&&
opposite_side(v.a,v.b,u);
}
//check线段和三角形是否有交点(end point and border inclusive)
int intersect_in(point3 l1,point3 l2,point3 s1,point3 s2,point3 s3)
int intersect_in(line3 l,plane3 s){
    return !same_side(l.a,l.b,s)&&!same_side(s.a,s.b,l.a,l.b,s.c)&&
        !same_side(s.b,s.c,l.a,l.b,s.a)&&!same_side(s.c,s.a,l.a,l.b,s.b);
}
//check线段和三角形是否有交点(end point and border exclusive)
int intersect_ex(point3 l1,point3 l2,point3 s1,point3 s2,point3 s3)
int intersect_ex(line3 l,plane3 s){
    return opposite_side(l.a,l.b,s)&&opposite_side(s.a,s.b,l.a,l.b,s.c)&&    opposite_side(s.b,s.c,l.a,l.b,s.a)&&opposite_side(s.c,s.a,l.a,l.b,s.b);}
//calculate the intersection of two line
//Must you should ensure they are co-plane and not parallel
point3 intersection(point3 u1,point3 u2,point3 v1,point3 v2)
point3 intersection(line3 u,line3 v){
    point3 ret=u.a; 
    double t=((u.a.x-v.a.x)*(v.a.y-v.b.y)-(u.a.y-v.a.y)*(v.a.x-v.b.x))
            /((u.a.x-u.b.x)*(v.a.y-v.b.y)-(u.a.y-u.b.y)*(v.a.x-v.b.x));
ret+=(u.b-u.a)*t;    return ret;
}
//calculate the intersection of plane and line
point3 intersection(point3 l1,point3 l2,point3 s1,point3 s2,point3 s3)
point3 intersection(line3 l,plane3 s){
    point3 ret=pvec(s);
double t=(ret.x*(s.a.x-l.a.x)+ret.y*(s.a.y-l.a.y)+ret.z*(s.a.z-l.a.z))/
        (ret.x*(l.b.x-l.a.x)+ret.y*(l.b.y-l.a.y)+ret.z*(l.b.z-l.a.z));
    ret=l.a + (l.b-l.a)*t;     return ret;
}
//calculate the intersection of two plane 
bool intersection(plane3 pl1 , plane3 pl2 , line3 &li) {
    if (parallel(pl1,pl2)) return false;    
    li.a=parallel(pl2.a,pl2.b, pl1) ? intersection(pl2.b,pl2.c, pl1.a,pl1.b,pl1.c) : intersection(pl2.a,pl2.b, pl1.a,pl1.b,pl1.c);    
    point3 fa; fa=det(pvec(pl1),pvec(pl2));
    li.b=li.a+fa;
    return true;
}
//distance from point to line
double ptoline(point3 p,point3 l1,point3 l2)
double ptoline(point3 p,line3 l){
    return vlen(det(p-l.a,l.b-l.a))/distance(l.a,l.b);}
//distance from point to plane
double ptoplane(point3 p,plane3 s){
    return fabs(dot(pvec(s),p-s.a))/vlen(pvec(s));}
double ptoplane(point3 p,point3 s1,point3 s2,point3 s3)
//distance between two line       当u,v平行时有问题
double linetoline(line3 u,line3 v){
    point3 n=det(u.a-u.b,v.a-v.b); return fabs(dot(u.a-v.a,n))/vlen(n);
}
double linetoline(point3 u1,point3 u2,point3 v1,point3 v2)
//cosine value of the angle formed by two lines
double angle_cos(line3 u,line3 v){
    return dot(u.a-u.b,v.a-v.b)/vlen(u.a-u.b)/vlen(v.a-v.b);
}
double angle_cos(point3 u1,point3 u2,point3 v1,point3 v2)
//cosine value of the angle formed by two planes
double angle_cos(plane3 u,plane3 v){
    return dot(pvec(u),pvec(v))/vlen(pvec(u))/vlen(pvec(v));}
double angle_cos(point3 u1,point3 u2,point3 u3,point3 v1,point3 v2,point3 v3)
//cosine value of the angle formed by plane and line
double angle_sin(line3 l,plane3 s){
    return dot(l.a-l.b,pvec(s))/vlen(l.a-l.b)/vlen(pvec(s));}
double angle_sin(point3 l1,point3 l2,point3 s1,point3 s2,point3 s3)
\end{lstlisting}

		\subsection{三维旋转操作}
			a点绕Ob向量,逆时针旋转弧度angle,sin(angle),cos(angle)先求出来,减少精度问题。
\begin{lstlisting}
point e1,e2,e3;
point Rotate( point a, point b, double angle ){
	b.std();//单位化,注意b不能为(0,0,0)
	e3=b; double lens=a*e3;//dot(a,e3)
	e1=a -  e3*lens; if (e1.len()>(1e-8)) e1.std(); else return a;
	e2=e1/e3; //det(e1,e3)
	double x1,y1,x,y;	
	y1=a*e1; x1=a*e2;	
	x=x1*cos(angle) - y1*sin(angle);
	y=x1*sin(angle) + y1*cos(angle);
	return e3*lens + e1*y + e2*x;
}
\end{lstlisting}

		\subsection{三维凸包随机增量}
			\begin{lstlisting}
#include<iostream>
#include<cstring>
#include<algorithm>
#include<vector>
#include<cmath>
#include<cstdio>
using namespace std;
#define SIZE(X) (int(X.size()))
#define Eps 1E-8
#define PI 3.14159265358979323846264338327950288
inline int Sign(double x) {
	return x < -Eps ? -1 : (x > Eps ? 1 : 0);
}
inline double Sqrt(double x) {
	return x < 0 ? 0 : sqrt(x);
}
struct Point {
	double x, y, z;
	Point() {
		x = y = z = 0;
	}
	Point(double x, double y, double z): x(x), y(y), z(z) {}
	bool operator <(const Point &p) const {
		return x < p.x || x == p.x && y < p.y || x == p.x && y == p.y && z < p.z;
	}
	bool operator ==(const Point &p) const {
		return Sign(x - p.x) == 0 && Sign(y - p.y) == 0 && Sign(z - p.z) == 0;
	}
	Point operator +(const Point &p) const {
		return Point(x + p.x, y + p.y, z + p.z);
	}
	Point operator -(const Point &p) const {
		return Point(x - p.x, y - p.y, z - p.z);
	}
	Point operator *(const double k) const {
		return Point(x * k, y * k, z * k);
	}
	Point operator /(const double k) const {
		return Point(x / k, y / k, z / k);
	}
	Point cross(const Point &p) const {
		return Point(y * p.z - z * p.y, z * p.x - x * p.z, x * p.y - y * p.x);
	}
	double dot(const Point &p) const {
		return x * p.x + y * p.y + z * p.z;
	}
	double norm() {
		return dot(*this);
	}
	double length() {
		return Sqrt(norm());
	}
	void read() {
		scanf("%lf%lf%lf", &x, &y, &z);
	}
	void write() {
		printf("(%.10f, %.10f, %.10f)\n", x, y, z);
	}
};
int mark[1005][1005];
Point info[1005];
int n, cnt;
double mix(const Point &a, const Point &b, const Point &c) {
	return a.dot(b.cross(c));
}
double area(int a, int b, int c) {
	return ((info[b] - info[a]).cross(info[c] - info[a])).length();
}
double volume(int a, int b, int c, int d) {
	return mix(info[b] - info[a], info[c] - info[a], info[d] - info[a]);
}
struct Face {
	int a, b, c;
	Face() {}
	Face(int a, int b, int c): a(a), b(b), c(c) {}
	int &operator [](int k) {
		if (k == 0) return a;
		if (k == 1) return b;
		return c;
	}
};
vector <Face> face;
inline void insert(int a, int b, int c) {
	face.push_back(Face(a, b, c));
}
void add(int v) {
	vector <Face> tmp;
	int a, b, c;
	cnt ++;
	for (int i = 0; i < SIZE(face); i ++) {
		a = face[i][0];
		b = face[i][1];
		c = face[i][2];
		if (Sign(volume(v, a, b, c)) < 0)
			mark[a][b] = mark[b][a] = mark[b][c] = mark[c][b] = mark[c][a] =
				mark[a][c] = cnt;
		else
			tmp.push_back(face[i]);
	}
	face = tmp;
	for (int i = 0; i < SIZE(tmp); i ++) {
		a = face[i][0];
		b = face[i][1];
		c = face[i][2];
		if (mark[a][b] == cnt) insert(b, a, v);
		if (mark[b][c] == cnt) insert(c, b, v);
		if (mark[c][a] == cnt) insert(a, c, v);
	}
}
int Find() {
	for (int i = 2; i < n; i ++) {
		Point ndir = (info[0] - info[i]).cross(info[1] - info[i]);
		if (ndir == Point())
			continue;
		swap(info[i], info[2]);
		for (int j = i + 1; j < n; j ++)
			if (Sign(volume(0, 1, 2, j)) != 0) {
				swap(info[j], info[3]);
				insert(0, 1, 2);
				insert(0, 2, 1);
				return 1;
			}
	}
	return 0;
}
int main() {
	double ans, ret;
	int Case;
	for (scanf("%d", &Case); Case; Case --) {
		scanf("%d", &n);
		for (int i = 0; i < n; i ++)
			info[i].read();
		sort(info, info + n);
		n = unique(info, info + n) - info;
		face.clear();
		random_shuffle(info, info + n);
		ans = ret = 0;
		if (Find()) {
			memset(mark, 0, sizeof(mark));
			cnt = 0;
			for (int i = 3; i < n; i ++) add(i);
			int first = face[0][0];
			for (int i = 0; i < SIZE(face); i ++) {
				ret += area(face[i][0], face[i][1], face[i][2]);
				ans += fabs(volume(first, face[i][0], face[i][1], face[i][2]));
			}
			ans /= 6;
			ret /= 2;
		}
		printf("%.3f %.3f\n", ret, ans);
	}
	return 0;
}
\end{lstlisting}

	\section{基本操作tEJtM Version}
		\begin{lstlisting}
struct triple
{
    double x, y, z;
    triple(){}
    triple(const double & _x, const doule & _y, const double & _z) : x(_x), y(_y), z(_z);
    double len2() const {return x * x + y * y + z * z;}
    double len() const {return sqrt(len2());}
    void scan() {scanf("%lf%lf%lf", &x, &y, &z);}
};
triple operator + (const triple & a, const triple & b)
{return triple(a.x + b.x, a.y + b.y, a.z + b.z);}
triple operator - (const triple & a, const triple & b)
{return triple(a.x - b.x, a.y - b.y, a.z - b.z);}
triple operator * (const double & a, const triple & b)
{return triple(a * b.x, a * b.y, a * b.z);}
triple operator * (const triple & a, const triple & b)
{return triple(a.y * b.z - a.z * b.y, a.z * b.x - a.x * b.z, a.x * b.y - a.y * b.x);}
double operator % (const triple & a, const triple & b)
{return a.x * b.x + a.y * b.y + a.z * b.z;}
struct line
{
    triple s, d;
    line(){}
    line(const triple & _s, const triple & _d) : s(_s), d(_d){}
    triple operator () const {return d;}
};
struct plane
{
    triple a, n;
    plane()
    plane(const triple & _a, const triple & _n) :a(_a), n(_n){}
    plane(const triple & _a, const triple & _b, const triple & _c)
    {a = _a; n = (b - a) * (b - c);}//valid for non-colinear _a, _b, _c.
};
const double eps = 1e-12;
int sign(const double & x)
{return x < -eps?-1:x > eps?1:0;}
//判断 相交 平行 垂直

bool parallel(const triple & a, const triple & b)//向量平行
{
    return sign((a * b).len2()) == 0;
}
bool orthogonal(const triple & a, const triple & b)//向量垂直
{
    return sign(a % b) == 0;
}
bool dis(const triple & a, const line & b)//点到直线距离 无正负
{
    return ((a - b.s) * b.d).len() / b.d.len();
}//做三角形求高法
bool perpendicular(const triple & a, const triple & b)//点到直线垂线
{
    return line(a, b.d * (b.s - a) * b.d);
}//若是点在直线上则求出来方向为0的直线
bool on(const triple & a, const line & b)//点在直线上
{
    return parallel(a - b.s, b.d);
}
bool parallel(const line & a, const line & b)//直线平行
{
    return parallel(a.d, b.d);
}//重合也算平行
bool orthogonal(const line & a, const line & b)//直线垂直
{
    return orthogonal(a.d, b.d);
}
double dis(const line & a, const line & b)//直线间最近距离 无正负
{
    triple nor = a.d * b.d;
    if(sign(nor.len2()) == 0) return dis(b.s, a);//平行直线距离
    else return nor % (a.s - b.s) / nor.len();//不平行直线最近距离
}
bool intersect(const line & a, const line & b)//直线相交:距离为0
{
    return sign(dis(a, b)) == 0;
}

triple intersection(const line & a, const line & b)//直线交点 假设不平行 且交点存在
{
    double lambda = (b.s - a.s) * b.d % (a.d * b.d) / (a.d * b.d).len2();
    return a.s + lambda * a.d;
}//若是不平行交点也不存在则求出来的是2条直线最近距离两点中第一条直线上的那个点

bool onRay(const triple & a, const line & b) //点在射线上 含
{
    return on(a, b) and sign((a - b.s) % (b.d)) >= 0;
}
bool onSeg(const triple & a, const triple & b, const triple & c)//点在线段上 含
{
    return onRay(a, line(b, c - b)) and onRay(a, line(c, b - c));
}//有了上面2个函数就可以处理射线 线段和直线相互之间存在交点的问题了

//下面是有关平面的算法
int dis(const triple & a, const plane & b)//点到平面距离 有正负
{
    return (a - b.a) % b.n;
}
int above(const triple & a, const plane & b)//above_1 on_0 under_-1
{
    return sign(dis(a, b));//和法向同向的算是平面的上面
}
bool parallel(const line & a, const plane & b)//直线和平面平行 重合也算平行
{
    return orthogonal(a.d, b.n);
}

bool parallel(const plane & a, const plane & b)//平面和平面平行
{
    return parallel(a.n, b.n);
}
bool intersect(const line & a, const plane & b)//直线和平面相交?
{
    return !parallel(a, b) or sign(a.s * b.n - b.s * b.n) == 0; //不平行或在平面内部
}
triple intersection(const line & a, const plane & b)//直线和平面的交点
{
    double lambda = b.n % (b.a - a.s) / (a.d % b.n);
    return a.s + lambda * a.d;
}
line intersection(const plane & a, const plane & b)//平面和平面的交线
{
    return line(intersection(line(a.a, a.n * b.n * a.n)), a.n * b.n);
}
\end{lstlisting}

	\section{点类$+$三维凸包$N^3+$凸包求重心}
		\begin{lstlisting}
#include<cstring>
#include<cstdio>
#include<vector>
#include<algorithm>
#include<set>
#include<string>
#include<cmath>
using namespace std;
multiset<string> st;
struct triple
{
	double x, y, z;
	double sqrlen() {return x * x + y * y + z * z;}
	double len() {return sqrt(sqrlen());}
	triple(){}
	triple(double _x, double _y, double _z) : x(_x), y(_y), z(_z){}
} a[111];
char name[111][211];
bool flag, ext[111];
int l, real[111], cnt, n, f[111][111];
struct plane
{
	int a[3];
	plane(int _x, int _y, int _z)
	{
		a[0] = _x;
		a[1] = _y;
		a[2] = _z;
	}
	int & operator [] (int x)
	{
		return a[x];
	}
};
vector<plane> surf;
triple operator * (const triple & a, const triple & b)
{
	return triple(a.y * b.z - a.z * b.y, a.z * b.x - a.x * b.z, a.x * b.y - a.y * b.x);
}
triple operator * (const double & lambda, const triple & b)
{
	return triple(lambda * b.x, lambda * b.y, lambda * b.z);
}
double operator % (const triple & a, const triple & b)
{
	return a.x * b.x + a.y * b.y + a.z * b.z;
}
triple operator - (const triple & a, const triple & b)
{
	return triple(a.x - b.x, a.y - b.y, a.z - b.z);
}
triple operator + (const triple & a, const triple & b)
{
	return triple(a.x + b.x, a.y + b.y, a.z + b.z);
}
double volume(const triple & o, int j)//volume of a tetrahedron := {a point and a triangle undersurface}
{
	return (a[surf[j][0]] - o) * (a[surf[j][1]] - o) % (a[surf[j][2]] - o);//can be negative
}
double volume(int i, int j)
{
	return volume(a[i], j);
}
double above(int i, int j) {return volume(i, j) > 0;}//point above plane
double on(int i, int j) {return volume(i, j) == 0;}//point on plane
void print(const triple & x, char ch)
{
	printf("(%lf, %lf, %lf)%c", x.x, x.y, x.z, ch);
}
double dis(const triple & o, int j)//point to plane
{
	return fabs(volume(o, j) / ((a[surf[j][1]] - a[surf[j][0]]) * (a[surf[j][2]] - a[surf[j][0]])).len());
}
int main()
{
	double ans = 0;
	for(int cv = 1; cv <= 2; cv++)
	{
		scanf("%d", &n);
		for(int i = 1; i <= n; i++)
		{
			scanf("%lf%lf%lf", &a[i].x, &a[i].y, &a[i].z);
		}
		//->degenerate checking
		flag = false;
		for(int i = 3; i <= n; i++)
		{
			if(((a[1] - a[i]) * (a[2] - a[i])).sqrlen() != 0)
			{
				swap(a[3], a[i]);
				swap(real[i], real[3]);
				for(int j = 4; j <= n; j++)	
				{
					if((a[1] - a[j]) * (a[2] - a[j]) % (a[3] - a[j]) != 0)
					{
						swap(a[4], a[j]);
						swap(real[4], real[j]);
						flag = true;
						break;
					}
				}
				break;
			}
		}
		/*if(flag == false)
		{
			//degenerate!
		}else
		{*/
		//->convex polyhedra
		memset(f, 0, sizeof(f));
		surf.clear();
		surf.push_back(plane(1, 2, 3));
		surf.push_back(plane(3, 2, 1));
		for(int i = 4; i <= n; i++)
		{
			vector<plane> tmp;
			for(int j = 0; j < surf.size(); j++)
				if(above(i, j))
				{
					for(int d = 0; d < 3; d++)
					{
						f[surf[j][d]][surf[j][(d + 2) % 3]] = i;
					}
				}else
				{
					tmp.push_back(surf[j]);
				}
			surf = tmp;
			for(int j = surf.size() - 1; j >= 0; j--)
			{
				for(int d = 0; d < 3; d++)
					if(f[surf[j][d]][surf[j][(d + 1) % 3]] == i) surf.push_back(plane(surf[j][(d + 1) % 3], surf[j][d], i));
			}
		}
		//end convex polyhedra, result := surf
		//->centre of gravity
		double svol = 0;
		triple qc(0, 0, 0);
		for(int i = 0; i < surf.size(); i++)
		{
			double vol1 = volume(1, i);
			qc = qc + (vol1 / 4) * (a[1] + a[surf[i][0]] + a[surf[i][1]] + a[surf[i][2]]);
			svol += vol1;
		}
		qc = (1 / svol) * qc;
		double mn = 1e9;
		for(int i = 0; i < surf.size(); i++)
		{
			mn = min(mn, dis(qc, i));
		}
		ans += mn;
		//end centre of gravity
		//}
	}
	printf("%.5f\n", ans);
	fclose(stdin);
	return 0;
}
\end{lstlisting}

	\section{三维旋转}
		\begin{lstlisting}
//sgu265
#include<cstring>
#include<cstdio>
#include<cmath>
#include<algorithm>
using namespace std;
const double pi = acos(-1.0);
int n, m; char ch1; bool flag;
double a[4][4], s1, s2, x, y, z, w, b[4][4], c[4][4];
double sqr(double x)
{
	return x*x;
}
int main()
{
	scanf("%d\n", &n);
	memset(b, 0, sizeof(b));
	b[0][0] = b[1][1] = b[2][2] = b[3][3] = 1;//initial matrix
	for(int i = 1; i <= n; i++)
	{
		scanf("%c", &ch1);
		if(ch1 == 'T')
		{
			//plus each coordinate by a number (x, y, z)
			scanf("%lf %lf %lf\n", &x, &y, &z);
			memset(a, 0, sizeof(a));
			a[0][0] = 1; a[3][0] = x;
			a[1][1] = 1; a[3][1] = y;
			a[2][2] = 1; a[3][2] = z;
			a[3][3] = 1;
		}else if(ch1 == 'S')
		{
			//multiply each coordinate by a number (x, y, z)
			scanf("%lf %lf %lf\n", &x, &y, &z);
			memset(a, 0, sizeof(a));
			a[0][0] = x;
			a[1][1] = y;
			a[2][2] = z;
			a[3][3] = 1;
		}else
		{
			//rotate in a clockwise about the ray from the origin through (x, y, z);
			scanf("%lf %lf %lf %lf\n", &x, &y, &z, &w);
			w = w*pi/180;
			memset(a, 0, sizeof(a));
			s1 = x*x+y*y+z*z;
			a[3][3] = 1;
			a[0][0] = ((y*y+z*z)*cos(w)+x*x)/s1;			
			a[0][1] = x*y*(1-cos(w))/s1+z*sin(w)/sqrt(s1);	
			a[0][2] = x*z*(1-cos(w))/s1-y*sin(w)/sqrt(s1);
			a[1][0] = x*y*(1-cos(w))/s1-z*sin(w)/sqrt(s1);	
			a[1][1] = ((x*x+z*z)*cos(w)+y*y)/s1;			
			a[1][2] = y*z*(1-cos(w))/s1+x*sin(w)/sqrt(s1);
			a[2][0] = x*z*(1-cos(w))/s1+y*sin(w)/sqrt(s1);	
			a[2][1] = y*z*(1-cos(w))/s1-x*sin(w)/sqrt(s1);	
			a[2][2] = ((x*x+y*y)*cos(w)+z*z)/s1;
		}
		memset(c, 0, sizeof(c));
		for(int i = 0; i < 4; i++)
			for(int j = 0; j < 4; j++)
				for(int k = 0; k < 4; k++)
					c[i][j] += b[i][k]*a[k][j];
		memcpy(b, c, sizeof(c));
	}
	scanf("%d", &m);
	for(int i = 1; i <= m; i++)
	{
		scanf("%lf%lf%lf", &x, &y, &z);//initial vector
		printf("%lf %lf %lf\n", x*b[0][0]+y*b[1][0]+z*b[2][0]+b[3][0], x*b[0][1]+y*b[1][1]+z*b[2][1]+b[3][1], x*b[0][2]+y*b[1][2]+z*b[2][2]+b[3][2]);
	}
	return 0;
}
\end{lstlisting}

	\section{最小球覆盖}
		\begin{lstlisting}
#include<iostream>
#include<cstring>
#include<algorithm>
#include<cstdio>
#include<cmath>

using namespace std;

const int eps = 1e-8;

struct Tpoint
{
	double x, y, z;
};

int npoint, nouter;

Tpoint pt[200000], outer[4],res;
double radius,tmp;
inline double dist(Tpoint p1, Tpoint p2) {
	double dx=p1.x-p2.x, dy=p1.y-p2.y, dz=p1.z-p2.z;
	return ( dx*dx + dy*dy + dz*dz );
}
inline double dot(Tpoint p1, Tpoint p2) {
	return p1.x*p2.x + p1.y*p2.y + p1.z*p2.z;
}
void ball() {
	Tpoint q[3]; double m[3][3], sol[3], L[3], det;
	int i,j;
	res.x = res.y = res.z = radius = 0;
	switch ( nouter ) {
		case 1: res=outer[0]; break;
		case 2:
				res.x=(outer[0].x+outer[1].x)/2;
				res.y=(outer[0].y+outer[1].y)/2;
				res.z=(outer[0].z+outer[1].z)/2;
				radius=dist(res, outer[0]);
				break;
		case 3:
				for (i=0; i<2; ++i ) {
					q[i].x=outer[i+1].x-outer[0].x;
					q[i].y=outer[i+1].y-outer[0].y;
					q[i].z=outer[i+1].z-outer[0].z;
				}
				for (i=0; i<2; ++i) for(j=0; j<2; ++j)
					m[i][j]=dot(q[i], q[j])*2;
				for (i=0; i<2; ++i ) sol[i]=dot(q[i], q[i]);
				if (fabs(det=m[0][0]*m[1][1]-m[0][1]*m[1][0])<eps)
					return;
				L[0]=(sol[0]*m[1][1]-sol[1]*m[0][1])/det;
				L[1]=(sol[1]*m[0][0]-sol[0]*m[1][0])/det;
				res.x=outer[0].x+q[0].x*L[0]+q[1].x*L[1];
				res.y=outer[0].y+q[0].y*L[0]+q[1].y*L[1];
				res.z=outer[0].z+q[0].z*L[0]+q[1].z*L[1];
				radius=dist(res, outer[0]);
				break;
		case 4:
				for (i=0; i<3; ++i) {
					q[i].x=outer[i+1].x-outer[0].x;
					q[i].y=outer[i+1].y-outer[0].y;
					q[i].z=outer[i+1].z-outer[0].z;
					sol[i]=dot(q[i], q[i]);
				}
				for (i=0;i<3;++i)
					for(j=0;j<3;++j) m[i][j]=dot(q[i],q[j])*2;
				det= m[0][0]*m[1][1]*m[2][2]
					+ m[0][1]*m[1][2]*m[2][0]
					+ m[0][2]*m[2][1]*m[1][0]
					- m[0][2]*m[1][1]*m[2][0]
					- m[0][1]*m[1][0]*m[2][2]
					- m[0][0]*m[1][2]*m[2][1];
				if ( fabs(det)<eps ) return;
				for (j=0; j<3; ++j) {
					for (i=0; i<3; ++i) m[i][j]=sol[i];
					L[j]=( m[0][0]*m[1][1]*m[2][2]
							+ m[0][1]*m[1][2]*m[2][0]
							+ m[0][2]*m[2][1]*m[1][0]
							- m[0][2]*m[1][1]*m[2][0]
							- m[0][1]*m[1][0]*m[2][2]
							- m[0][0]*m[1][2]*m[2][1]
						 ) / det;
					for (i=0; i<3; ++i)
						m[i][j]=dot(q[i], q[j])*2;
				}
				res=outer[0];
				for (i=0; i<3; ++i ) {
					res.x += q[i].x * L[i];
					res.y += q[i].y * L[i];
					res.z += q[i].z * L[i];
				}
				radius=dist(res, outer[0]);
	}
}
void minball(int n) {
	ball();
	//printf("(%.3lf,%.3lf,%.3lf) %.3lf\n", res.x,res.y,res.z,radius);
	if ( nouter<4 )
		for (int i=0; i<n; ++i)
			if (dist(res, pt[i])-radius>eps) {
				outer[nouter]=pt[i];
				++nouter;
				minball(i);
				--nouter;
				if (i>0) {
					Tpoint Tt = pt[i];
					memmove(&pt[1], &pt[0], sizeof(Tpoint)*i);
					pt[0]=Tt;
				}
			}
}
void solve()
{
	for (int i=0;i<npoint;i++) scanf("%lf%lf%lf",&pt[i].x,&pt[i].y,&pt[i].z);
	random_shuffle(pt, pt + npoint);
	radius=-1;
	for (int i=0;i<npoint;i++){
		if (dist(res,pt[i])-radius>eps){
			nouter=1;
			outer[0]=pt[i];
			minball(i);
		}
	}
	printf("%.5f\n",sqrt(radius));
}
int main(){
	for( ; cin >> npoint && npoint; )
		solve();
	return 0;
}
\end{lstlisting}

		
\chapter{图论}
    \section{Dijkstra}
	    求s到其他点的最短路
	    \begin{lstlisting}
int used[MAX_N], dis[MAX_N];
void dijstra(int s) {
	fill(dis, dis + N, INF); dis[s] = 0;
	priority_queue<pair<int, int> > que;
	que.push(make_pair(-dis[s], s));
	while (!que.empty()) {
		int u = que.top().second; que.pop();
		if (used[u]) continue;
		used[u] = true;
		foreach(e, E[u])
			if (dis[u] + e->w < dis[e->t]) {
				dis[e->t] = dis[u] + e->w;
				que.push(make_pair(-dis[e->t], e->t));
			}
	}
}
\end{lstlisting}


	\section{最大流}
	    iSAP算法求S到T的最大流,点数为cntN,边表存储在*E[]中
	    \begin{lstlisting}
struct Edge
{
	int t, c;
	Edge *n, *r;
} *E[MAX_V], edges[MAX_M], *totEdge;

Edge* makeEdge(int s, int t, int c)
{
	Edge *e = totEdge ++;
	e->t = t; e->c = c; e->n = E[s]; 
	return E[s] = e;
}

void addEdge(int s, int t, int c)
{
	Edge *p = makeEdge(s, t, c), *q = makeEdge(t, s, 0);
	p->r = q; q->r = p;
}

int maxflow()
{
	static int	cnt		[MAX_V];
	static int	h		[MAX_V];
	static int	que		[MAX_V];
	static int	aug		[MAX_V];
	static Edge	*cur	[MAX_V];
	static Edge	*prev	[MAX_V];
	fill(h, h + cntN, cntN);
	memset(cnt, 0, sizeof cnt);
	int qt = 0, qh = 0; h[T] = 0;
	for(que[qt ++] = T; qh < qt; ) {
		int u = que[qh ++];
		++ cnt[h[u]];
		for(Edge *e = E[u]; e; e = e->n) 
			if (e->r->c && h[e->t] == cntN) {
				h[e->t] = h[u] + 1;
				que[qt ++] = e->t;
			}
	}
	memcpy(cur, E, sizeof E);
	aug[S] = INF; Edge *e;
	int flow = 0, u = S;
	while (h[S] < cntN) {
		for(e = cur[u]; e; e = e->n)
			if (e->c && h[e->t] + 1 == h[u])
				break;
		if (e) {
			int v = e->t;
			cur[u] = prev[v] = e;
			aug[v] = min(aug[u], e->c);
			if ((u = v) == T) {
				int by = aug[T];
				while (u != S) {
					Edge *p = prev[u];
					p->c -= by;
					p->r->c += by;
					u = p->r->t;
				}
				flow += by;
			}
		} else {
			if (!-- cnt[h[u]]) return flow;
			h[u] = cntN;
			for(e = E[u]; e; e = e->n)
				if (e->c && h[u] > h[e->t] + 1)
					h[u] = h[e->t] + 1, cur[u] = e;
			++ cnt[h[u]];
			if (u != S) u = prev[u]->r->t;
		}
	}
	return flow;
}
\end{lstlisting}


	\section{上下界流}
	    上下界无源汇可行流: 不用添T—>S,判断是否流量平衡\\
上下界无源汇可行流: 添T—>S(下界0,上界oo),判断是否流量平衡\\
上下界最小流:不添T—>S先流一遍,再添T—>S(下界0,上界oo)在残图上流一遍,答案为S—>T的流量值\\
上下界最大流: 添T—>S(下界0,上界oo)流一遍,再在残图上流一遍S到T的最大流,答案为前者的S—>T的值+残图中S—>T的最大流\\

		\subsection{上下界无源汇可行流}
			\begin{lstlisting}
#include <cstdio>
#include <cstdlib>
#include <cstring>
#include <ctime>
#include <cmath>
#include <iostream>
#include <algorithm>

using namespace std;

struct {
       int x, y, down, up, what;
} a[100001];

int S, T, DS, DT, n, m, out[100001], what[100001], first[501], pre[501], way[501], len, dist[501], c[501], ans, flow[201], where[100001], next[100001], v[100001], l, cnt;

inline void makelist(int x, int y, int z, int q){
    where[++l] = y;
    v[l] = z;
    what[l] = q;
    next[l] = first[x];
    first[x] = l;
}

bool check(){
    memset(dist, 0, sizeof(dist));
    dist[DS] = 1; c[1] = DS;
    for (int k = 1, l = 1; l <= k; l++)
    {
        int m = c[l];
        if (m == DT)
        {
            len = dist[m] + 1;
            return(true);
        }
        for (int x = first[m]; x; x = next[x])
            if (v[x] && !dist[where[x]]) dist[where[x]] = dist[m] + 1, c[++k] = where[x];
    }
    return(false);
}

inline void dinic(int now){
    if (now == DT)
    {
        int Minflow = 1 << 30;
        for (; now != DS; now = pre[now]) Minflow = min(Minflow, v[way[now]]);
        ans += Minflow;
        for (now = DT; now != DS; now = pre[now])
        {
            v[way[now]] -= Minflow;
            v[way[now] ^ 1] += Minflow;
            if (!v[way[now]]) len = dist[now];
        }
        return;
    }
    for (int x = first[now]; x; x = next[x])
        if (v[x] && dist[now] + 1 == dist[where[x]])
        {
            pre[where[x]] = now;
            way[where[x]] = x;
            dinic(where[x]);
            if (dist[now] >= len) return;
            len = dist[DT] + 1;
        }
    dist[now] = -1;
}
            
int main(){
  //  freopen("194.in", "r", stdin);
  //  freopen("194.out", "w", stdout);
    scanf("%d%d", &n, &m);
    memset(flow, 0, sizeof(flow));
    for (int i = 1; i <= m; i++)
    {
        scanf("%d%d%d%d", &a[i].x, &a[i].y, &a[i].down, &a[i].up);
        flow[a[i].y] += a[i].down;
        flow[a[i].x] -= a[i].down;
    }
    cnt = 0;
    memset(first, 0, sizeof(first)); l = 1;
    S = 1; T = n; DS = 0; DT = n + 1; cnt = 0;
    for (int i = 1; i <= n; i++)
        if (flow[i] > 0) makelist(DS, i, flow[i], 0), makelist(i, DS, 0, 0), cnt += flow[i];
                    else makelist(i, DT, abs(flow[i]), 0), makelist(DT, i, 0, 0);
 //   makelist(T, S, 1 << 30, 0); makelist(S, T, 0, 0);
    for (int i = 1; i <= m; i++) makelist(a[i].x, a[i].y, a[i].up - a[i].down, i),
                                 makelist(a[i].y, a[i].x, 0, i);
    ans = 0;
    for (; check();) dinic(DS);
    if (ans != cnt) printf("NO\n");
    else
    {
        printf("YES\n");
        for (int i = 3; i <= l; i += 2) 
            if (what[i]) out[what[i]] = v[i];
        for (int i = 1; i <= m; i++) printf("%d\n", a[i].down + out[i]);
    }
}
\end{lstlisting}

		\subsection{上下界最大流}
			\begin{lstlisting}
#include <cstdio>
#include <cstdlib>
#include <cstring>
#include <ctime>
#include <cmath>
#include <iostream>
#include <algorithm>

using namespace std;

int n, m, S, T, DS, DT, a[1001], first[1501], next[100001], where[100001], v[100001], what[100001],
l, c[1501], dist[1501], len, pre[1501], way[1501], flow[1501], out[100001], tot, cnt, ans;

inline void makelist(int x, int y, int z, int q){
    where[++l] = y;
    v[l] = z;
    what[l] = q;
    next[l] = first[x];
    first[x] = l;
}

bool check(int S, int T){
    memset(dist, 0, sizeof(dist));
    c[1] = S; dist[S] = 1;
    for (int k = 1, l = 1; l <= k; l++)
    {
        int m = c[l];
        if (m == T)
        {
            len = dist[m] + 1;
            return(true);
        }
        for (int x = first[m]; x; x = next[x])
            if (v[x] && !dist[where[x]])
            {
                dist[where[x]] = dist[m] + 1; 
                c[++k] = where[x];
            }
    }
    return(false);
}

inline void dinic(int now, int S, int T){
    if (now == T)
    {
        int Minflow = 1 << 30;
        for (; now != S; now = pre[now]) Minflow = min(Minflow, v[way[now]]);
        ans += Minflow;
        for (now = T; now != S; now = pre[now]) 
        {
            v[way[now]] -= Minflow;
            v[way[now] ^ 1] += Minflow;
            if (!v[way[now]]) len = dist[now];
        }
        return;
    }
    for (int x = first[now]; x; x = next[x])
        if (v[x] && dist[where[x]] == dist[now] + 1)
        {
            pre[where[x]] = now;
            way[where[x]] = x;
            dinic(where[x], S, T);
            if (dist[now] >= len) return;
            len = dist[T] + 1;
        }
    dist[now] = -1;
}

int main(){
//    freopen("3229.in", "r", stdin);
//    freopen("3229.out", "w", stdout);
    for (;;)
    {
        if (scanf("%d%d", &n, &m) != 2) return 0;
        memset(first, 0, sizeof(first)); l = 1;
        memset(flow, 0, sizeof(flow));
        S = 0; T = n + m + 1; DS = T + 1; DT = DS + 1;
        for (int i = 1; i <= m; i++) 
        {
            scanf("%d", &a[i]);
            flow[S] -= a[i]; flow[i] += a[i];
            makelist(S, i, 1 << 30, 0); makelist(i, S, 0, 0);
        }
        tot = 0;
        for (int i = 1; i <= n; i++)
        {
            int C, D;
            scanf("%d%d", &C, &D);
            if (D) makelist(m + i, T, D, 0), makelist(T, m + i, 0, 0);
            for (int j = 1; j <= C; j++)
            {
                int idx, x, y;
                scanf("%d%d%d", &idx, &x, &y);
                idx++;
                flow[idx] -= x; flow[i + m] += x;
                out[++tot] = x;
                if (y != x) makelist(idx, i + m, y - x, tot), makelist(i + m, idx, 0, tot);
            }
        }
        cnt = 0;
        for (int i = S; i <= T; i++)
            if (flow[i] > 0) makelist(DS, i, flow[i], 0), makelist(i, DS, 0, 0), cnt += flow[i];
            else makelist(i, DT, abs(flow[i]), 0), makelist(DT, i, 0, 0);
        makelist(T, S, 1 << 30, 0); makelist(S, T, 0, 0);
        ans = 0;
        for (; check(DS, DT);) dinic(DS, DS, DT);
        if (ans != cnt) 
        {
            printf("-1\n\n");
            continue;
        }
        else
        {
            v[l] = v[l - 1] = 0;
            for (; check(S, T);) dinic(S, S, T);
            printf("%d\n", ans);
            for (int i = 3; i <= l; i += 2)
                if (what[i]) out[what[i]] += v[i];
            for (int i = 1; i <= tot; i++) printf("%d\n", out[i]);
            printf("\n");
        }
    }
}
\end{lstlisting}

		\subsection{上下界最小流}
			\begin{lstlisting}
#include <cstdio>
#include <cstdlib>
#include <cstring>
#include <ctime>
#include <cmath>
#include <iostream>
#include <algorithm>

using namespace std;

struct {
       int x, y, down, up;
} a[10001];
int out[100001], what[100001], cnt, S, T, DS, DT, l, ans, n, m, flow[101], first[201], next[100001], where[100001], v[100001], dist[201], c[201], pre[201], way[201], len;

int read(){
    char ch;
    for (ch = getchar(); ch < '0' || ch > '9'; ch = getchar());
    int cnt = 0;
    for (; ch >= '0' && ch <= '9'; ch = getchar()) cnt = cnt * 10 + ch - '0';
    return(cnt);
}

inline void makelist(int x, int y, int z, int q){
    where[++l] = y;
    v[l] = z;
    what[l] = q;
    next[l] = first[x];
    first[x] = l;
}

inline void makemap(){       
    memset(first, 0, sizeof(first)); l = 1;
    S = 1, T = n, DS = 0, DT = n + 1; cnt = 0;
    for (int i = 1; i <= n; i++)
        if (flow[i] > 0) makelist(DS, i, flow[i], 0), makelist(i, DS, 0, 0), cnt += flow[i];
        else makelist(i, DT, abs(flow[i]), 0), makelist(DT, i, 0, 0);
    for (int i = 1; i <= m; i++) makelist(a[i].x, a[i].y, a[i].up - a[i].down, i),
                                 makelist(a[i].y, a[i].x, 0, i);
}

bool check(){
    memset(dist, 0, sizeof(dist));
    dist[DS] = 1; c[1] = DS;
    for (int k = 1, l = 1; l <= k; l++)
    {
        int m = c[l];
        if (m == DT) 
        {
           len = dist[m] + 1;
           return(true);
        }
        for (int x = first[m]; x; x = next[x])
            if (v[x] && !dist[where[x]]) dist[where[x]] = dist[m] + 1, c[++k] = where[x];
    }
    return(false);
}

inline void dinic(int now){
    if (now == DT)
    {
        int Minflow = 1 << 30;
        for (; now != DS; now = pre[now]) Minflow = min(Minflow, v[way[now]]);
        ans += Minflow;
        for (now = DT; now != DS; now = pre[now])
        {
            v[way[now]] -= Minflow;
            v[way[now] ^ 1] += Minflow;
            if (!v[way[now]]) len = dist[now];
        }
        return;
    }
    for (int x = first[now]; x; x = next[x])
        if (dist[where[x]] == dist[now] + 1 && v[x])
        {
            pre[where[x]] = now;
            way[where[x]] = x;
            dinic(where[x]);
            if (dist[now] >= len) return;
            len = dist[DT] + 1;
        }
    dist[now] = -1;
}

inline void network(){
    for (; check();) dinic(DS);
}

int main(){
   // freopen("176.in", "r", stdin);
   // freopen("176.out", "w", stdout);
    scanf("%d%d", &n, &m);
    memset(flow, 0, sizeof(flow));    
    for (int i = 1; i <= m; i++) 
    {
        a[i].x = read(), a[i].y = read(), a[i].up = read();
        int status = read();
        if (status) a[i].down = a[i].up;
        else a[i].down = 0;
        flow[a[i].y] += a[i].down;
        flow[a[i].x] -= a[i].down;
    }
    makemap();
    ans = 0;
    network();
    makelist(T, S, 1 << 30, 0); makelist(S, T, 0, 0);
    network();
    if (ans != cnt) 
    {
        printf("Impossible\n");
        return 0;
    }
    printf("%d\n", v[l]);
    for (int i = 3; i <= l; i += 2)
        if (what[i]) out[what[i]] = v[i];
    for (int i = 1; i <= m; i++) 
    {
        printf("%d", a[i].down + out[i]);
        if (i != m) printf(" ");
        else printf("\n");
    }
}
\end{lstlisting}

		\subsection{上下界有源汇可行流}
			\begin{lstlisting}
#include <cstdio>
#include <cstdlib>
#include <cstring>
#include <ctime>
#include <cmath>
#include <iostream>
#include <algorithm>

using namespace std;

int test, n, m, Q, first[501], a1[201], a2[201], flow[501], next[100001], where[100001], v[100001], len,
l, dist[501], c[501], up[201][201], down[201][201], S, T, DS, DT, ans, out[201][201], pre[501], way[501];

inline void makelist(int x, int y, int z){
    where[++l] = y;
    v[l] = z;
    next[l] = first[x];
    first[x] = l;
}

bool check(){
    memset(dist, 0, sizeof(dist));
    dist[DS] = 1; c[1] = DS;
    for (int k = 1, l = 1; l <= k; l++)
    {
        int m = c[l];
        if (m == DT)
        {
            len = dist[m] + 1;
            return(true);
        }
        for (int x = first[m]; x; x = next[x])
            if (v[x] && !dist[where[x]])
            {
               dist[where[x]] = dist[m] + 1;
               c[++k] = where[x];
            }
    }
    return(false);
}

inline void dinic(int now){
    if (now == DT)
    {
        int Minflow = 1 << 30;
        for (; now != DS; now = pre[now]) Minflow = min(Minflow, v[way[now]]);
        ans += Minflow;
        for (now = DT; now != DS; now = pre[now])
        {
            v[way[now]] -= Minflow;
            v[way[now] ^ 1] += Minflow;
            if (!v[way[now]]) len = dist[now];
        }
        return;
    }
    for (int x = first[now]; x; x = next[x])
        if (v[x] && dist[now] + 1 == dist[where[x]])
        {
            pre[where[x]] = now;
            way[where[x]] = x;
            dinic(where[x]);
            if (dist[now] >= len) return;
            len = dist[DT] + 1;
        }
    dist[now] = -1;
}

int main(){
  //  freopen("2396.in", "r", stdin);
  //  freopen("2396.out", "w", stdout);
    scanf("%d", &test);
    for (int uu = 1; uu <= test; uu++)
    {
        scanf("%d%d", &n, &m);
        for (int i = 1; i <= n; i++) scanf("%d", &a1[i]);
        for (int i = 1; i <= m; i++) scanf("%d", &a2[i]);
        memset(up, 127, sizeof(up));
        memset(down, 0, sizeof(down));
        scanf("%d", &Q);
        for (int i = 1; i <= Q; i++)
        {
            int x, y, z;
            char str[2];
            scanf("%d%d%s%d", &x, &y, str, &z);
            int L1, L2, R1, R2;
            if (x == 0) L1 = 1, R1 = n;
            else L1 = R1 = x;
            if (y == 0) L2 = 1, R2 = m;
            else L2 = R2 = y;
            for (int j = L1; j <= R1; j++)
                for (int k = L2; k <= R2; k++)
                    if (str[0] == '>') down[j][k] = max(down[j][k], z + 1);
                    else if (str[0] == '<') up[j][k] = min(up[j][k], z - 1);
                    else down[j][k] = max(down[j][k], z), up[j][k] = min(up[j][k], z);
        }
        bool ok = true;
        for (int i = 1; i <= n && ok; i++)
            for (int j = 1; j <= m; j++)
                if (down[i][j] > up[i][j])
                {
                    ok = false;
                    break;
                }
        if (!ok) 
        {
           printf("IMPOSSIBLE\n");
           if (uu != test) printf("\n");
           continue;
        }
        memset(flow, 0, sizeof(flow));
        memset(first, 0, sizeof(first)); l = 1;
        S = 0; T = n + m + 1;
        for (int i = 1; i <= n; i++) flow[S] -= a1[i], flow[i] += a1[i];
        for (int i = 1; i <= m; i++) flow[i + n] -= a2[i], flow[T] += a2[i];
        for (int i = 1; i <= n; i++)
            for (int j = 1; j <= m; j++)
            {
                flow[i] -= down[i][j]; flow[j + n] += down[i][j];
                if (down[i][j] != up[i][j]) makelist(i, j + n, up[i][j] - down[i][j]),  
                                            makelist(j + n, i, 0);
            }
        DS = T + 1; DT = DS + 1;
        int cnt = 0;
        for (int i = S; i <= T; i++)
            if (flow[i] > 0) makelist(DS, i, flow[i]), makelist(i, DS, 0), cnt += flow[i];
            else if (flow[i] < 0) makelist(i, DT, abs(flow[i])), makelist(DT, i, 0);
        makelist(T, S, 1 << 30); makelist(S, T, 0);
        ans = 0;
        for (; check();) dinic(DS);
        if (ans != cnt)
        {
            printf("IMPOSSIBLE\n");
            if (uu != test) printf("\n");
            continue;
        }
        for (int i = 1; i <= n; i++)
            for (int x = first[i]; x; x = next[x])
                if (where[x] >= n + 1 && where[x] <= n + m)
                   down[i][where[x] - n] += v[x ^ 1];
        for (int i = 1; i <= n; i++)
            for (int j = 1; j <= m; j++)
            {
                printf("%d", down[i][j]);
                if (j != m) printf(" ");
                else printf("\n");
            }
        if (uu != test) printf("\n");
    }
}
\end{lstlisting}


	\section{费用流}
		\subsection{Logic\_IU$+$负费用路}
			注意图的初始化,费用和流的类型依题目而定
			\begin{lstlisting}
int flow, cost;

struct Edge
{
	int t, c, w;
	Edge *n, *r;
} *totEdge, edges[MAX_M], *E[MAX_V];

Edge* makeEdge(int s, int t, int c, int w)
{
	Edge *e = totEdge ++;
	e->t = t; e->c = c; e->w = w; e->n = E[s];
	return E[s] = e;
}

void addEdge(int s, int t, int c, int w)
{
	Edge *st = makeEdge(s, t, c, w), *ts = makeEdge(t, s, 0, -w);
	st->r = ts; ts->r = st;
}

int SPFA()
{
	static int que[MAX_V];
	static int aug[MAX_V];
	static int in[MAX_V];
	static int dist[MAX_V];
	static Edge *prev[MAX_V];
	int qh = 0, qt = 0;
	
	int u, v;
	fill(dist, dist + cntN, INF); dist[S] = 0;
	fill(in, in + cntN, 0); in[S] = true;
	que[qt ++] = S; aug[S] = INF;
	for( ; qh != qt; ) {
		u = que[qh]; qh = (qh + 1) % MAX_N;
		for(Edge *e = E[u]; e; e = e->n) {
			if (! e->c) continue;
			v = e->t;
			if (dist[v] > dist[u] + e->w) {
				dist[v] = dist[u] + e->w;
				aug[v] = min(aug[u], e->c);
				prev[v] = e;
				if (! in[v]) {
					in[v] = true;
					if (qh != qt && dist[v] <= dist[que[qh]]) {
						qh = (qh - 1 + MAX_N) % MAX_N;
						que[qh] = v;
					} else {
						que[qt] = v;
						qt = (qt + 1) % MAX_N;
					}
				}
			}
		}
		in[u] = false;
	}
	
	if (dist[T] == INF) return false;
	cost += dist[T] * aug[T];
	flow += aug[T];
	for(u = T; u != S; ) {
		prev[u]->c -= aug[T];
		prev[u]->r->c += aug[T];
		u = prev[u]->r->t;
	}
	return true;
}

int minCostFlow()
{
	flow = cost = 0;
	while(SPFA());
	return cost;
}
\end{lstlisting}

		\subsection{shytangyuan$+$ZKW}
			\begin{lstlisting}
#include <cstdio>
#include <cstdlib>
#include <cstring>
#include <ctime>
#include <cmath>
#include <iostream>
#include <algorithm>

using namespace std;

int n, m, S, T, slk[1001], dist[1001], first[1001], l, c[1000001], next[1000001], where[1000001], ll[1000001], v[1000001];
bool b[1001];
long long ans1, ans2;

inline void makelist(int x, int y, int z, int p){
    where[++l] = y; 
    ll[l] = z;
    v[l] = p;
    next[l] = first[x];
    first[x] = l;
}

inline void spfa(){    
    memset(dist, 127, sizeof(dist));
    memset(b, false, sizeof(b));
    dist[T] = 0; c[1] = T;
    for (int k = 1, l = 1; l <= k; l++)
    {
        int m = c[l];
        b[m] = false;
        for (int x = first[m]; x; x = next[x])
            if (ll[x ^ 1] && dist[m] - v[x] < dist[where[x]])
            {
               dist[where[x]] = dist[m] - v[x];
               if (!b[where[x]]) b[where[x]] = true, c[++k] = where[x];
            }
    }
}

int zkw_work(int now, int cap){
    b[now] = true;
    if (now == T)
    {
        ans1 += cap;
        ans2 += (long long)cap * dist[S];
        return(cap);
    }
    int Left = cap;
    for (int x = first[now]; x; x = next[x])
        if (ll[x] && !b[where[x]]) 
           if (dist[now] == dist[where[x]] + v[x])
           {
               int use = zkw_work(where[x], min(Left, ll[x]));
               ll[x] -= use; ll[x ^ 1] += use;
               Left -= use;
               if (!Left) return(cap);
           }
           else slk[where[x]] = min(slk[where[x]], dist[where[x]] + v[x] - dist[now]);
    return(cap - Left);
}

bool relax(){
    int Min = 1 << 30;
    for (int i = 0; i <= T; i++) 
        if (!b[i]) Min = min(Min, slk[i]);
    if (Min == 1 << 30) return(false);
    for (int i = 0; i <= T; i++)
        if (b[i]) dist[i] += Min;
    return(true);
}

inline void zkw(){
    ans1 = ans2 = 0;
    spfa();   //hint memset(dist, 0, sizeof(dist)); if all values of edges are nonnegative
    for (;;)
    {
        memset(slk, 127, sizeof(slk));
        for (;;)
        {
            memset(b, false, sizeof(b));
            if (!zkw_work(S, 1 << 30)) break;
        }
        if (!relax()) break;
    }
    printf("%I64d %I64d\n", ans1, ans2);
}

int main(){
    scanf("%d%d", &n, &m);
    S = 1; T = n;
    memset(first, 0, sizeof(first)); l = 1;
    for (int i = 1; i <= m; i++)
    {
        int x, y, z, q;
        scanf("%d%d%d%d", &x, &y, &z, &q);
        makelist(x, y, z, q); makelist(y, x, 0, -q);
    }
    zkw();
}
\end{lstlisting}


	\section{强联通分量}
		\subsection{Logic\_IU}
			N个点的图求SCC,totID为时间标记,top为栈顶,totCol为强联通分量个数,注意初始化
			\begin{lstlisting}
int totID, totCol;
int col[MAX_N], low[MAX_N], dfn[MAX_N];
int top, stack[MAX_N], instack[MAX_N];

int tarjan(int u)
{
	low[u] = dfn[u] = ++ totID;
	instack[u] = true; stack[++ top] = u;
	
	int v;
	foreach(it, adj[u]) {
		v = it->first;
		if (dfn[v] == -1)
			low[u] = min(low[u], tarjan(v));
		else if (instack[v])
			low[u] = min(low[u], dfn[v]);
	}
	
	if (low[u] == dfn[u]) {
		do {
			v = stack[top --];
			instack[v] = false;
			col[v] = totCol;
		} while(v != u);
		++ totCol;
	}
	return low[u];
}

void solve()
{
	totID = totCol = top = 0;
	fill(dfn, dfn + N, 0);
	for(int i = 0; i < N; ++ i)
		if (! dfn[i])
			tarjan(i);
}
\end{lstlisting}

		\subsection{shytangyuan$+$手写栈}
			\begin{lstlisting}
#include <cstdio>
#include <cstdlib>
#include <cstring>
#include <ctime>
#include <cmath>
#include <iostream>
#include <algorithm>

using namespace std;

int n, m, first[10001], father[10001], dfn[10001], low[10001], c[10001], pos[10001], todo[10001], 
cnt, len, next[2000001], where[2000001], l, kuai, Max, color[10001], number;
bool b[10001];

int read(){
    char ch;
    for (ch = getchar(); ch < '0' || ch > '9'; ch = getchar());
    int cnt = 0;
    for (; ch >= '0' && ch <= '9'; ch = getchar()) cnt = cnt * 10 + ch - '0';
    return(cnt);
}

inline void makelist(int x, int y){
    where[++l] = y;
    next[l] = first[x];
    first[x] = l;
}

inline void tarjan(int S){
    int now = S; todo[now] = first[now];
    for (;;)
    {
        if (!now) return;
        if (first[now] == todo[now])
        {
            b[now] = true;
            dfn[now] = low[now] = ++cnt;   
            c[++len] = now; pos[now] = len;
        }
        int x = todo[now];
        if (!x) 
        {
            if (father[now])
                low[father[now]] = min(low[father[now]], low[now]);
            int delta = -1;
            if (father[now]) ++delta;
            for (int x = first[now]; x; x = next[x])
                if (father[where[x]] == now) 
                    if (low[where[x]] >= dfn[now]) ++delta;
            Max = max(Max, delta);
            if (low[now] == dfn[now])
            {
               ++number;
               for (int i = pos[now]; i <= len; i++) color[c[i]] = number;
               len = pos[now] - 1;
            }
            now = father[now];
            continue;
        }
        todo[now] = next[todo[now]];
        if (father[now] != where[x]) 
            if (!b[where[x]])
            {
                father[where[x]] = now;
                now = where[x];
                todo[now] = first[now];
                continue;
            }
            else if (!color[where[x]]) low[now] = min(low[now], dfn[where[x]]);
    }
}
                
int main(){
   // freopen("2117.in", "r", stdin);
   // freopen("2117.out", "w", stdout);
    for (;;)
    {
        n = read(); m = read();
        if (!n && !m) return 0;
        memset(first, 0, sizeof(first));
        l = 0;
        for (int i = 1; i <= m; i++)
        {
            int x = read() + 1, y = read() + 1;
            makelist(x, y);
            makelist(y, x);
        }
        memset(dfn, 0, sizeof(dfn));
        memset(low, 0, sizeof(low));
        memset(color, 0, sizeof(color));
        memset(b, false, sizeof(b));
        memset(father, 0, sizeof(father));
        cnt = 0; len = 0;
        Max = - (1 << 30); 
        kuai = 0; number = 0;
        for (int i = 1; i <= n; i++)
            if (!b[i]) tarjan(i), ++kuai;
        printf("%d\n", kuai + Max);
    }
}
\end{lstlisting}

	
	\section{KM}
		\subsection{tEJtM}
			\begin{lstlisting}
//sgu206
#include<cstring>
#include<cstdio>
#include<algorithm>
using namespace std;
struct recmap
{
	int y, c;
	recmap *next;
} *p, ma1[1000], *id1[1001];
int n, m, l1, l2, c[1001], i, x, y, z, g[401][401], gn, ll[1001], lr[1001], rm[1001], exd, t;
bool f[1001], vl[1001];
void build(int x, int y)
{
	ma1[++l1].y = y;
	ma1[l1].next = id1[x];
	id1[x] = &ma1[l1];
}
bool dfs(int v)
{
	if(v == y) return true;
	f[v] = false;
	for(recmap *p=id1[v];p;p=p->next) if(f[p->y] and dfs(p->y))
	{
		g[(p-ma1)>>1][i-n+1] = c[(p-ma1)>>1]-z;
		return true;
	}
	return false;
}
int hgry(int v)
{

	vl[v] = true;
	for(int u = 1; u <= gn; u++)
		if(f[u]) 
		{
			if((t=ll[v]+lr[u] - g[v][u])==0)
			{
				f[u] = false;
				if(rm[u] == 0 or hgry(rm[u])) return rm[u] = v;
			}else exd = min(exd, t);
		}
	return 0;
}
void KM()
{
	memset(ll, 0x7f, sizeof(ll));
	memset(lr, 0, sizeof(lr));
	memset(rm, 0, sizeof(rm));
	for(int i = 1; i < n; i++)
	{
		for(;;)
		{
			memset(vl, false, sizeof(vl));
			memset(f, true, sizeof(f));
			exd = 0x7fffffff;
			if(hgry(i)) break;
			for(int i = 1; i < n; i++) if(vl[i]) ll[i] -= exd;
			for(int i = n; i <= m; i++) if(!f[i-n+1]) lr[i-n+1] += exd;
		}
	}
}
int main()
{
	scanf("%d%d", &n, &m);
	memset(id1, 0, sizeof(id1));
	l1 = 1;
	for(i = 1; i < n; i++)
	{
		scanf("%d%d%d", &x, &y, &c[i]);
		build(x, y);
		build(y, x);
	}
	memset(g, 0, sizeof(g));
	for(i = n; i <= m; i++)
	{
		scanf("%d%d%d", &x, &y, &z);
		c[i] = z;
		memset(f, true, sizeof(f));
		dfs(x);
	}
	gn = max(n-1, m-n+1);
	KM();
	for(int i = 1; i < n; i++) printf("%d\n", c[i]-ll[i]);
	for(int i = n; i <= m; i++) printf("%d\n", c[i] + lr[i-n+1]);
	return 0;
}
\end{lstlisting}

        \subsection{Logic\_IU}
	        求完备匹配的最大权匹配,建好的完全图用w[][]存储,点数为N
	        \begin{lstlisting}
#include <cstdio>
#include <cstdlib>
#include <algorithm>
#include <vector>
#include <cstring>
#include <string>
#include <iostream>

#define foreach(e, x) for(__typeof(x.begin()) e = x.begin(); e != x.end(); ++e)

using namespace std;

const int N = 333;
const int INF = (1 << 30);

int mat[N][N], lx[N], ly[N], vx[N], vy[N], slack[N];
int n, match[N];

bool find(int x) {
	vx[x] = 1;
	for(int i = 1; i <= n; i++) {
		if (vy[i]) {
			continue;
		}
		int temp = lx[x] + ly[i] - mat[x][i];
		if (temp == 0) {
			vy[i] = 1;
			if (match[i] == -1 || find(match[i])) {
				match[i] = x;
				return true;
			}
		} else {
			slack[i] = min(slack[i], temp);
		}
	}
	return false;
}

int KM() {
	for(int i = 1; i <= n; i++) {
		lx[i] = -INF;
		ly[i] = 0;
		match[i] = -1;
		for(int j = 1; j <= n; j++) {
			lx[i] = max(lx[i], mat[i][j]);
		}
	}
	for(int i = 1; i <= n; i++) {
		for(int j = 1; j <= n; j++) {
			slack[j] = INF;
		}
		for(; ;) {
			memset(vx, 0, sizeof(vx));
			memset(vy, 0, sizeof(vy));
			for(int j = 1; j <= n; j++) {
				slack[j] = INF;
			}
			if (find(i)) {
				break;
			}
			int delta = INF;
			for(int j = 1; j <= n; j++) {
				if (!vy[j]) {
					delta = min(delta, slack[j]);
				}
			}
			for(int j = 1; j <= n; j++) {
				if (vx[j]) {
					lx[j] -= delta;
				}
				if (vy[j]) {
					ly[j] += delta;
				} else {
					slack[j] -= delta;
				}
			}
		}
	}
	int answer = 0;
	for(int i = 1; i <= n; i++) {
		answer += mat[match[i]][i];
	}
	return answer;
}

int main() {
	while(scanf("%d", &n) != EOF) {
		for(int i = 1; i <= n; i++) {
			for(int j = 1; j <= n; j++) {
				scanf("%d", &mat[i][j]);
			}
		}
		printf("%d\n", KM());
	}
	return 0;
}
\end{lstlisting}

		\subsection{shytangyuan$+$邻接阵}
			\begin{lstlisting}
#include <cstdio>
#include <cstdlib>
#include <cstring>
#include <ctime>
#include <cmath>
#include <iostream>
#include <algorithm>

using namespace std;

const int oo = 1 << 30;
int ans, value[501][501], n, m, L[501], R[501], v[501];
bool bx[501], by[501];


bool find(int now){
    bx[now] = true;
    for (int i = 1; i <= m; i++)
        if (!by[i] && L[now] + R[i] == value[now][i])
        {
           by[i] = true;
           if (!v[i] || find(v[i]))
           {
              v[i] = now;
              return(true);
           }
        }
    return(false);
}

inline void km(){
    memset(L, 0, sizeof(L));
    memset(R, 0, sizeof(R));
    for (int i = 1; i <= n; i++)
        for (int j = 1; j <= m; j++)
            L[i] = max(L[i], value[i][j]);
    ans = 0;
    memset(v, 0, sizeof(v));
    for (int i = 1; i <= min(n, m); i++) 
        for (;;)
        {
            memset(bx, false, sizeof(bx));
            memset(by, false, sizeof(by));
            if (find(i)) break;
            int Min = oo;
            for (int j = 1; j <= n; j++) 
                if (bx[j]) 
                   for (int k = 1; k <= m; k++)
                       if (!by[k]) 
                          Min = min(Min, L[j] + R[k] - value[j][k]);
            for (int j = 1; j <= n; j++) if (bx[j]) L[j] -= Min;
            for (int j = 1; j <= m; j++) if (by[j]) R[j] += Min;
        }
    for (int i = 1; i <= n; i++)
        for (int j = 1; j <= m; j++)
            if (v[j] == i) ans += value[i][j];
    printf("%d\n", abs(ans));
}

int main(){
    scanf("%d%d", &n, &m);
    for (int i = 1; i <= n; i++) 
        for (int j = 1; j <= m; j++) scanf("%d", &value[i][j]), value[i][j] = -value[i][j]; 
    km();
    for (int i = 1; i <= n; i++) 
        for (int j = 1; j <= m; j++) 
            value[i][j] = -value[i][j];
    km();
}

/*hint 500 * 500 1.5s
can solve problems whose n != m 
must be complete graph, or should change some values of matrix to satisfy the condition*/
\end{lstlisting}

		\subsection{shytangyuan$+$链表}
			\begin{lstlisting}
#include <cstdio>
#include <cstdlib>
#include <cstring>
#include <ctime>
#include <cmath>
#include <iostream>
#include <algorithm>

using namespace std;

const int oo = 1 << 30;
int ans, first[501], l, where[250001], next[250001], value[250001], n, m, L[501], R[501], v[501];
bool bx[501], by[501];

inline void makelist(int x, int y, int z){
    where[++l] = y;
    value[l] = z;
    next[l] = first[x];
    first[x] = l;
}

bool find(int now){
    bx[now] = true;
    for (int x = first[now]; x; x = next[x])
        if (!by[where[x]] && L[now] + R[where[x]] == value[x])
        {
           by[where[x]] = true;
           if (!v[where[x]] || find(v[where[x]]))
           {
              v[where[x]] = now;
              return(true);
           }
        }
    return(false);
}

inline void km(){
    memset(L, 0, sizeof(L));
    memset(R, 0, sizeof(R));
    for (int i = 1; i <= n; i++)
        for (int x = first[i]; x; x = next[x])
            L[i] = max(L[i], value[x]);
    ans = 0;
    memset(v, 0, sizeof(v));
    for (int i = 1; i <= min(n, m); i++) 
        for (;;)
        {
            memset(bx, false, sizeof(bx));
            memset(by, false, sizeof(by));
            if (find(i)) break;
            int Min = oo;
            for (int j = 1; j <= n; j++) 
                if (bx[j]) 
                   for (int x = first[j]; x; x = next[x])
                       if (!by[where[x]]) 
                          Min = min(Min, L[j] + R[where[x]] - value[x]);
            for (int j = 1; j <= n; j++) if (bx[j]) L[j] -= Min;
            for (int j = 1; j <= m; j++) if (by[j]) R[j] += Min;
        }
    for (int i = 1; i <= n; i++)
        for (int x = first[i]; x; x = next[x])
            if (v[where[x]] == i) ans += value[x];
    printf("%d\n", abs(ans));
}

int main(){
    scanf("%d%d", &n, &m);
    memset(first, 0, sizeof(first)); l = 0;
    for (int i = 1; i <= n; i++) 
        for (int j = 1; j <= m; j++) 
        {
            int x;
            scanf("%d", &x);
            makelist(i, j, -x);
        }
    km();
    for (int i = 1; i <= l; i++) value[i] = -value[i];
    km();
}

//hint 500 * 500 2.2s
//can solve problems whose n != m
\end{lstlisting}

	\section{Hopcroft}
		\begin{lstlisting}
#include <cstdio>
#include <cstring>
#define maxn 50005
#define maxm 150005
int cx[maxn],cy[maxn],mk[maxn],q[maxn],src[maxn],pre[maxn];
int head[maxn],vtx[maxm],next[maxm],tot,n,m;
inline void Add(int a,int b)
{
	vtx[tot]=b;
	next[tot]=head[a];
	head[a]=tot++;
}
inline int Maxmatch()
{
	memset(mk,-1,sizeof(mk));
	memset(cx,-1,sizeof(cx));
	memset(cy,-1,sizeof(cy));
	for (int p=1,fl=1,h,tail;fl;++p)
	{
		fl=0;
		h=tail=0;
		for (int i=0;i<n;++i)
			if (cx[i]==-1)
				q[++tail]=i,pre[i]=-1,src[i]=i;
		for (h=1;h<=tail;++h)
		{
			int u=q[h];
			if (cx[src[u]]!=-1) continue;
			for (int pp=head[u],v=vtx[pp];pp;pp=next[pp],v=vtx[pp])
				if (mk[v]!=p)
				{
					mk[v]=p;
					q[++tail]=cy[v];
					if (cy[v]>=0)
					{
						pre[cy[v]]=u;
						src[cy[v]]=src[u];
						continue;
					}
					int d,e,t;
					for
						(--tail,fl=1,d=u,e=v;d!=-1;t=cx[d],cx[d]=e,cy[e]=d,e=t,d=pre[d]);
					break;
				}
		}
	}
	int res=0;
	for (int i=0;i<n;++i)
		res+=(cx[i]!=-1);
	return res;
}
int main()
{
	freopen("4206.in","r",stdin);
	freopen("4206.out","w",stdout);
	int P;
	scanf("%d%d%d",&n,&m,&P);
	tot=2;
	for (int i=0;i<P;++i)
	{
		int a,b;
		scanf("%d%d",&a,&b);
		--a;--b;
		Add(a,b);
	}
	printf("%d\n",Maxmatch());
	return 0;
}
\end{lstlisting}

	\section{一般图最大匹配}
		\begin{lstlisting}
#include <cstdio>
#include <cstdlib>
#include <cstring>
#include <iostream>
#include <algorithm>
using namespace std;
const int N=250;
int n;//点数(1->n)
int head;
int tail;
int Start;
int Finish;
int match[N];//表示哪个点匹配了哪个点
int Father[N];//增广路的Father
int Base[N];//该点属于哪朵花
int Q[N];
bool mark[N];
bool mat[N][N];//邻接矩阵
bool InBlossom[N];
bool in_Queue[N];

void BlossomContract(int x,int y){
	memset(mark,0,sizeof(mark));
	memset(InBlossom,0,sizeof(InBlossom));
#define pre Father[match[i]]
	int lca,i;
	for (i=x;i;i=pre) {i=Base[i]; mark[i]=true; }
	for (i=y;i;i=pre) {i=Base[i]; if (mark[i]) {lca=i; break;} }//寻找lca,一定要注意i=Base[i]
	for (i=x;Base[i]!=lca;i=pre){
		if (Base[pre]!=lca) Father[pre]=match[i];//对于BFS树中的父边是匹配边的点,Father向后跳
		InBlossom[Base[i]]=true;
		InBlossom[Base[match[i]]]=true;
	}
	for (i=y;Base[i]!=lca;i=pre){
		if (Base[pre]!=lca) Father[pre]=match[i];//同理
		InBlossom[Base[i]]=true;
		InBlossom[Base[match[i]]]=true;
	}
#undef pre
	if (Base[x]!=lca) Father[x]=y;//注意不能从lca这个奇环的关键点跳回来
	if (Base[y]!=lca) Father[y]=x;
	for (i=1;i<=n;i++)
		if (InBlossom[Base[i]]){
			Base[i]=lca;
			if (!in_Queue[i]){
				Q[++tail]=i;
				in_Queue[i]=true;//要注意如果本来连向BFS树中父结点的边是非匹配边的点,可能是没有入队的
			}
		}
}

void Change(){
	int x,y,z;
	z=Finish;
	while (z){
		y=Father[z];
		x=match[y];
		match[y]=z;
		match[z]=y;
		z=x;
	}
}

void FindAugmentPath(){
	memset(Father,0,sizeof(Father));
	memset(in_Queue,0,sizeof(in_Queue));
	for (int i=1;i<=n;i++) Base[i]=i;
	head=0; tail=1;
	Q[1]=Start;
	in_Queue[Start]=1;
	while (head!=tail){
		int x=Q[++head];
		for (int y=1;y<=n;y++)
			if (mat[x][y] && Base[x]!=Base[y] && match[x]!=y) {//无意义的边
				if ( Start==y || (match[y] && Father[match[y]]))//精髓地用Father表示该点是否
					BlossomContract(x,y);
				else if (!Father[y]){
					Father[y]=x;
					if (match[y]){
						Q[++tail]=match[y];
						in_Queue[match[y]]=true;
					}
					else{
						Finish=y;
						Change();
						return;
					}
				}
			}
	}
}

void Edmonds(){
	memset(match,0,sizeof(match));
	for (Start=1;Start<=n;Start++)
		if (match[Start]==0)
			FindAugmentPath();
}

void output(){
	memset(mark,0,sizeof(mark));
	int cnt=0;//一般图最大匹配  最大点数
	for (int i=1;i<=n;i++)
		if (match[i]) cnt++;
	printf("%d\n",cnt);
	for (int i=1;i<=n;i++)
		if (!mark[i] && match[i]){
			mark[i]=true;//i和match[i]匹配
			mark[match[i]]=true;
			printf("%d %d\n",i,match[i]);
		}
}

int main(){
	int x,y;
	scanf("%d",&n);
	memset(mat,0,sizeof(mat));
	while (scanf("%d%d",&x,&y)!=EOF)
		mat[x][y]=mat[y][x]=1;
	Edmonds();
	output();
	return 0;
}
\end{lstlisting}

	\section{无向图最小割}
		\begin{lstlisting}
const int V = 100;
#define typec int
const typec inf = 0x3f3f3f; // max of res
const typec maxw = 1000; // maximum edge weight
typec g[V][V], w[V]; //g[i][j] = g[j][i]
int a[V], v[V], na[V];
typec mincut(int n) {
	int i, j, pv, zj;
	typec best = maxw * n * n;
	for (i = 0; i < n; i++) v[i] = i; // vertex: 0 ~ n-1
	while (n > 1) {
		for (a[v[0]] = 1, i = 1; i < n; i++) {
			a[v[i]] = 0; na[i - 1] = i;
			w[i] = g[v[0]][v[i]];
		}
		for (pv = v[0], i = 1; i < n; i++) {
			for (zj = -1, j = 1; j < n; j++)
				if (!a[v[j]] && (zj < 0 || w[j] > w[zj]))
					zj = j;
			a[v[zj]] = 1;
			if (i == n - 1) {
				if (best > w[zj]) best = w[zj];
				for (i = 0; i < n; i++)
					g[v[i]][pv] = g[pv][v[i]] +=
						g[v[zj]][v[i]];
				v[zj] = v[--n];
				break;
			}
			pv = v[zj];
			for (j = 1; j < n; j++)
				if(!a[v[j]])
					w[j] += g[v[zj]][v[j]];
		}
	}
	return best;
}

int main()
{
	return 0;
}
\end{lstlisting}

	\section{最小树形图($ElogE+V^2$)}
		\begin{lstlisting}
const int N = 1111;
const int M = 1111111;
int n, m, a, b, c, x[N], y[N], z[N],
	edgeCnt, firstEdge[N], from[M], length[M], nextEdge[M],
	inEdge[N], key[M], delta[M], depth[M], child[M][2],
	parent[N], choosen[N], degree[N], queue[N];
void pass (int x) {
	if (delta[x] != 0) {
		key[child[x][0]] += delta[x];
		delta[child[x][0]] += delta[x];
		key[child[x][1]] += delta[x];
		delta[child[x][1]] += delta[x];
		delta[x] = 0;
	}
}
int merge (int x, int y) {
	if (x == 0 or y == 0) {
		return x ^ y;
	}
	if (key[x] > key[y]) {
		swap(x, y);
	}
	pass(x);
	child[x][1] = merge(child[x][1], y);
	if (depth[child[x][0]] < depth[child[x][1]]) {
		swap(child[x][0], child[x][1]);
	}
	depth[x] = depth[child[x][1]] + 1;
	return x;
}
void addEdge (int u, int v, int w) {
	from[++ edgeCnt] = u;
	length[edgeCnt] = w;
	nextEdge[edgeCnt] = firstEdge[v];
	firstEdge[v] = edgeCnt;
	key[edgeCnt] = w;
	delta[edgeCnt] = 0;
	depth[edgeCnt] = 0;
	child[edgeCnt][0] = child[edgeCnt][1] = 0;
	inEdge[v] = merge(inEdge[v], edgeCnt);
}
void deleteMin (int &r) {
	pass(r);
	r = merge(child[r][0], child[r][1]);
}
int findRoot (int u) {
	if (parent[u] != u) {
		parent[u] = findRoot(parent[u]);
	}
	return parent[u];
}
void clear () {
	edgeCnt = 0;
	depth[0] = -1;
	memset(inEdge, 0, sizeof(inEdge));
	memset(firstEdge, 0, sizeof(firstEdge));
}
int solve (int root) {
	int result = 0;
	for (int i = 0; i < n; ++ i) {
		parent[i] = i;
	}
	while (true) {
		memset(degree, 0, sizeof(degree));
		for (int i = 0; i < n; ++ i) {
			if (i == root or parent[i] != i) {
				continue;
			}
			while (findRoot(from[inEdge[i]]) == findRoot(i)) {
				deleteMin(inEdge[i]);
			}
			choosen[i] = inEdge[i];
			degree[findRoot(from[choosen[i]])] += 1;
		}
		int head = 0, tail = 0;
		for (int i = 0; i < n; ++ i) {
			if (i != root and parent[i] == i and degree[i] == 0) {
				queue[tail ++] = i;
			}
		}
		while (head < tail) {
			if (-- degree[findRoot(from[choosen[queue[head]]])] == 0) {
				queue[tail ++] = findRoot(from[choosen[queue[head]]]);
			}
			head += 1;
		}
		bool found = false;
		for (int i = 0; i < n; ++ i) {
			if (i != root and parent[i] == i and degree[i] > 0) {
				found = true;
				int j = i, temp = 0;
				do{
					j = findRoot(from[choosen[j]]);
					parent[j] = i;
					deleteMin(inEdge[j]);
					result += key[choosen[j]];
					key[inEdge[j]] -= key[choosen[j]];
					delta[inEdge[j]] -= key[choosen[j]];
					temp = merge(temp, inEdge[j]);
				} while (j != i);
				inEdge[i] = temp;
			}
		}
		if (not found) {
			break;
		}
	}
	for (int i = 0; i < n; ++ i) {
		if (i != root and parent[i] == i) {
			result += key[choosen[i]];
		}
	}
	return result;
}
\end{lstlisting}

	\section{最小树形图($V^3$)}
		\begin{lstlisting}
const int maxn=1100;
int n,m , g[maxn][maxn] , used[maxn] , pass[maxn] , eg[maxn] , more , queue[maxn];
void combine (int id , int &sum ) {
	int tot = 0 , from , i , j , k ;
	for ( ; id!=0 && !pass[ id ] ; id=eg[id] ) {
		queue[tot++]=id ; pass[id]=1;
	}
	for ( from=0; from<tot && queue[from]!=id ; from++);
	if
		( from==tot ) return ;
	more = 1 ;
	for ( i=from ; i<tot ; i++) {
		sum+=g[eg[queue[i]]][queue[i]] ;
		if ( i!=from ) {
			used[queue[i]]=1;
			for ( j = 1 ; j <= n ; j++) if ( !used[j] )
				if ( g[queue[i]][j]<g[id][j] ) g[id][j]=g[queue[i]][j] ;
		}
	}
	for ( i=1; i<=n ; i++) if ( !used[i] && i!=id ) {
		for ( j=from ; j<tot ; j++){
			k=queue[j];
			if ( g[i][id]>g[i][k]-g[eg[k]][k] ) g[i][id]=g[i][k]-g[eg[k]][k];
		}
	}
}
int mdst( int root ) { // return the total length of MDST
	int i , j , k , sum = 0 ;
	memset ( used , 0 , sizeof ( used ) ) ;
	for ( more =1; more ; ) {
		more = 0 ;
		memset (eg,0,sizeof(eg)) ;
		for ( i=1 ; i <= n ; i ++) if ( !used[i] && i!=root ) {
			for ( j=1 , k=0 ; j <= n ; j ++) if ( !used[j] && i!=j )
				if ( k==0 || g[j][i] < g[k][i] ) k=j ;
			eg[i] = k ;
		}
		memset(pass,0,sizeof(pass));
		for ( i=1; i<=n ; i++) if ( !used[i] && !pass[i] && i!= root ) combine
			( i , sum ) ;
	}
	for ( i =1; i<=n ; i ++) if ( !used[i] && i!= root ) sum+=g[eg[i]][i];
	return sum ;
}
int main(){
	freopen("input.txt","r",stdin);
	freopen("output.txt","w",stdout);
	int i,j,k,test,cases;
	cases=0;
	scanf("%d",&test);
	while (test){
		test--;
		//if (n==0) break;
		scanf("%d%d",&n,&m);
		//
		memset(g,60,sizeof(g));
		foru(i,1,n)
			foru(j,1,n) g[i][j]=1000001;
		foru(i,1,m) {
			scanf("%d%d",&j,&k);
			j++;k++;
			scanf("%d",&g[j][k]);
		}
		cases++;
		printf("Case #%d: ",cases);
		k=mdst(1);
		if (k>1000000) printf("Possums!\n");
		//===no
		else printf("%d\n",k);
	}
	return 0;
}
\end{lstlisting}


\chapter{数据结构}
	\section{KD树}
		\subsection{tEJtM$+$高维}
			\begin{lstlisting}
#include<cstring>
#include<cstdio>
#include<algorithm>
#include<cmath>
using namespace std;
int nkd = 0, n, d, m;
template<typename T> struct vector
{
	T a[2];
	T & operator [] (int x) {return a[x];}
	const T & operator [] (int x) const {return a[x];}
};
struct kd
{
   kd * s[2];
   int i;
   vector<int> x;
   vector<int> find();
} kdpool[222222], *root;
vector<int> u, v, vec[111111];
long long dis(const vector<int> & a, const vector<int> & b)
{
   long long rtn = 0;
   for(int i = 0; i < d; i++) rtn += (long long)(a[i] - b[i]) * (a[i] - b[i]);
   return rtn;
}
void bize(int le, int ri, int index, int i)
{
	if(ri <= le) return;
   if(ri == le + 1)
   {
       if(vec[le][i] > vec[ri][i]) swap(vec[le], vec[ri]);
       return;
   }
   int l = le, r = ri, x = vec[le + rand() % (ri - le + 1)][i];
   for(;;)
   {
       while(vec[l][i] < x) l++;
       while(vec[r][i] > x) r--;
       if(l < r)
       {
           swap(vec[l], vec[r]);
           l++; r--;
       }
       if(l >= r) break;
   }
   int posi = le;
   while(posi <= ri and vec[posi][i] <= x) posi++;
   if(index <= posi - 1) bize(le, posi - 1, index, i);
   if(index >= posi) bize(posi, ri, index, i);
}
kd * build(int le, int ri, int i)
{
   if(le > ri) return 0;
   kd * p = &kdpool[++nkd];
   bize(le, ri, (le + ri) / 2, i);
   p->x = vec[(le + ri) / 2];
   p->i = i;
   if(le != ri)
   {
       p->s[0] = build(le, (le + ri) / 2 - 1, (i + 1) % d);
       p->s[1] = build((le + ri) / 2 + 1, ri, (i + 1) % d);
   }else p->s[0] = p->s[1] = 0;
   return p;
}
vector<int> kd::find()
{
   vector<int> rtn(x), tmp;
   double l; int go = v[i] > x[i];
   if(s[go])
   {
       tmp = s[go]->find();
       if(dis(tmp, v) < dis(rtn, v)) rtn = tmp;
   }
   l = sqrt(dis(rtn, v));
   if(v[i] - l < x[i] and x[i] < v[i] + l and s[go ^ 1])
   {
       tmp = s[go ^ 1]->find();
       if(dis(tmp, v) < dis(rtn, v)) rtn = tmp;
   }
   return rtn;
}
int main()
{
   scanf("%d%d", &n, &d);
   for(int i = 1; i <= n; i++)
   {
       for(int j = 0; j < d; j++)
           scanf("%d", &vec[i][j]);
   }
   root = build(1, n, 0);
   scanf("%d", &m);
   for(int i = 1; i <= m; i++)
   {
       for(int j = 0; j < d; j++)
           scanf("%d", &v[j]);
       u = root->find();
       for(int j = 0; j < d; j++)
       {
           printf("%d%c", u[j], j == d - 1?'\n':' ');
       }	
   }
   return 0;
}
\end{lstlisting}


	    \subsection{Logic\_IU}
	        读入N个点,输出距离每个点的最近点。
	        \begin{lstlisting}
const int MAX_N = 100000 + 10;
const int MAX_NODE = 200000 + 10;
const LL INF = 2000000000000000020LL;

int N;

struct Point
{
    int x, y, id;
};

LL dis(const Point &a, const Point &b)
{
    return 1LL * (a.x - b.x) * (a.x - b.x) + 1LL * (a.y - b.y) * (a.y - b.y);
}

struct Node
{
    Point p;
    int maxX, minX, maxY, minY;
    int l, r, d;
    Node *ch[2];
};

LL ret;
LL ans[MAX_N];
Node *root;
Point p[MAX_N], queryPoint;
Node *totNode, nodePool[MAX_NODE];

int cmpx(const Point &a, const Point &b)
{
    return a.x < b.x;
}
int cmpy(const Point &a, const Point &b)
{
    return a.y < b.y;
}

Node* newNode(int l, int r, Point p, int deep)
{
    Node *t = totNode ++;
    t->l = l; t->r = r;
    t->p = p; t->d = deep;
    t->maxX = t->minX = p.x;
    t->maxY = t->minY = p.y;
    return t;
}

void updateInfo(Node *t, Node *p)
{
    t->maxX = max(t->maxX, p->maxX);
    t->maxY = max(t->maxY, p->maxY);
    t->minX = min(t->minX, p->minX);
    t->minY = min(t->minY, p->minY);
}

Node* build(int l, int r, int deep)
{
    if (l == r) return NULL;
    if (deep & 1) sort(p + l, p + r, cmpx);
    else sort(p + l, p + r, cmpy);
    int mid = (l + r) >> 1;
    Node *t = newNode(l, r, p[mid], deep & 1);
    if (l + 1 == r) return t;
    t->ch[0] = build(l, mid, deep + 1);
    t->ch[1] = build(mid + 1, r, deep + 1);
    if (t->ch[0]) updateInfo(t, t->ch[0]);
    if (t->ch[1]) updateInfo(t, t->ch[1]);
    return t;
}

void updateAns(Point p)
{
    ret = min(ret, dis(p, queryPoint));
}

LL calc(Node *t, LL d)
{
    LL tmp;
    if (d) {
        if (queryPoint.x >= t->minX && queryPoint.x <= t->maxX) tmp = 0;
        else tmp = min(abs(queryPoint.x - t->maxX), abs(queryPoint.x - t->minX));
    } else {
        if (queryPoint.y >= t->minY && queryPoint.y <= t->maxY) tmp = 0;
        else tmp = min(abs(queryPoint.y - t->maxY), abs(queryPoint.y - t->minY));
    }
    return tmp * tmp;
}

void query(Node *t)
{
    if (t == NULL) return;
    if (t->p.id != queryPoint.id) updateAns(t->p);
    if (t->l + 1 == t->r) return;
    LL dl = t->ch[0] ? calc(t->ch[0], t->d) : INF;
    LL dr = t->ch[1] ? calc(t->ch[1], t->d) : INF;
    if (dl < dr) {
        query(t->ch[0]);
        if (ret > dr) query(t->ch[1]);
    } else {
        query(t->ch[1]);
        if (ret > dl) query(t->ch[0]);
    }
}

void solve()
{
    scanf("%d", &N);
    for(int i = 0; i < N; ++ i) {
        scanf("%d%d", &p[i].x, &p[i].y);
        p[i].id = i;
    }
    totNode = nodePool;
    root = build(0, N, 1);
    
    for(int i = 0; i < N; ++ i) {
        queryPoint = p[i];
        ret = INF;
        query(root);
        ans[p[i].id] = ret;
    }
    for(int i = 0; i < N; ++ i)
        printf("%I64d\n", ans[i]);
}

int main()
{
    int T; scanf("%d", &T);
    for( ; T --; )
        solve();
    return 0;
}
\end{lstlisting}

	\section{后缀自动机}
		\subsection{tEJtM$+$LCA非递归Tarjan}
			\begin{lstlisting}
#include<cstdio>
#include<cstring>
#include<ctime>
int np=0, rela[2000022], n, nod[2000011], l, cl, v, p, idx[2000022],iddx[2000022], ll, Q, siz[2000022];
int ans[1000011], x, y;
struct recq
{
	int v, p;
} q[2000022];
bool f[2000022];
char c;
struct recmap
{
	int y, next, i;
} map[2000022], map1[2000011];
void build(int x, int y)
{
	map[++l].y = y;
	map[l].next = idx[x];
	idx[x] = l;
}
void build(int x, int y, int z)
{
	map1[++ll].y = y;
	map1[ll].i = z;
	map1[ll].next = iddx[x];
	iddx[x] = ll;
}
int getr(int x)
{
	int p = x, p1, p2;
	while(rela[p]!=p) p = rela[p];
	p1 = p; p = x;
	while(rela[p]!=p) {p2 = rela[p]; rela[p] = p1; p = p2;}
	return p1;
}
struct recsam
{
	int l, v;
	recsam * go[26], *p;
} *leaf, *root, polsam[2000022];
recsam * newrecsam(int _l)
{
	recsam * p = &polsam[++np];
	p->l = _l; p->v = np; p->p=0;
	memset(p->go, 0, sizeof(p->go));
	return p;
}
recsam * newrecsam(recsam & x)
{
	recsam * p = &polsam[++np];
	*p=x;
	p->v = np;
	return p;
}
int main()
{
	scanf("%d\n", &n);
	root = newrecsam(0); leaf = root;
	memset(siz, 0, sizeof(siz));
	for(int i = 1; i <= n; i++)
	{
		scanf("%c", &c); c -= 'a';
		recsam * np = newrecsam(i);
		nod[i] = np->v; siz[np->v] = 1;
		recsam * p = leaf;
		for(;p and p->go[c] == 0;p=p->p)p->go[c] = np;
		if(!p) np->p = root;
		else
		{
			recsam * q;
			if((q=p->go[c])->l == p->l+1) np->p = q;
			else
			{
				recsam * nq = newrecsam(*q);
				nq->l = p->l+1;
				nq->p = q->p;
				q->p = np->p = nq;
				for(;p and p->go[c] == q; p=p->p) p->go[c] = nq;
			}
		}
		leaf = np;
	}
	l = 0; memset(idx, 0, sizeof(idx));
	for(int i = 1; i <= np; i++) if(polsam[i].p)build(polsam[i].p->v, polsam[i].v);
	scanf("%d", &Q);
	ll = 0; memset(iddx, 0, sizeof(iddx));
	for(int i = 1; i <= Q; i++)
	{
		scanf("%d%d", &x, &y);
		build(nod[x], nod[y], i);
		build(nod[y], nod[x], i);
	}
	q[cl=1].v = 1;
	q[cl=1].p = idx[1];
	memset(f, false, sizeof(f));
	for(int i = 1; i <= np; i++) rela[i] = i;
	while(cl)
	{
		v = q[cl].v;
		q[cl].p = map[p=q[cl].p].next;
		if(p)
		{
			q[++cl].v = map[p].y;
			q[cl].p = idx[q[cl].v];
		}else
		{
			f[v] = true;
			for(int p = iddx[v]; p; p = map1[p].next)
				if(f[map1[p].y] == true) ans[map1[p].i] = getr(map1[p].y);
			siz[q[cl-1].v] += siz[v];
			rela[getr(v)] = getr(q[cl-1].v);
			cl--;
		}
	}
	for(int i = 1; i <= Q; i++) printf("%d\n", ans[i]!=1?siz[ans[i]]:0);
	return 0;
}
\end{lstlisting}

		\subsection{Logic\_IU}
			\begin{lstlisting}
struct State
{
	int val;
	State *suf, *go[26];
} *root, *last;

State statePool[MAX_N], *curState;

void extend(int w)
{
	State *p = last, *np = curState ++;
	np->val = p->val + 1;
	for( ; p && ! p->go[w]; p = p->suf)
		p->go[w] = np;
	if (! p)
		np->suf = root;
	else {
		State *q = p->go[w];
		if (q->val == p->val + 1)
			np->suf = q;
		else {
			State *nq = curState ++;
			nq->val = p->val + 1;
			memcpy(nq->go, q->go, sizeof q->go);
			nq->suf = q->suf;
			q->suf = np->suf = nq;
			for( ; p && p->go[w] == q; p = p->suf)
				p->go[w] = nq;
		}
	}
	last = np;
}
\end{lstlisting}



    \section{Splay树}
		\subsection{Logic\_IU}
			注意初始化内存池和null节点,以及根据需要修改update和relax,区间必须是1-based
			\begin{lstlisting}
const int MAX_NODE = 50000 + 10;
const int INF = 2000000000;

struct Node *null;

struct Node
{
	int rev, add;
	int val, maxv, size;
	Node *ch[2], *p;
	
	void set(Node *t, int _d) {
		ch[_d] = t;
		t->p = this;
	}
	int dir() {
		return this == p->ch[1];
	}
	void update() {
		maxv = max(max(ch[0]->maxv, ch[1]->maxv), val);
		size = ch[0]->size + ch[1]->size + 1;
	}
	void relax() {
		if (add) {
			ch[0]->appAdd(add);
			ch[1]->appAdd(add);
			add = 0;
		}
		if (rev) {
			ch[0]->appRev();
			ch[1]->appRev();
			rev = false;
		}
	}
	void appAdd(int x) {
		if (this == null) return;
		add += x;
		val += x;
		maxv += x;
	}
	void appRev() {
		if (this == null) return;
		rev ^= true;
		swap(ch[0], ch[1]);
	}
};

Node nodePool[MAX_NODE], *curNode;

Node *newNode(int val = 0)
{
	Node *t = curNode ++;
	t->maxv = t->val = val;
	t->rev = t->add = 0;
	t->size = 1;
	t->ch[0] = t->ch[1] = t->p = null;
	return t;
}

struct Splay
{
	Node *root;
	
	Splay() {
		root = newNode();
		root->set(newNode(), 0);
		root->update();
	}
	
	Splay(int *a, int N) { //sequence is 1-based
		root = build(a, 0, N + 1);
	}
	
	Node* build(int *a, int l, int r) {
		if (l > r) return null;
		int mid = l + r >> 1;
		Node *t = newNode(a[mid]);
		t->set(build(a, l, mid - 1), 0);
		t->set(build(a, mid + 1, r), 1);
		t->update();
		return t;
	}
	
	void rot(Node *t)
	{
		Node *p = t->p; int d = t->dir();
		p->relax(); t->relax();
		if (p == root) root = t;
		p->set(t->ch[! d], d);
		p->p->set(t, p->dir());
		t->set(p, ! d);
		p->update();
	}
	
	void splay(Node *t, Node *f = null)
	{
		for(t->relax(); t->p != f; ) {
			if (t->p->p == f) rot(t);
			else t->dir() == t->p->dir() ? (rot(t->p), rot(t)) : (rot(t), rot(t));
		}
		t->update();
	}
	
	Node* getKth(int k) {
		Node *t = root;
		int tmp;
		for( ; ; ) {
			t->relax();
			tmp = t->ch[0]->size + 1;
			if (tmp == k) return t;
			if (tmp < k) {
				k -= tmp;
				t = t->ch[1];
			} else
				t = t->ch[0];
		}
	}
	
	//make range[l,r] root->ch[1]->ch[0]
	//make range[x+1,x] to add something after position x
	void getRng(int l, int r) { 
		r += 2;
		Node *p = getKth(l);
		Node *q = getKth(r);
		splay(p); splay(q, p);
	}
	
	void addRng(int l, int r, int x) {
		getRng(l, r);
		root->ch[1]->ch[0]->appAdd(x);
	}
	
	void revRng(int l, int r) {
		getRng(l, r);
		root->ch[1]->ch[0]->appRev();
	}
	
	int maxvRng(int l, int r) {
		getRng(l, r);
		return root->ch[1]->ch[0]->maxv;
	}
};

void initNull()
{
	curNode = nodePool;
	null = curNode ++;
	null->size = 0;
	null->maxv = - INF;
}
\end{lstlisting} 

		\subsection{shytangyuan}
			\begin{lstlisting}
#include <cstdio>
#include <cstdlib>
#include <cstring>
#include <ctime>

using namespace std;

struct {
       int L,R,father,key,size;
} f[100001];
int root,n,Q;

inline void zig(int now){
    int x=f[now].father,y=f[x].father;
    if (y)
       if (f[y].L==x) f[y].L=now;
       else f[y].R=now;
    f[now].father=y;
    f[x].father=now;
    f[x].L=f[now].R;
    f[f[x].L].father=x;
    f[now].R=x;
    f[x].size=f[f[x].L].size+f[f[x].R].size+1;
    f[now].size=f[f[now].L].size+f[f[now].R].size+1;
}

inline void zag(int now){
    int x=f[now].father,y=f[x].father;
    if (y)
       if (f[y].L==x) f[y].L=now;
       else f[y].R=now;
    f[now].father=y;
    f[x].father=now;
    f[x].R=f[now].L;
    f[f[x].R].father=x;
    f[now].L=x;
    f[x].size=f[f[x].L].size+f[f[x].R].size+1;
    f[now].size=f[f[now].L].size+f[f[now].R].size+1;
}

inline void splay(int now){
    int x=f[now].father,y=f[x].father;
    while (x)
    {
          if (!y)
              if (f[x].L==now) zig(now);
              else zag(now);
          else if (f[y].L==x)
                  if (f[x].L==now) zig(x),zig(now);
                  else zag(now),zig(now);
               else if (f[x].L==now) zig(now),zag(now);
               else zag(x),zag(now);
          x=f[now].father;y=f[x].father;
    }
    root=now;
}
         
inline void insert(int now,int k){
    if (!root)
    {
       root=k;
       return;
    }
    if (f[k].key<=f[now].key) 
        if (!f[now].L) f[now].L=k,f[k].father=now,splay(k);
        else insert(f[now].L,k);
    else if (!f[now].R) f[now].R=k,f[k].father=now,splay(k);
         else insert(f[now].R,k);
}

inline void del(int now){
    splay(now);
    int LL=f[now].L,RR=f[now].R;
    f[now].L=f[now].R=f[now].key=f[now].father=f[now].size=0;
    f[LL].father=f[RR].father=0;
    if (!LL && !RR) root=0;
    else if (!LL) root=RR;
    else if (!RR) root=LL;
    else
    {
        root=RR;
        while (f[RR].L) RR=f[RR].L;
        f[RR].L=LL;
        f[LL].father=RR;
        splay(LL);
    }
}

int findkth(int now,int k){
    if (k==f[f[now].L].size+1) return(now);

int main(){
    scanf("%d",&n);
    root=0;
    for (int i=1;i<=n;i++)
        scanf("%d",&f[i].key),f[i].size=1,insert(root,i);
    scanf("%d",&Q);
    for (;Q--;)
    {
        int type;
        scanf("%d",&type);
        if (type==1)
        {
            int x,y;
            scanf("%d%d",&x,&y);
            del(x);
            f[x].key=y;f[x].size=1;
            insert(root,x);
        }
        else
        {
            int x;
            scanf("%d",&x);
            printf("%d\n",f[findkth(root,x)].key);
        }
    }
}
\end{lstlisting}


	\section{动态树}
		根据需求修改Node中的relax和update函数,修改access,以及Node的构造函数,注意初始化内存池和null节点
		\begin{lstlisting}
struct Node
{
	Node *ch[2], *p;
	int isroot;
	bool dir();
	void set(Node*, bool);
	void update();
	void relax();
} *null;

void rot(Node *t)
{
	Node *p = t->p; bool d = t->dir();
	p->relax(); t->relax();
	p->set(t->ch[! d], d);
	if (p->isroot) t->p = p->p, swap(p->isroot, t->isroot);
	else p->p->set(t, p->dir());
	t->set(p, ! d);
	p->update();
}

void Splay(Node *t)
{
	for(t->relax(); ! t->isroot; ) {
		if (t->p->isroot) rot(t);
		else t->dir() == t->p->dir() ? (rot(t->p), rot(t)) : (rot(t), rot(t));
	}
	t->update();
}

void Access(Node *t)
{
	for(Node *s = null; t != null; s = t, t = t->p) {
		Splay(t);
		t->ch[1]->isroot = true;
		s->isroot = false;
		t->ch[1] = s;
		t->update();
	}
}
bool Node::dir()
{
	return this == p->ch[1];
}
void Node::set(Node *t, bool _d)
{
	ch[_d] = t; t->p = this;
}
void Node::Update()
{

}
void Node::Relax()
{
	if (this == Null) return;

}
\end{lstlisting}



	\section{二叉堆}
		双射堆,ind[v]表示标号为v的节点在堆中的位置
		\begin{lstlisting}
const int MAX_V = 100000 + 10;
struct Heap
{
	int tot;
	int a[MAX_V], h[MAX_V], ind[MAX_V];
	void exchange(int i, int j) {
		swap(h[i], h[j]);
		swap(ind[h[i]], ind[h[j]]);
	}
	inline int val(int x) {
		return a[h[x]];
	}
	void fixUp(int x) {
		if (x / 2 && val(x / 2) < val(x)) 
			exchange(x, x / 2), fixUp(x / 2);
	}
	void fixDown(int x) {
		int p = x * 2; if (p > tot) return;
		if (p < tot && val(p + 1) > val(p)) ++ p;
		if (val(p) > val(x))
			exchange(p, x), fixDown(p);
	}
	void Update(int i, int x) {
		a[i] = x;
		fixUp(ind[i]);
		fixDown(ind[i]);
	}
	int top() {
		return h[1];
	}
	void pop() {
		exchange(1, tot);
		-- tot;
		fixDown(1);
	}
	void insert(int i, int x) {
		++ tot;
		h[tot] = i;
		ind[i] = tot;
		a[i] = x;
		fixUp(tot);
	}
} H;
\end{lstlisting}


	\section{左偏树}
		没写delete操作,注意初始化内存池和null节点
		\begin{lstlisting}
struct Node
{
	int dis, val;
	Node *ch[2];
} *null;

Node* merge(Node *u, Node *v)
{
	if (u == null) return v;
	if (v == null) return u;
	if (u->val < v->val) swap(u, v);
	u->ch[1] = merge(u->ch[1], v);
	if (u->ch[1]->dis > u->ch[0]->dis)
		swap(u->ch[1], u->ch[0]);
	u->dis = u->ch[1]->dis + 1;
	return u;
}

Node* newNode(int w)
{
	Node *t = totNode ++;
	t->ch[0] = t->ch[1] = null;
	t->val = w; t->dis = 0;
	return t;
}
\end{lstlisting}


	\section{Treap}
		包含build, insert和erase,执行时注意初始化内存池和null节点
		\begin{lstlisting}
namespace treap {
	struct node {
		node *left, *right;
		int key;
		int size, count, aux;
		inline node(int _aux) {
			left = right = 0;
			key = size = count = 0;
			aux = _aux;
		}
		inline void update() {
			this->size = this->left->size + this->count + this->right->size;
		}
	};
 
	node *null;
 
	inline void print(node *&x) {
		if (x == null) {
			return;
		}
		print(x->left);
		printf("%d ", x->key);
		print(x->right);
	}
 
	inline node* create(int key) {
		node *x = new node(rand() % INT_MAX);
		x->key = key;
		x->count = x->size = 1;
		x->left = x->right = null;
		return x;
	}
 
	inline void left_rotate(node *&x) {
		node *y = x->right;
		x->right = y->left;
		y->left = x;
		x->update();
		y->update();
		x = y;
	}
 
	inline void right_rotate(node *&x) {
		node *y = x->left;
		x->left = y->right;
		y->right = x;
		x->update();
		y->update();
		x = y;
	}
 
	inline void insert(node *&x, int key) {
		if (x == null) {
			x = create(key);
			return;
		}
		if (x->key == key) {
			x->count++;
		} else if (x->key > key) {
			insert(x->left, key);
			if (x->left->aux < x->aux) {
				right_rotate(x);
			}
		} else {
			insert(x->right, key);
			if (x->right->aux < x->aux) {
				left_rotate(x);
			}
		}
		x->update();
	}
 
	inline void erase(node *&x, int key) {
		if (x == null) {
			return;
		}
		if (x->key == key) {
			if (x->count > 1) {
				x->count--;
			} else if (x->left == null && x->right == null) {
				delete(x);
				x = null;
				return;
			} else if (x->left->aux < x->right->aux) {
				right_rotate(x);
				erase(x->right, key);
			} else {
				left_rotate(x);
				erase(x->left, key);
			}
		} else if (x->key > key) {
			erase(x->left, key);
		} else {
			erase(x->right, key);
		}
		x->update();
	}
 
	inline void prepare() {
		null = new node(INT_MAX);
	}
}
\end{lstlisting}


	\section{线段树}
		包含建树和区间操作样例,没有写具体操作
		\begin{lstlisting}
struct Tree
{
	int l, r;
	Tree *ch[2];
	Tree() {}
	Tree(int _l, int _r, int *sqn) {
		l = _l; r = _r;
		if (l + 1 == r)
			return;
		int mid = l + r >> 1;
		ch[0] = new Tree(l, mid, sqn);
		ch[1] = new Tree(mid, r, sqn);
	}
	
	void insert(int p, int x) {
		if (p < l || p >= r)
			return;
		//some operations
		if (l + 1 == r)
			return;
		ch[0]->insert(p, x);
		ch[1]->insert(p, x);
	}
	
	int query(int _l, int _r, int x) {
		if (_r <= l || _l >= r)
			return 0;
		if (_l <= l && _r >= r)
			// return information in [l, r)
		//merge ch[0]->query(_l, _r, x), ch[1]->query(_l, _r, x) and return
	}
};
\end{lstlisting}


	\section{轻重链剖分}
		包含BFS剖分过程,线段树部分视题目而定
		\begin{lstlisting}
struct Tree()
{
	
};

int father[MAX_N], size[MAX_N], depth[MAX_N];
int bfsOrd[MAX_N], pathId[MAX_N], ordInPath[MAX_N], sqn[MAX_N];
Tree *root[MAX_N];

void doBfs(int s)
{
	int *que = bfsOrd;
	int qh = 0, qt = 0;
	father[s] = -1; depth[s] = 0;
	
	for(que[qt ++] = s; qh < qt; ) {
		int u = que[qh ++];
		foreach(iter, adj[u]) {
			int v = *iter;
			if (v == father[u])
				continue;
			father[v] = u;
			depth[v] = depth[u] + 1;
			que[qt ++] = v;
		}
	}
}

void doSplit()
{
	for(int i = N - 1; i >= 0; -- i) {
		int u = bfsOrd[i];
		size[u] = 1;
		foreach(iter, adj[u]) {
			int v = *iter;
			if (v == father[u])
				continue;
			size[u] += size[v];
		}
	}
	
	memset(pathId, -1, sizeof pathId);
	for(int i = 0; i < N; ++ i) {
		int top = bfsOrd[i];
		if (pathId[top] != -1)
			continue;
		
		int cnt = 0;
		for(int u = top; u != -1; ) {
			sqn[cnt] = val[u];
			ordInPath[u] = cnt;
			pathId[u] = top;
			++ cnt;
			
			int next = -1;
			foreach(iter, adj[u]) {
				int v = *iter;
				if (v == father[u])
					continue;
				if (next < 0 || size[next] < size[v])
					next = v;
			}
			u = next;
		}
		
		root[top] = new Tree(0, cnt, sqn);
	}
}

void prepare()
{
	doBfs(0);
	doSplit();
}

\end{lstlisting}


	
	\section{KMP}
		\begin{lstlisting}
vector<int> KMP()
{
	vector<int> ans;
	nxt[0] = -1;
	nxt[1] = 0;
	for(int i = 2; i <= m; i++)
	{
		nxt[i] = nxt[i - 1];
		while(nxt[i] >= 0 and st[i] != st[nxt[i] + 1])
			nxt[i] = nxt[nxt[i]];
		nxt[i]++;
	}
	for(int i = 1, p = 1; i <= n; i++)
	{
		while(p and str1[i] != st[p])
			p = nxt[p - 1] + 1;
		p++;
		if(p == m + 1) p = nxt[m] + 1, ans.push_back(i - m);
	}
	return ans;
}
\end{lstlisting}

	
	\section{扩展KMP}
		传入字符串s和长度N,next[i]=LCP(s, s[i..N-1])
		\begin{lstlisting}
void z(char *s, int *next, int N)
{
	int j = 0, k = 1;
	while (j + 1 < N && s[j] == s[j + 1]) ++ j;
	next[0] = N - 1; next[1] = j;
	for(int i = 2; i < N; ++ i) {
		int far = k + next[k] - 1, L = next[i - k];
		if (L < far - i + 1) next[i] = L;
		else {
			j = max(0, far - i + 1);
			while (i + j < N && s[j] == s[i + j]) ++ j;
			next[i] = j; k = i;
		}
	}
}
\end{lstlisting}

	
	\section{Manacher}
		\begin{lstlisting}
void manacher(char *text, int length) {
    palindrome[0] = 1;
    for (int i = 1, j = 0; i < length; ++i) {
        if (j + palindrome[j] <= i) {
            palindrome[i] = 0;
        } else {
            palindrome[i] = std::min(palindrome[(j << 1) - i], j + palindrome[j] - i);
        }
        while (i - palindrome[i] >= 0 && i + palindrome[i] < length 
                && text[i - palindrome[i]] == text[i + palindrome[i]]) {
            palindrome[i]++;
        }
        if (i + palindrome[i] > j + palindrome[j]) {
            j = i;
        }
    }
}
\end{lstlisting}


	\section{AC自动机}
		\subsection{Logic\_IU}
			包含建trie和构造自动机的过程
			\begin{lstlisting}

struct acNode
{
    int id;
    acNode *ch[26], *fail;
} *totNode, *root, nodePool[MAX_V];

acNode* newNode()
{
    acNode *now = totNode ++;
    now->id = 0; now->fail = 0;
    memset(now->ch, 0, sizeof now->ch);
    return now;
}

void acInsert(char *c, int id)
{
    acNode *cur = root;
    while (*c) {
        int p = *c - 'A'; //change the index
        if (! cur->ch[p]) cur->ch[p] = newNode();
        cur = cur->ch[p];
        ++ c;
    }
    cur->id = id;
}

void getFail()
{
    acNode *cur;
    queue<acNode*> Q;
    for(int i = 0; i < 26; ++ i)
        if (root->ch[i]) {
            root->ch[i]->fail = root;
            Q.push(root->ch[i]);
        } else root->ch[i] = root;
    while (! Q.empty()) {
        cur = Q.front(); Q.pop();
        for(int i = 0; i < 26; ++ i)
            if (cur->ch[i]) {
                cur->ch[i]->fail = cur->fail->ch[i];
                Q.push(cur->ch[i]);
            } else cur->ch[i] = cur->fail->ch[i];
    }
}
\end{lstlisting}

		\subsection{shytangyuan}
			\begin{lstlisting}
#include <cstdio>
#include <cstdlib>
#include <cstring>
#include <ctime>

using namespace std;

int a[100001][27],fail[100001],last[100001],c[100001],l,father[100001],type[100001];
char can[1001];

inline void maketrie(){
    memset(a,0,sizeof(a));
    memset(type,0,sizeof(type));
    memset(last,0,sizeof(last));
    int n=strlen(can),now=0;
    l=0;
    for (int i=0;i<n;i++) 
    {
        if (!a[now][can[i]-'A']) a[now][can[i]-'A']=++l,type[l]=can[i]-'A',father[l]=now;
        now=a[now][can[i]-'A'];
    }
    last[now]=1;
}

inline void makefail(){
    memset(fail,255,sizeof(fail));
    fail[0]=0;
    int k=0;
    for (int i=0;i<=25;i++) 
       if (a[0][i]) fail[a[0][i]]=0,c[++k]=a[0][i];
    for (int l=1;l<=k;l++)
    {
        int m=c[l];
        if (fail[m]==-1)
        {
           int p=father[m];
           while (p && !a[fail[p]][type[m]]) p=fail[p];
           fail[m]=a[fail[p]][type[m]];
           last[m]+=last[fail[m]];
        }
        for (int i=0;i<=25;i++) 
            if (a[m][i]) c[++k]=a[m][i];
    }
}
         
int main(){ 
    scanf("%s",can);
    maketrie();
    makefail();     
}
\end{lstlisting}


	\section{后缀数组}
		\subsection{Logic\_IU}
			对于串a求SA,长度为N,M为元素值范围,height[i]=LCP(suf[rank[i]],suf[rank[i]-1])
			\begin{lstlisting}
const int MAX_N = 1000000 + 10;

int rank[MAX_N], height[MAX_N];

int cmp(int *x, int a, int b, int d)
{
	return x[a] == x[b] && x[a + d] == x[b + d];
}

void doubling(int *a, int N, int M)
{
	static int sRank[MAX_N], tmpA[MAX_N], tmpB[MAX_N];
	int *x = tmpA, *y = tmpB;
	for(int i = 0; i < M; ++ i) sRank[i] = 0;
	for(int i = 0; i < N; ++ i) ++ sRank[x[i] = a[i]];
	for(int i = 1; i < M; ++ i) sRank[i] += sRank[i - 1];
	for(int i = N - 1; i >= 0; -- i) sa[-- sRank[x[i]]] = i;
	
	for(int d = 1, p = 0; p < N; M = p, d <<= 1) {
		p = 0; for(int i = N - d; i < N; ++ i) y[p ++] = i;
		for(int i = 0; i < N; ++ i)
			if (sa[i] >= d) y[p ++] = sa[i] - d;
		for(int i = 0; i < M; ++ i) sRank[i] = 0;
		for(int i = 0; i < N; ++ i) ++ sRank[x[i]];
		for(int i = 1; i < M; ++ i) sRank[i] += sRank[i - 1];
		for(int i = N - 1; i >= 0; -- i) sa[-- sRank[x[y[i]]]] = y[i];
		swap(x, y); x[sa[0]] = 0; p = 1;
		for(int i = 1; i < N; ++ i)
			x[sa[i]] = cmp(y, sa[i], sa[i - 1], d) ? p - 1 : p ++;
	}
}

void calcHeight()
{
	for(int i = 0; i < N; ++ i) rank[sa[i]] = i;
	int cur = 0;
	for(int i = 0; i < N; ++ i)
		if (rank[i]) {
			if (cur) cur --;
			for( ; a[i + cur] == a[sa[rank[i] - 1] + cur]; ++ cur);
			height[rank[i]] = cur;
		}
}
\end{lstlisting}

		\subsection{shytangyuan}
			\begin{lstlisting}
#include <cstdio>
#include <cstdlib>
#include <cstring>
#include <ctime>

using namespace std;

int test,n,SA[100001],c[100001],Rank[100001],tmp[100001],H[100001],f[100001];
char can[50001];

int main(){
    //freopen("1.txt","r",stdin);
    //freopen("2.txt","w",stdout);
    scanf("%d",&test);                              
    for (test;test;test--)
    {
             scanf("%s\n",&can);
             n=strlen(can);
             memset(f,0,sizeof(f));
             for (int i=1;i<=n;i++) f[i]=int (can[i-1]);
             memset(c,0,sizeof(c));
             for (int i=1;i<=n;i++) c[f[i]]++;
             for (int i=1;i<=1000;i++) c[i]+=c[i-1];
             for (int i=n;i;i--) SA[c[f[i]]--]=i;
             Rank[SA[1]]=1;
             for (int i=2;i<=n;i++) 
                if (f[SA[i]]==f[SA[i-1]]) Rank[SA[i]]=Rank[SA[i-1]];
                else Rank[SA[i]]=Rank[SA[i-1]]+1;
             for (int L=1;L<=n;L+=L)
             {
                 if (Rank[SA[n]]==n) break;
                 memset(c,0,sizeof(c));
                 for (int i=1;i<=n;i++) c[Rank[L+i]]++;
                 for (int i=1;i<=n;i++) c[i]+=c[i-1];
                 for (int i=n;i;i--) tmp[c[Rank[L+i]]--]=i;
                 memset(c,0,sizeof(c));
                 for (int i=1;i<=n;i++) c[Rank[i]]++;
                 for (int i=1;i<=n;i++) c[i]+=c[i-1];
                 for (int i=n;i;i--) SA[c[Rank[tmp[i]]]--]=tmp[i];
                 tmp[SA[1]]=1;
                 for (int i=2;i<=n;i++)
                    if ((Rank[SA[i]]==Rank[SA[i-1]])&&(Rank[SA[i]+L]==Rank[SA[i-1]+L]))
                        tmp[SA[i]]=tmp[SA[i-1]];
                    else tmp[SA[i]]=tmp[SA[i-1]]+1;
                 for (int i=1;i<=n;i++) Rank[i]=tmp[i];
             }
    int p=0;
    for (int i=1;i<=n;i++)
    {
        int j=SA[Rank[i]-1];
        p-=1;
        if (p<0) p=0;
        while ((f[i+p]==f[j+p])) p++;
        H[i]=p;
    }
    int ans=0;
    for (int i=1;i<=n;i++)
        ans+=n-SA[i]+1-H[i];
    printf("%d\n",ans);
    }
}
\end{lstlisting}

		\subsection{DC3}
			\begin{lstlisting}
//DC3 待排序的字符串放在r 数组中,从r[0]到r[n-1],长度为n,且最大值小于m。
//约定除r[n-1]外所有的r[i]都大于0, r[n-1]=0。
//函数结束后,结果放在sa 数组中,从sa[0]到sa[n-1]。
//r必须开长度乘3
#define maxn 10000
#define F(x) ((x)/3+((x)%3==1?0:tb))
#define G(x) ((x)<tb?(x)*3+1:((x)-tb)*3+2)
int wa[maxn],wb[maxn],wv[maxn],wss[maxn];
int s[maxn*3],sa[maxn*3];
int c0(int *r,int a,int b)
{
	return r[a]==r[b]&&r[a+1]==r[b+1]&&r[a+2]==r[b+2];
}
int c12(int k,int *r,int a,int b)
{
	if(k==2) return r[a]<r[b]||r[a]==r[b]&&c12(1,r,a+1,b+1);
	else return r[a]<r[b]||r[a]==r[b]&&wv[a+1]<wv[b+1];
}
void sort(int *r,int *a,int *b,int n,int m)
{
	int i;
	for(i=0;i<n;i++) wv[i]=r[a[i]];
	for(i=0;i<m;i++) wss[i]=0;
	for(i=0;i<n;i++) wss[wv[i]]++;
	for(i=1;i<m;i++) wss[i]+=wss[i-1];
	for(i=n-1;i>=0;i--) b[--wss[wv[i]]]=a[i];
}
void dc3(int *r,int *sa,int n,int m)
{
	int i,j,*rn=r+n,*san=sa+n,ta=0,tb=(n+1)/3,tbc=0,p;
	r[n]=r[n+1]=0;
	for(i=0;i<n;i++)
		if(i%3!=0) wa[tbc++]=i;
	sort(r+2,wa,wb,tbc,m);
	sort(r+1,wb,wa,tbc,m);
	sort(r,wa,wb,tbc,m);
	for(p=1,rn[F(wb[0])]=0,i=1;i<tbc;i++)
		rn[F(wb[i])]=c0(r,wb[i-1],wb[i])?p-1:p++;
	if (p<tbc) dc3(rn,san,tbc,p);
	else for (i=0;i<tbc;i++) san[rn[i]]=i;
	for (i=0;i<tbc;i++)
		if(san[i]<tb) wb[ta++]=san[i]*3;
	if(n%3==1) wb[ta++]=n-1;
	sort(r,wb,wa,ta,m);
	for(i=0;i<tbc;i++)
		wv[wb[i]=G(san[i])]=i;
	for(i=0,j=0,p=0;i<ta && j<tbc;p++)
		sa[p]=c12(wb[j]%3,r,wa[i],wb[j])?wa[i++]:wb[j++];
	for(;i<ta;p++) sa[p]=wa[i++];
	for(;j<tbc;p++) sa[p]=wb[j++];
}
int main(){
	int n,m=0;
	scanf("%d",&n);
	for (int i=0;i<n;i++) scanf("%d",&s[i]),s[i]++,m=max(s[i]+1,m);
	printf("%d\n",m);
	s[n++]=0;
	dc3(s,sa,n,m);
	for (int i=0;i<n;i++) printf("%d ",sa[i]);printf("\n");
}
\end{lstlisting}


\chapter{杂}
	\section{$m^2logn$求线性递推第n项}
		\begin{lstlisting}
// given first m a[i] and coef c[i] (0-based),
// calc a[n] mod p in O(m*m*log(n)).
// a[n] = sum(c[m-i]*a[n-i]), i = 1...m
// i.e. a[m] = sum(c[i]*a[i]), i = 0...m-1
int linear_recurrence(LL n, int m, int a[], int c[], int p) {
	LL v[M] = {1 % p}, u[M<<1], msk = !!n;
	for(LL i = n; i > 1; i >>= 1) msk <<= 1;
	for(LL x = 0; msk; msk >>= 1, x <<= 1) {
		fill_n(u, m<<1, 0);
		int b = !!(n & msk); x |= b;
		if(x < m) u[x] = 1 % p;
		else {
			for(int i = 0; i < m; ++i)
				for(int j = 0, t = i+b; j < m; ++j, ++t)
					u[t] = (u[t]+v[i]*v[j]) % p;
			for(int i = (m<<1)-1; i >= m; --i)
				for(int j = 0, t = i-m; j < m; ++j, ++t)
					u[t] = (u[t]+c[j]*u[i]) % p;
		}
		copy(u, u+m, v);
	}
	int an = 0;
	for(int i = 0; i < m; ++i) an = (an+v[i]*a[i]) % p;
	return an;
}
\end{lstlisting}

	
	\section{FFT}
		\begin{lstlisting}
#include<cstdio>
#include<cmath>
#include<cstring>
#include<algorithm>
using namespace std;
double pi = 2 * acos(0.0);
const int pw2lim = 65536;
struct recmap
{
	int y, next;
} map[100011];
struct recq
{
	int p, v;
} st[50011];
int idx[50001], ii[5000000], li1, n, K, x, y, z, ans[50001], l, l2, ele, siz[50001], q[50001], cl, dis[50001], fa[50001];
int L, go[2 * pw2lim], d, nrec, rec[50001], v, p;
bool f[50001], isprime[50001];
int mnpw2(int x)
{
	int rtn = 1;
	while(x) 
	{
		x >>= 1;
		rtn <<= 1;
	}
	return rtn;
}
struct vector
{
	int siz, *a;
	int & operator [] (int x) 
	{
		return a[x];
	}
	vector()
	{
		a = ii + li1;
	}
};
struct C
{
	double real, imag;
	C(const double & _real, const double & _imag) : real(_real), imag(_imag){}
	C(){}
	C(const double & x){real = x; imag = 0;}
	void print(char c)
	{
		printf("(%f, %f)%c", real, imag, c);
	}
} a[pw2lim], b[pw2lim], res1[pw2lim], res2[pw2lim], res[pw2lim], tmp[pw2lim], unit[pw2lim], temp;
const C operator * (const C & a, const C & b)
{
	return C(a.real * b.real - a.imag * b.imag, a.real * b.imag + a.imag * b.real);
}
const C operator + (const C & a, const C & b)
{
	return C(a.real + b.real, a.imag + b.imag);
}
const C operator - (const C & a, const C & b)
{
	return C(a.real - b.real, a.imag - b.imag);
}
void build(int x, int y)
{
	map[++l].y = y;
	map[l].next = idx[x];
	idx[x] = l;
}
void dft(C * sour, C * dest)
{
	for(int i = 0; i < L; i++) tmp[go[i + L]] = sour[i];
	for(int l = 1; l < L; l <<= 1)
	{
		l2 = l << 1;
		ele = pw2lim / l2;
		for(int i = 0; i < L; i += l2)
			for(int j = 0; j < l; j++)
			{
				temp = tmp[i + l + j] * unit[ele * j];
				tmp[i + l + j] = tmp[i + j] - temp;
				tmp[i + j] = tmp[i + j] + temp;
			}
	}
	for(int i = 0; i < L; i++) dest[i] = tmp[i];
}
void fft(vector p1, vector p2)
{
	L = mnpw2(p1.siz + p2.siz - 1);
	for(int i = 0; i < p1.siz; i++) a[i] = p1[i];
	for(int i = p1.siz; i < L; i++) a[i] = 0;
	for(int i = 0; i < p2.siz; i++) b[i] = p2[i];
	for(int i = p2.siz; i < L; i++) b[i] = 0;
	dft(a, res1);
	dft(b, res2);
	for(int i = 0; i < L; i++) res[i] = res1[i] * res2[i];
	dft(res, res1);
	for(int i = 1; i <= nrec and rec[i] < p1.siz + p2.siz - 1; i++) ans[i] += (int)(res1[L - rec[i]].real / L + 0.5);
	//ans[i](0<=i<L) = res1[(L - i) % L].
}
vector dvcq(int v)
{
	if(siz[v] == 2)
	{
		vector vec;
		vec.siz = 2;
		vec[0] = 0;
		vec[1] = 1;
		li1 += vec.siz;
		return vec;
	}
	int u=v, sum=1;
	for(int p=idx[v]; p;)
	{
		if(siz[y=map[p].y] > siz[u]/2)
		{
			siz[y] += siz[u] -= siz[y];
			u = y;
			p = idx[y];
		}else p = map[p].next;
	}
	int bak, biz;
	vector p1, p2;
	for(int p = idx[u]; p; p = map[p].next)
	{
		sum += siz[map[p].y];
		if(sum >= siz[u]/2)
		{
			biz = siz[u];
			bak = idx[u];
			idx[u] = map[p].next;
			siz[u] -= sum-1;
			p1 = dvcq(u);
			idx[u] = bak;
			bak = map[p].next;
			map[p].next = 0;
			siz[u] = sum;
			p2 = dvcq(u);
			siz[u] = biz;
			map[p].next = bak;
			break;
		}
	}
	fft(p1, p2);
	vector vec;
	fa[v] = 0;
	q[cl=1] = v;
	dis[v] = 0;
	vec[0] = 0;
	vec.siz = 0;
	for(int op = 1; op <= cl; op++)
		for(int p = idx[u = q[op]], y; p; p = map[p].next)
			if((y=map[p].y) != fa[u])
			{
				vec[dis[y] = dis[fa[q[++cl] = y] = u] + 1] ++;
				vec.siz = max(vec.siz, dis[y]);
			}
	vec.siz++;
	li1 += vec.siz;
	return vec;
}
int main()
{
	go[1] = 0;
	for(int i = 2; i <= pw2lim; i <<= 1)
	{
		for(int j = 0; j < i / 2; j++)
		{
			go[i + j] = go[i / 2 + j] * 2;
		}
		for(int j = i / 2; j < i; j++)
		{
			go[i + j] = go[j] * 2 + 1;
		}
	}
	unit[0] = 1;
	unit[32768] = -1;
	unit[16384] = C(0, 1);
	for(int i = 8192; i >= 1; i /= 2)
	{
		unit[i] = C((unit[0].real + unit[i * 2].real) / 2, (unit[0].imag + unit[i * 2].imag) / 2);
		double len = sqrt(unit[i].imag * unit[i].imag + unit[i].real * unit[i].real);
		unit[i].imag *= 1/len; unit[i].real *= 1/len;
	}
	for(int i = 1; i <= 65536; i++)
	{
		if(i - (i & -i))
		{
			unit[i] = C(1, 0);
			for(int x = i; x; x -= x & -x)
			{
				unit[i] = unit[i] * unit[x & -x];
			}
		}
	}//求单位复根
	memset(isprime, true, sizeof(isprime));
	nrec = 0; rec[0] = 0x7fffffff;
	for(int i = 2; i <= 50000; i++)
	{
		if(isprime[i]) rec[++nrec] = i;
		for(int j = 1; j <= nrec and i * rec[j] <= 50000 and i % rec[j - 1]; j++)
			isprime[i * rec[j]] = false;
	}
	scanf("%d", &n);
	memset(idx, 0, sizeof(idx));
	l = 1;
	for(int i = 1; i < n; i++)
	{
		scanf("%d%d", &x, &y);
		build(x, y);
		build(y, x);
	}
	memset(siz, 0, sizeof(siz));
	memset(f, true, sizeof(f));
	f[1] = false;
	st[cl = 1].v = 1;
	st[1].p = idx[1];
	while(cl)
	{
		v = st[cl].v;
		st[cl].p = map[p = st[cl].p].next;
		if(p)
		{
			if(f[map[p].y])
			{
				st[++cl].v = map[p].y;
				st[cl].p = idx[map[p].y];
				f[map[p].y] = false;
			}
		}else
		{
			siz[v]++;
			siz[st[cl - 1].v] += siz[v];
			cl--;
		}
	}
	li1 = 0;
	memset(ans, 0, sizeof(ans));
	dvcq(1);
	long long tot = 0;
	for(int i = 1; i <= nrec; i++) tot += ans[i];;
	printf("%lf\n", (double)tot / ((long long)n * (n - 1) / 2));	
	fclose(stdin);
	fclose(stdout);
	return 0;
}
\end{lstlisting}


	\section{中国剩余定理}
	    包括扩展欧几里得,求逆元,和保证除数互质条件下的CRT
	    包括扩展欧几里得,求逆元,和保证除数互质条件下的CRT
\begin{lstlisting}
LL x, y;
void exGcd(LL a, LL b)
{
	if (b == 0) {
		x = 1;
		y = 0;
		return;
	}
	exGcd(b, a % b);
	LL k = y;
	y = x - a / b * y;
	x = k;
}

LL inversion(LL a, LL b)
{
	exGcd(a, b);
	return (x % b + b) % b;
}

LL CRT(vector<LL> m, vector<LL> a)
{
	int N = m.size();
	LL M = 1, ret = 0;
	for(int i = 0; i < N; ++ i)
		M *= m[i];
	
	for(int i = 0; i < N; ++ i) {
		ret = (ret + (M / m[i]) * a[i] % M * inversion(M / m[i], m[i])) % M;
	}
	return ret;
}
\end{lstlisting}



	\section{Pollard's Rho$+$Miller-Rabbin}
	    大数分解和素性判断
	    \begin{lstlisting}
typedef long long LL;

LL modMul(LL a, LL b, LL P)
{
	LL ret = 0;
	for( ; a; a >>= 1) {
		if (a & 1) {
			ret += b;
			if (ret >= P) ret -= P;
		}
		b <<= 1;
		if (b >= P) b -= P;
	}
	return ret;
}

LL modPow(LL a, LL u, LL P)
{
	LL ret = 1;
	for( ; u; u >>= 1, a = modMul(a, a, P))
		if (u & 1) ret = modMul(ret, a, P);
	return ret;
}

int millerRabin(LL N)
{
	if (N == 2) return true;
	LL t = 0, u = N - 1, x, y, a;
	for( ; ! (u & 1); ++ t, u >>= 1) ;
	for(int k = 0; k < 10; ++ k) {
		a = rand() % (N - 2) + 2;
		x = modPow(a, u, N);
		for(int i = 0; i < t; ++ i, x = y) {
			y = modMul(x, x, N);
			if (y == 1 && x > 1 && x < N - 1) return false;
		}
		if (x != 1) return false;
	}
	return true;
}

LL gcd(LL a, LL b)
{
	return ! b ? a : gcd(b, a % b);
}

LL pollardRho(LL N)
{
	LL i = 1, x = rand() % N;
	LL y = x, k = 2, d = 1;
	do {
		d = gcd(x - y + N, N);
		if (d != 1 && d != N) return d;
		if (++ i == k) y = x, k <<= 1;
		x = (modMul(x, x, N) - 1 + N) % N;
	} while (y != x);
	return N;
}

void getFactor(LL N)
{
	if (N < 2) return;
	if (millerRabin(N)) {
		//do some operations
		return;
	}
	LL x = pollardRho(N);
	getFactor(x);
	getFactor(N / x);
}
\end{lstlisting}

	
	\section{素数判定(long long内确定性算法)}
		\begin{lstlisting}
int strong_pseudo_primetest(long long n,int base) {
	long long n2=n-1,res;
	int s; s=0;
	while(n2%2==0) n2>>=1,s++;
	res=powmod(base,n2,n);
	if((res==1)||(res==n-1)) return 1;
	s--;
	while(s>=0) {
		res=mulmod(res,res,n);
		if(res==n-1) return 1;
		s--;
	}
	return 0; // n is not a strong pseudo prime
}
int isprime(long long n) {
	if(n<2) return 0;
	if(n<4) return 1;
	if(strong_pseudo_primetest(n,2)==0) return 0;
	if(strong_pseudo_primetest(n,3)==0) return 0;
	if(n<1373653LL) return 1;
	if(strong_pseudo_primetest(n,5)==0) return 0;
	if(n<25326001LL) return 1;
	if(strong_pseudo_primetest(n,7)==0) return 0;
	if(n==3215031751LL) return 0;
	if(n<25000000000LL) return 1;
	if(strong_pseudo_primetest(n,11)==0) return 0;
	if(n<2152302898747LL) return 1;
	if(strong_pseudo_primetest(n,13)==0) return 0;
	if(n<3474749660383LL) return 1;
	if(strong_pseudo_primetest(n,17)==0) return 0;
	if(n<341550071728321LL) return 1;
	if(strong_pseudo_primetest(n,19)==0) return 0;
	if(strong_pseudo_primetest(n,23)==0) return 0;
	if(strong_pseudo_primetest(n,29)==0) return 0;
	if(strong_pseudo_primetest(n,31)==0) return 0;
	if(strong_pseudo_primetest(n,37)==0) return 0;
	return 1;
}
\end{lstlisting}

	
	\section{求前$P$个数的逆元}
		\begin{lstlisting}
void solve (int m) {
	int inv[m];
	inv[1] = 1;
	for (int i = 2; i < m; ++ i) {
		inv[i] = ((long long)(m - m / i) * inv[m % i]) % m;
	}
}
\end{lstlisting}

	
	\section{广义离散对数(不需要互质)}
		\begin{lstlisting}
void extendedGcd (int a, int b, long long &x, long long y) {
	if (b) {
		extendedGcd(b, a % b, y, x);
		y -= a / b * x;
	} else {
		x = a;
		y = 0;
	}
}
int inverse (int a, int m) {
	long long x, y;
	extendedGcd(a, m, x, y);
	return (x % m + m) % m;
}
// a ^ x = b (mod m)
int solve (int a, int b, int m) {
	int tmp = 1 % m, c;
	map<int, int> s;
	if (tmp == b) {
		return 0;
	}
	for (int i = 1; i <= 50; ++ i) {
		tmp = ((long long)tmp * a) % m;
		if (tmp == b) {
			return i;
		}
	}
	int x_0 = 0, d = 1 % m;
	while (true) {
		tmp = gcd(a, m);
		if (tmp == 1) {
			break;
		}
		x_0 ++;
		d = ((long long)d * (a / tmp)) % m;
		if (b % tmp) {
			return -1;
		}
		b /= tmp;
		m /= tmp;
	}
	b = ((long long)b * inverse(d, m)) % m;
	c = int(ceil(sqrt(m)));
	s.clear();
	tmp = b;
	int tmpInv = intverse(a, m);
	for (int i = 0; i != c; ++ i) {
		if (s.find(tmp) == s.end()) {
			s[tmp] = i;
		}
		tmp = ((long long)tmp * tmpInv) % m;
	}
	tmp = 1;
	for (int i = 0; i != c; ++ i) {
		tmp = ((long long)tmp * a) % m;
	}
	int ans = 1;
	for (int i = 0; i != c; ++ i) {
		if (s.find(ans) != s.end()) {
			return x_0 + i * c + s.find(ans)->second;
		}
		ans = ((long long)ans * tmp) % m;
	}
	return -1;
}
\end{lstlisting}

	
	\section{n次剩余}
		\begin{lstlisting}
const int LimitSave=100000;
long long P,K,A;
vector<long long>ans;
struct tp{
	long long expo,res;
}data[LimitSave+100];
long long _mod(long long a, long long mo) {
	a=a%mo;
	if (a<0) a+=mo;
	return a;
}
long long powers(long long a , long long K , long long modular) {
	long long res;
	res=1;
	while (K!=0) {
		if (K & 1) res=_mod(res*a,modular);
		K=K>>1;
		a=_mod(a*a , modular);
	}
	return res;
}
long long get_originroot(long long p) {
	long long primes[100];
	long long tot,i,tp,j;
	i=2; tp=P-1; tot=0;
	while (i*i<=P-1) {
		if (_mod(tp,i)==0) {
			tot++;
			primes[tot]=i;
			while (_mod(tp,i)==0) tp/=i;
		}
		i++;
	}
	if (tp!=1) {tot++; primes[tot]=tp;}
	i=2;
	bool ok;
	while (1) {
		ok=true;
		foru(j,1,tot) {
			if (powers(i, (P-1)/primes[j] , P)==1) {
				150
				ok=false;
				break;
			}
		}
		if (ok) break;
		i++;
	}
	return i;
}
bool
euclid_extend(long long A ,long long B ,long long C ,long long &x, long
long &y, long long
&gcdnum) {
	long long t;
	if (A==0) {
		gcdnum = B;
		if (_mod(C , B) ==0) {
			x=0; y=C/B;
			return true;
		}
		else return false;
	}
	else if (euclid_extend(_mod(B , A) , A , C , y , t , gcdnum)) {
		x = t - int(B / A) * y;
		return true;
	}
	else return false;
}
long long Division(long long A, long long B, long long modular) {
	long long gcdnum,K,Y;
	euclid_extend(modular, B,A,K,Y,gcdnum);
	Y=_mod(Y,modular);
	if (Y<0) Y+=modular;
	return Y;
}
bool Binary_Search(long long key, long long &position) {
	long long start,stop;
	start=1; stop=LimitSave;
	bool flag=true;
	while (start<=stop) {
		position=(start+stop)/2;
		if (data[position].res==key) return true;
		else
			if (data[position].res<key) start=position+1;
			else stop=position-1;
	}
	return false;
}
bool compareab(const tp &a, const tp &b) {
	return a.res<b.res;
}
long long get_log(long long root, long long A, long long modular) {
	long long i,j,times,XD,XT,position;
	if (modular-1<LimitSave) {
		long long now=1;
		foru(i,0,modular-1) {
			if (now==A) {
				return i;
			}
			now=_mod(now * root , modular);
		}
	}
	data[1].expo=0; data[1].res=1;
	foru(i,1,LimitSave-1) {
		data[i+1].expo=i;
		data[i+1].res=_mod(data[i].res*root,modular);
	}
	sort(data+1,data+LimitSave+1,compareab);
	times=powers(root,LimitSave,modular);
	j=0;
	XD=1;
	while (1) {
		XT=Division(A,XD,modular);
		if (Binary_Search(XT,position)) {
			return j+data[position].expo;
		}
		j=j+LimitSave;
		XD=_mod(XD*times,modular);
	}
}
void work_ans() {
	ans.clear();
	if (A==0) {
		ans.push_back(0);
		return;
	}
	long long root,logs,delta,deltapower,now,gcdnum, i,x,y;
	root=get_originroot(P);
	logs=get_log(root,A,P);
	if (euclid_extend(K,P-1,logs,x,y,gcdnum)) {
		delta=(P-1)/gcdnum;
		x=_mod(x,delta);
		if (x<0) x+=delta;
		now=powers(root,x,P);
		deltapower=powers(root,delta,P);
		while (x<P-1) {
			ans.push_back(now);
			now=_mod(now*deltapower,P);
			x=x+delta;
		}
	}
	if (ans.size()>1)
		sort(ans.begin(),ans.end());
}
int main(){
	int i,j,k,test,cases=0;
	scanf("%d",&test);
	prepare();
	while (test) {
		test--;
		cin>>P>>K>>A;
		A=A % P;
		//x^K mod P = A
		cases++;
		printf("Case #%d:\n",cases);
		work_ans();
	}
	return 0;
}
\end{lstlisting}

	
	\section{二次剩余}
		\begin{lstlisting}
/*
   a*x^2+b*x+c==0 (mod P)
   求 0..P-1 的根
 */
#include <cstdio>
#include <cstdlib>
#include <ctime>
#define sqr(x) ((x)*(x))
int pDiv2,P,a,b,c,Pb,d;
inline int calc(int x,int Time)
{
	if (!Time) return 1;
	int tmp=calc(x,Time/2);
	tmp=(long long)tmp*tmp%P;
	if (Time&1) tmp=(long long)tmp*x%P;
	return tmp;
}
inline int rev(int x)
{
	if (!x) return 0;
	return calc(x,P-2);
}
inline void Compute()
{
	while (1)
	{
		b=rand()%(P-2)+2;
		if (calc(b,pDiv2)+1==P) return;
	}
}
int main()
{
	srand(time(0)^312314);
	int T;
	for (scanf("%d",&T);T;--T)
	{
		scanf("%d%d%d%d",&a,&b,&c,&P);
		if (P==2)
		{
			int cnt=0;
			for (int i=0;i<2;++i)
				if ((a*i*i+b*i+c)%P==0) ++cnt;
			printf("%d",cnt);
			for (int i=0;i<2;++i)
				if ((a*i*i+b*i+c)%P==0) printf(" %d",i);
			puts("");
		}else
		{
			int delta=(long long)b*rev(a)*rev(2)%P;
			a=(long long)c*rev(a)%P-sqr( (long long)delta )%P;
			a%=P;a+=P;a%=P;
			a=P-a;a%=P;
			pDiv2=P/2;
			if (calc(a,pDiv2)+1==P) puts("0");
			else
			{
				int t=0,h=pDiv2;
				while (!(h%2)) ++t,h/=2;
				int root=calc(a,h/2);
				if (t>0)
				{
					Compute();
					Pb=calc(b,h);
				}
				for (int i=1;i<=t;++i)
				{
					d=(long long)root*root*a%P;
					for (int j=1;j<=t-i;++j)
						d=(long long)d*d%P;
					if (d+1==P)
						root=(long long)root*Pb%P;
					Pb=(long long)Pb*Pb%P;
				}
				root=(long long)a*root%P;
				int root1=P-root;
				root-=delta;
				root%=P;
				if (root<0) root+=P;
				root1-=delta;
				root1%=P;
				if (root1<0) root1+=P;
				if (root>root1)
				{
					t=root;root=root1;root1=t;
				}
				if (root==root1) printf("1 %d\n",root);
				else printf("2 %d %d\n",root,root1);
			}
		}
	}
	return 0;
}
\end{lstlisting}

    
    \section{长方体表面两点最短距离}
        \begin{lstlisting}
#include<cstdio>
#include<iostream>
#include<algorithm>

using namespace std;

int r;
void turn(int i, int j, int x, int y, int z, int x0, int y0, int L, int W, int H)
{
	if (z == 0) {
		int R = x * x + y * y;
		if (R < r) r = R;
	} else {
		if (i >= 0 && i < 2)
			turn(i + 1, j, x0 + L + z, y, x0 + L - x, x0 + L, y0, H, W, L);
		if (j >= 0 && j < 2)
			turn(i, j + 1, x, y0 + W + z, y0 + W - y, x0, y0 + W, L, H, W);
		if (i <= 0 && i > -2)
			turn(i - 1, j, x0 - z, y, x - x0, x0 - H, y0, H, W, L);
		if (j <= 0 && j > -2)
			turn(i, j - 1, x, y0 - z, y - y0, x0, y0 - H, L, H, W);
	}
}

int main()
{
	int L, H, W, x1, y1, z1, x2, y2, z2;
	cin >> L >> W >> H >> x1 >> y1 >> z1 >> x2 >> y2 >> z2;
	if (z1 != 0 && z1 != H) {
		if (y1 == 0 || y1 == W)
			swap(y1, z1), swap(y2, z2), swap(W, H);
		else
			swap(x1, z1), swap(x2, z2), swap(L, H);
	}
	if (z1 == H) z1 = 0, z2 = H - z2;
	r = 0x3fffffff; 
	turn(0, 0, x2 - x1, y2 - y1, z2, -x1, -y1, L, W, H);
	cout << r << endl;
	return 0;
}
\end{lstlisting}


	\section{字符串的最小表示}
		\subsection{Logic\_IU}
			传入字符串s,返回i,表示以i开始的循环串字典序最小,但不保证i在同样字典序最小的循环串里起始位置最小
\begin{lstlisting}
int minCycle(char *a)
{
	int n = strlen(a);
	for(int i = 0; i < n; ++ i) {
		a[i + n] = a[i];
	}
	a[n + n] = 0;
	int i = 0, j = 1, k = 0;
	do {
		for(k = 0; a[i + k] == a[j + k]; ++ k);
		if (a[i + k] > a[j + k]) i = i + k + 1;
		else j = j + k + 1;
		j += i == j;
		if (i > j) swap(i, j);
	} while(j < n);
	return i;
}
\end{lstlisting}

		\subsection{tEJtM}
			\begin{lstlisting}
struct cyc_string
{
	int n, offset;
	char str[max_length];
	char & operator [] (int x)
	{return str[((offset + x) % n)];}
	cyc_string(){offset = 0;}
};
void minimum_circular_representation(cyc_string & a)
{
	int i = 0, j = 1, dlt = 0, n = a.n;
	while(i < n and j < n and dlt < n)
	{
	  if(a[i + dlt] == a[j + dlt]) dlt++;
	  else
	  {
	    if(a[i + dlt] > a[j + dlt]) i += dlt + 1; else j += dlt + 1;
	    dlt = 0;
	  }
	}
	a.offset = min(i, j);
}
int main()
{return 0;}
\end{lstlisting}

    
    \section{牛顿迭代开根号}
        速度慢,精度有保证
\begin{lstlisting}
typedef unsigned long long ull;
ull sqrtll(ull n)
{
	if (n == 0) return 0;
	ull x = 1ull << ((63 - __builtin_clzll(n)) >> 1);
	ull xx = -1;
	for( ; ; ) {
		ull nx = (x + n / x) >> 1;
		if (nx == xx)
			return min(x, nx);
		xx = x;
		x = nx;
	}
}
\end{lstlisting}


    \section{求某年某月某日星期几}
        #include<iostream>
using namespace std;

int whatday(int d, int m, int y)
{
	int ans;
	if (m == 1 || m == 2) {
		m += 12; y --;
	}
	if ((y < 1752) || (y == 1752 && m < 9) || (y == 1752 && m == 9 && d < 3))
		ans = (d + 2 * m + 3 * (m + 1) / 5 + y + y / 4 + 5) % 7;
	else ans = (d + 2 * m + 3 * (m + 1) / 5 + y + y / 4 - y / 100 + y / 400) % 7;
	return ans;
}

int main()
{
	cout << whatday(30, 10, 2013) << endl;
}

	
	\section{A*}
		\begin{lstlisting}
#include <cstdio>
#include <cstdlib>
#include <cstring>
#include <ctime>

using namespace std;

struct {
       int pos,tot;
} w[2000001];

const int inf=120234234;
int n,m,l,len,first[5001],c[2000001],dist[5001],where[200001],next[200001],v[200001],f[5001],lenn[5001];
bool b[5001];

inline void makelist(int x,int y,int z){
    where[++l]=y;
    v[l]=z;
    next[l]=first[x];
    first[x]=l;
}

inline void spfa(){
    memset(b,false,sizeof(false));
    memset(f,127,sizeof(f));
    f[n]=0;
    c[1]=n;
    for (int k=1,l=1;l<=k;l++)
     {
             int m=c[l];
             b[m]=false;
             for (int x=first[m];x;x=next[x])
                 if (f[m]+v[x]<f[where[x]])
                 {
                     f[where[x]]=f[m]+v[x];
                     if (!b[where[x]])
                     {
                         b[where[x]]=true;
                         c[++k]=where[x];
                     }
                 }
     }
}

inline void insect(int re,int uu){
    w[++len].pos=re;
    w[len].tot=uu;
    int now=len;
    while (now!=1)
      if (w[now].tot+f[w[now].pos]<w[now>>1].tot+f[w[now>>1].pos])
        {
           int k=w[now].tot;
           w[now].tot=w[now>>1].tot;
           w[now>>1].tot=k;
           k=w[now].pos;
           w[now].pos=w[now>>1].pos;
           w[now>>1].pos=k;
           now=now>>1;
        }
      else break;
}

inline void delete1(){
    w[1].pos=w[len].pos;
    w[1].tot=w[len].tot;
    w[len].pos=inf;
    w[len].tot=inf;
    len--;
    int now=1;
    while ((now<<1)<=len)
      if ((now<<1)==len)
       if (w[now<<1].tot+f[w[now<<1].pos]<w[now].tot+f[w[now].pos])
      {
           int k=w[now].tot;
           w[now].tot=w[now<<1].tot;
           w[now<<1].tot=k;
           k=w[now].pos;
           w[now].pos=w[now<<1].pos;
           w[now<<1].pos=k;
           now=now<<1;
      }
      else break; 
      else
      if ((w[now<<1].tot+f[w[now<<1].pos]<w[(now<<1)+1].tot+f[w[(now<<1)+1].pos]))
      if (w[now<<1].tot+f[w[now<<1].pos]<w[now].tot+f[w[now].pos])
      {
           int k=w[now].tot;
           w[now].tot=w[now<<1].tot;
           w[now<<1].tot=k;
           k=w[now].pos;
           w[now].pos=w[now<<1].pos;
           w[now<<1].pos=k;
           now=now<<1;
      }
      else break;
      else 
      if (w[(now<<1)+1].tot+f[w[(now<<1)+1].pos]<w[now].tot+f[w[now].pos])
      {
           int k=w[now].tot;
           w[now].tot=w[(now<<1)+1].tot;
           w[(now<<1)+1].tot=k;
           k=w[now].pos;
           w[now].pos=w[(now<<1)+1].pos;
           w[(now<<1)+1].pos=k;
           now=(now<<1)+1;
      }
      else break;
}

inline void spfa_ans(){
    memset(dist,127,sizeof(dist));
    memset(lenn,0,sizeof(lenn));
    memset(w,127,sizeof(w));
    c[1]=1;
    len=1;
    w[1].pos=1;
    w[1].tot=0;
    for (int k=1,l=1;l<=k;)
      {
             int m=w[1].pos,flow=w[1].tot;
             delete1();
             dist[m]=inf;
             lenn[m]++;
             if (lenn[m]>1000) 
             {
                printf("%d\n",-1);
                return;
             }
             if ((m==n)&&(lenn[m]==2)) 
              {
                    printf("%d\n",flow);
                    return;
              }
             for (int x=first[m];x;x=next[x])
                    {
                       dist[where[x]]=flow+v[x];
                       insect(where[x],dist[where[x]]);
                    }
      }
}
    
int main(){
    scanf("%d%d",&n,&m);
    l=0;
    for (int i=1;i<=m;i++)
      {
             int x,y,z;
             scanf("%d%d%d",&x,&y,&z);
             makelist(x,y,z);
             makelist(y,x,z);
      }
    spfa();
    spfa_ans();
}
\end{lstlisting}


	\section{Dancing Links}
		\begin{lstlisting}
#include<stdio.h>
#include<stdlib.h>
#include<string.h>
#include<time.h>
#define maxn 105
#define N maxn*maxn
int a[maxn][maxn],l[N],r[N],d[N],u[N],c[N],s[maxn],head[maxn],n,m,ans;
inline int getid(int x,int y){return (x-1)*n+y;}
void remove(int x){
	l[r[x]]=l[x];r[l[x]]=r[x];
	for (int i=d[x];i!=x;i=d[i])
		for (int j=r[i];j!=i;j=r[j]){
			u[d[j]]=u[j];d[u[j]]=d[j];
			--s[c[j]];
		}
}
void resume(int x){
	for (int i=u[x];i!=x;i=u[i])
		for (int j=l[i];j!=i;j=l[j]){
			u[d[j]]=j;d[u[j]]=j;
			++s[c[j]];
		}
	l[r[x]]=x;r[l[x]]=x;
}
void dfs(int t){
	if (t>=ans)return;
	if (!r[0]){
		if (t<ans)ans=t;
		return;
	}
	int x=0,min=1<<30;
	for (int i=r[0];i;i=r[i])
		if (s[i]<min)min=s[i],x=i;
	remove(x);
	for (int i=d[x];i!=x;i=d[i]){
		for (int j=r[i];j!=i;j=r[j])remove(c[j]);
		dfs(t+1);
		for (int j=l[i];j!=i;j=l[j])resume(c[j]);
	}
	resume(x);
}
int main()
{
	//freopen("1.in","r",stdin);
	//freopen("1.out","w",stdout);
	memset(a,0,sizeof(a));
	scanf("%d%d",&m,&n);
	for (int i=1;i<=n;++i){
		int x,y;scanf("%d",&x);
		for (int j=1;j<=x;++j){
			scanf("%d",&y);a[i][y]=1;
		}
	}
	for (int i=1;i<=m;++i)head[i]=n*m+i; head[0]=0;
	for (int i=1;i<=m;++i)r[head[i]]=head[i+1];
	for (int i=1;i<=m;++i)l[head[i]]=head[i-1];
	r[head[0]]=head[1];l[head[1]]=head[0];
	l[head[0]]=head[m];r[head[m]]=head[0];
	for (int i=1;i<=n;++i){
		int pre=0,first=0;
		for (int j=1;j<=m;++j)if (a[i][j]){
			if (pre)l[getid(i,j)]=getid(i,pre),r[getid(i,pre)]=getid(i,j);
			pre=j;if (!first)first=j;
		}
		if (first){
			l[getid(i,first)]=getid(i,pre);r[getid(i,pre)]=getid(i,first);
		}
	}
	for (int j=1;j<=m;++j){
		int pre=0,first=0;
		for (int i=1;i<=n;++i)if (a[i][j]){
			if (pre)u[getid(i,j)]=getid(pre,j),d[getid(pre,j)]=getid(i,j);
			pre=i;if (!first)first=i;
		}
		if (pre){
			u[getid(first,j)]=head[j];d[head[j]]=getid(first,j);
			u[head[j]]=getid(pre,j);d[getid(pre,j)]=head[j];
		}
	}
	for (int i=1;i<=n;++i)
		for (int j=1;j<=m;++j)if (a[i][j])c[getid(i,j)]=head[j];
	memset(s,0,sizeof(s));
	for (int i=1;i<=n;++i)
		for (int j=1;j<=m;++j)if (a[i][j])++s[j];
	ans=1<<30;
	dfs(0);
	if (ans==1<<30)printf("-1\n");
	else printf("%d\n",ans);
	system("pause");for (;;);
	return 0;
}
\end{lstlisting}



	\section{弦图判定}
		\begin{lstlisting}
int n, m, first[1001], l, next[2000001], where[2000001],f[1001], a[1001], c[1001], L[1001], R[1001],
v[1001], idx[1001], pos[1001];
bool b[1001][1001];

int read(){
    char ch;
    for (ch = getchar(); ch < '0' || ch > '9'; ch = getchar());
    int cnt = 0;
    for (; ch >= '0' && ch <= '9'; ch = getchar()) cnt = cnt * 10 + ch - '0';
    return(cnt);
}

inline void makelist(int x, int y){
    where[++l] = y;
    next[l] = first[x];
    first[x] = l;
}

bool cmp(const int &x, const int &y){
    return(idx[x] < idx[y]);
}

int main(){
   //freopen("1015.in", "r", stdin);
   // freopen("1015.out", "w", stdout);
    for (;;)
    {
        n = read(); m = read();
        if (!n && !m) return 0;
        memset(first, 0, sizeof(first)); l = 0;
        memset(b, false, sizeof(b));
        for (int i = 1; i <= m; i++) 
        {
            int x = read(), y = read();
            if (x != y && !b[x][y])
            {
               b[x][y] = true; b[y][x] = true;
               makelist(x, y); makelist(y, x);
            }
        }
        memset(f, 0, sizeof(f));
        memset(L, 0, sizeof(L));
        memset(R, 255, sizeof(R));
        L[0] = 1; R[0] = n;
        for (int i = 1; i <= n; i++) c[i] = i, pos[i] = i;
        memset(idx, 0, sizeof(idx));
        memset(v, 0, sizeof(v));
        for (int i = n; i; --i)
        {
            int now = c[i];
            R[f[now]]--;
            if (R[f[now]] < L[f[now]]) R[f[now]] = -1;
            idx[now] = i; v[i] = now;
            for (int x = first[now]; x; x = next[x])
                if (!idx[where[x]]) 
                {
                   swap(c[pos[where[x]]], c[R[f[where[x]]]]);
                   pos[c[pos[where[x]]]] = pos[where[x]];
                   pos[where[x]] = R[f[where[x]]];
                   L[f[where[x]] + 1] = R[f[where[x]]]--;
                   if (R[f[where[x]]] < L[f[where[x]]]) R[f[where[x]]] = -1;
                   if (R[f[where[x]] + 1] == -1)
                       R[f[where[x]] + 1] = L[f[where[x]] + 1];
                   ++f[where[x]];
                }
        }
        bool ok = true;
        //v是完美消除序列.
        for (int i = 1; i <= n && ok; i++)
        {
            int cnt = 0;
            for (int x = first[v[i]]; x; x = next[x]) 
                if (idx[where[x]] > i) c[++cnt] = where[x];
            sort(c + 1, c + cnt + 1, cmp);
            bool can = true;
            for (int j = 2; j <= cnt; j++)
                if (!b[c[1]][c[j]])
                {
                    ok = false;
                    break;
                }
        }
        if (ok) printf("Perfect\n");
        else printf("Imperfect\n");
        printf("\n");
    }
}
\end{lstlisting}

	
	\section{弦图求团数}
		\begin{lstlisting}
#include <cstdio>
#include <cstdlib>
#include <cstring>
#include <ctime>
#include <cmath>
#include <iostream>
#include <algorithm>

using namespace std;

int n, m, first[100001], next[2000001], where[2000001], l, L[100001], R[100001], c[100001], f[100001],
pos[100001], idx[100001], v[100001], ans;

inline void makelist(int x, int y){
    where[++l] = y;
    next[l] = first[x];
    first[x] = l;
}

int read(){
    char ch;
    for (ch = getchar(); ch < '0' || ch > '9'; ch = getchar());
    int cnt = 0;
    for (; ch >= '0' && ch <= '9'; ch = getchar()) cnt = cnt * 10 + ch - '0';
    return(cnt);
}

int main(){
    freopen("1006.in", "r", stdin);
    freopen("1006.out", "w", stdout);
    memset(first, 0, sizeof(first)); l = 0;
    n = read(); m = read();
    for (int i = 1; i <= m; i++)
    {
        int x, y;
        x = read(); y = read();
        makelist(x, y); makelist(y, x);
    }
    memset(L, 0, sizeof(L));
    memset(R, 255, sizeof(R));
    memset(f, 0, sizeof(f));
    memset(idx, 0, sizeof(idx));
    for (int i = 1; i <= n; i++) c[i] = i, pos[i] = i;
    L[0] = 1; R[0] = n; ans = 0;
    for (int i = n; i; --i)
    {
        int now = c[i], cnt = 1;
        idx[now] = i; v[i] = now;
        if (--R[f[now]] < L[f[now]]) R[f[now]] = -1;
        for (int x = first[now]; x; x = next[x])
            if (!idx[where[x]])
            {
                swap(c[pos[where[x]]], c[R[f[where[x]]]]);
                pos[c[pos[where[x]]]] = pos[where[x]];
                pos[where[x]] = R[f[where[x]]];
                L[f[where[x]] + 1] = R[f[where[x]]]--;
                if (R[f[where[x]]] < L[f[where[x]]]) R[f[where[x]]] = -1;
                if (R[f[where[x]] + 1] == -1) R[f[where[x]] + 1] = L[f[where[x]] + 1];
                ++f[where[x]];
            }
            else ++cnt;
        ans = max(ans, cnt);
    }
    printf("%d\n", ans);
}
\end{lstlisting}


	\section{有根树的同构}
		\begin{lstlisting}
//http://acm.sdut.edu.cn/judgeonline/showproblem?problem_id=1861 �и�����ͬ�� 
#include <cstdio>
#include <cstdlib>
#include <cstring>
#include <ctime>

using namespace std;

const int mm=1051697,p=4773737;
int m,n,first[101],where[10001],next[10001],l,hash[10001],size[10001],pos[10001];
long long f[10001],rt[10001];
bool in[10001];


inline void makelist(int x,int y){
    where[++l]=y;
    next[l]=first[x];
    first[x]=l;
}


inline void hashwork(int now){
    int a[1001],v[1001],tot=0;
    size[now]=1;
    for (int x=first[now];x;x=next[x])
    {
        hashwork(where[x]);
        a[++tot]=f[where[x]];
        v[tot]=size[where[x]];
        size[now]+=size[where[x]];
    }
    a[++tot]=size[now];
    v[tot]=1;
    int len=0;
    for (int i=1;i<=tot;i++) 
       for (int j=i+1;j<=tot;j++)
          if (a[j]<a[i])
          {
             int u=a[i];a[i]=a[j];a[j]=u;
             u=v[i];v[i]=v[j];v[j]=u;
          }
    f[now]=1;
    for (int i=1;i<=tot;i++)
       {
             f[now]=((f[now]*a[i])%p*rt[len])%p;
             len+=v[i];
       }
}

int main(){
    //freopen("1.txt","r",stdin);
    //freopen("2.txt","w",stdout);
    scanf("%d%d",&n,&m);
    rt[0]=1;
    for (int i=1;i<=100;i++)
        rt[i]=(rt[i-1]*mm)%p; 
    for (int i=1;i<=n;i++)
    {
        memset(first,0,sizeof(first));
        memset(in,false,sizeof(in));
        l=0;
        for (int j=1;j<m;j++)
        {
            int x,y;
            scanf("%d%d",&x,&y);
            makelist(x,y);
            in[y]=true;
        }
        int root=0;
        for (int j=1;j<=m;j++)
        if (!in[j]) 
        {
            root=j;
            break;
        }
        memset(size,0,sizeof(size));
        memset(f,0,sizeof(f));
        hashwork(root);
        hash[i]=f[root];
    }
    for (int i=1;i<=n;i++) pos[i]=i;
    memset(in,false,sizeof(in));
    for (int i=1;i<=n;i++)
     if (!in[i])
     {
                printf("%d",i);
                for (int j=i+1;j<=n;j++)
                if (hash[j]==hash[i])
                {
                    in[j]=true;
                    printf("=%d",j);
                }       
                printf("\n");
     }
}
\end{lstlisting}           

	
	\section{极大团搜索算法}
		Int g[][]为图的邻接矩阵。\\
MC(V)表示点集V的最大团\\
令Si={vi, vi+1, ..., vn}, mc[i]表示MC(Si)\\
倒着算mc[i],那么显然MC(V)=mc[1]\\
此外有mc[i]=mc[i+1] or mc[i]=mc[i+1]+1\\
\begin{lstlisting}
void init(){
	int i, j;
	for (i=1; i<=n; ++i) for (j=1; j<=n; ++j) scanf("%d", &g[i][j]);
}
void dfs(int size){
	int i, j, k;
	if (len[size]==0) {
		if (size>ans) {
			ans=size; found=true;
		}
		return;
	}
	for (k=0; k<len[size] && !found; ++k) {
		if (size+len[size]-k<=ans) break;
		i=list[size][k];
		if (size+mc[i]<=ans) break;
		for (j=k+1, len[size+1]=0; j<len[size]; ++j)
			if (g[i][list[size][j]]) list[size+1][len[size+1]++]=list[size][j];
		dfs(size+1);
	}
}
void work(){
	int i, j;
	mc[n]=ans=1;
	for (i=n-1; i; --i) {
		found=false;
		len[1]=0;
		for (j=i+1; j<=n; ++j) if (g[i][j]) list[1][len[1]++]=j;
		dfs(1);
		mc[i]=ans;
	}
}
void print(){
	printf("%d\n", ans);
}
\end{lstlisting}

	
	\section{极大团的计数}
		Bool g[][] 为图的邻接矩阵,图点的标号由1至n。\\
\begin{lstlisting}
void dfs(int size){
	int i, j, k, t, cnt, best = 0;
	bool bb;
	if (ne[size]==ce[size]){
		if (ce[size]==0) ++ans;
		return;
	}
	for (t=0, i=1; i<=ne[size]; ++i) {
		for (cnt=0, j=ne[size]+1; j<=ce[size]; ++j)
			if (!g[list[size][i]][list[size][j]]) ++cnt;
		if (t==0 || cnt<best) t=i, best=cnt;
	}
	if (t && best<=0) return;
	for (k=ne[size]+1; k<=ce[size]; ++k) {
		if (t>0){
			for (i=k; i<=ce[size]; ++i) if (!g[list[size][t]][list[size][i]])
				break;
			swap(list[size][k], list[size][i]);
		}
		i=list[size][k];
		ne[size+1]=ce[size+1]=0;
		for (j=1; j<k; ++j)if (g[i][list[size][j]])
			list[size+1][++ne[size+1]]=list[size][j];
		for (ce[size+1]=ne[size+1], j=k+1; j<=ce[size]; ++j)
			if (g[i][list[size][j]]) list[size+1][++ce[size+1]]=list[size][j];
		dfs(size+1);
		++ne[size];
		--best;
		for (j=k+1, cnt=0; j<=ce[size]; ++j) if (!g[i][list[size][j]]) ++cnt;
		if (t==0 || cnt<best) t=k, best=cnt;
		if (t && best<=0) break;
	}
}
int work(){
	int i;
	ne[0]=0; ce[0]=0;
	for (i=1; i<=n; ++i) list[0][++ce[0]]=i;
	ans=0;
	dfs(0);
	return 0;
}
\end{lstlisting}

	
	\section{多项式求根(求导二分)}
		\begin{lstlisting}
const double error=1e-12;
const double infi=1e+12;
double a[10],x[10];
int n;
int sign(double x) {
	return (x<-error)?(-1):(x>error);
}
double f(double a[],int n,double x) {
	double tmp=1,sum=0;
	for (int i=0;i<=n;i++) {
		sum=sum+a[i]*tmp;
		tmp=tmp*x;
	}
	return sum;
}
double binary(double l,double r,double a[],int n) {
	int sl=sign(f(a,n,l)),sr=sign(f(a,n,r));
	if (sl==0) return l;
	if (sr==0) return r;
	if (sl*sr>0) return infi;
	while (r-l>error) {
		double mid=(l+r)/2;
		int ss=sign(f(a,n,mid));
		if (ss==0) return mid;
		if (ss*sl>0) l=mid; else r=mid;
	}
	return l;
}
void solve(int n,double a[],double x[],int &nx) {
	if (n==1) {
		x[1]=-a[0]/a[1];
		nx=1;
		return;
	}
	double da[10],dx[10];
	int ndx;
	for (int i=n;i>=1;i--) da[i-1]=a[i]*i;
	solve(n-1,da,dx,ndx);
	nx=0;
	if (ndx==0) {
		double tmp=binary(-infi,infi,a,n);
		if (tmp<infi) x[++nx]=tmp;
		return;
	}
	double tmp;
	tmp=binary(-infi,dx[1],a,n);
	if (tmp<infi) x[++nx]=tmp;
	for (int i=1;i<=ndx-1;i++) {
		tmp=binary(dx[i],dx[i+1],a,n);
		if (tmp<infi) x[++nx]=tmp;
	}
	tmp=binary(dx[ndx],infi,a,n);
	if (tmp<infi) x[++nx]=tmp;
}
int main() {
	scanf("%d",&n);
	for (int i=n;i>=0;i--) scanf("%lf",&a[i]);
	int nx;
	solve(n,a,x,nx);
	for (int i=1;i<=nx;i++) printf("%0.6lf\n",x[i]);
	return 0;
}
\end{lstlisting}

	
	\section{有多少个点在多边形内}
		\begin{lstlisting}
//rn中的标号必须逆时针给出。一开始要旋转坐标,保证同一个x值上只有一个点。正向减点,
//反向加点。num[i][j]=num[j][i]=严格在这根线下方的点。 on[i][j]=on[j][i]=严格
//在线段上的点,包括两个端点。若有回边的话注意计算onit的方法,不要多算了线段上的点。
int ans=0,z,onit=0,lows=0;
rep(z,t) {
	i=rn[z]; j=rn[z+1]; onit+=on[i][j]-1;
	if (a[j].x>a[i].x){ans-=num[i][j];lows+=on[i][j]-1;}
	else ans+=num[i][j];
}
//ans-lows+1 is inside. 只会多算一次正向上的点(除去最左和最右的点)。Lows只算了除开最左边的点,但会多算最右边的点,所以要再加上1.
printf("%d\n",ans-lows+1+ onit);
\end{lstlisting}

	
	\section{斜线下格点统计}
		\begin{lstlisting}
LL solve(LL n, LL a, LL b, LL m){
	//计算for (int i=0;i<n;++i) s+=floor((a+b*i)/m)
	//n,m,a,b>0
	//printf("%lld %lld %lld %lld\n", n, a, b, m);
	if(b == 0){
		return n * (a / m);
	}
	if(a >= m){
		return n * (a / m) + solve(n, a % m, b, m);
	}
	if(b >= m){
		return (n - 1) * n / 2 * (b / m) + solve(n, a, b % m, m);
	}
	LL q = (a + b * n) / m;
	return solve(q, (a + b * n) % m, m, b);
}
\end{lstlisting}


	\section{杂知识}
	    \subsection*{牛顿迭代}
x1=x0-func(x0)/func1(x0);进行牛顿迭代计算\\
我们要求 f(x)=0 的解。func(x)为原方程,func1 为原方程的导数方程
\subsection*{图同构 Hash}
	\begin{displaymath}
		F_t(i) = (F_{t - 1}(i) * A + \sum_{i \to j} (F_{t - 1}(j) * B) + \sum_{j \to i} (F_{t - 1}(j) * C) + D * (i == a)) \mod P
	\end{displaymath}
	枚举点a, 迭代K次后求得的$F_k(a)$就是a点所对应的hash值。\\
	其中 K、A、B、C、D、P 为 hash 参数,可自选。

\subsection*{圆上有整点的充要条件}
设正整数 n 的质因数分解为 $n = \Pi p_i^{a_i}$,则 $x^2+y^2=n$ 有整数解的充要条件是 n 中不存在形
如 $p_i \mod 4 = 3$ 且指数 $a_i$ 为奇数的质因数 $p_i$

\subsection*{Pick 定理}
	简单多边形,不自交。(严格在多边形内部的整点数*2 + 在边上的整点数– 2)/2 = 面积

\subsection*{图定理}
	定理 1:最小覆盖数 = 最大匹配数\\
	定理 2:最大独立集 S 与 最小覆盖集 T 互补。\\
	算法:\\
	1. 做最大匹配,没有匹配的空闲点$\in S$\\
	2. 如果 $u \in S$ 那么 u 的临点必然属于 T\\
	3. 如果一对匹配的点中有一个属于 T 那么另外一个属于 S\\
	4. 还不能确定的,把左子图的放入 S,右子图放入 T\\
	算法结束\\

\subsection*{梅森素数}
	p 是素数且 $2^p-1$ 的是素数,n 不超过 258 的全部梅森素数终于确定!是:\\
	n=2,3,5,7,13,17,19,31,61,89,107,127

\subsection*{上下界网络流}
	有上下界网络流,求可行流部分,增广的流量不是实际流量。若要求实际流量应该强算一遍源点出去的流量。\\
	求最小下届网络流:\\
	方法一:加 t-s 的无穷大流,求可行流,然后把边反向后(减去下届网络流),在残留网络中从汇到源做最大流。\\
	方法二:在求可行流的时候,不加从汇到源的无穷大边,得到最大流 X, 加上从汇到源无穷大边后,再求最大流得到 Y。那么 Y 即是答案最小下界网络流。\\
	原因:感觉上是在第一遍已经把内部都消耗光了,第二遍是必须的流量。

\subsection*{平面图定理}
	平面图一定存在一个度小于等于 5 的点,且可以四染色\\
	( 欧拉公式 ) 设 G 是连通的平面图,n,m,r分别是其顶点数、边数和面数,n-m+r=2\\
	极大平面图 $m\leq 3n-6$

\subsection*{Fibonacci相关结论}
	gcd(F[n],F[m])=F[gcd(n,m)]\\
	Fibonacci 质数(和前面所有的 Fibonacci 数互质), 下标为质数或4\\
	定理:如果 a 是 b 的倍数,那么 F[a] 是 F[b] 的倍数。\\

\subsection*{二次剩余}
	p 为奇素数,若(a,p)=1, a 为 p 的二次剩余必要充分条件为 $a^{(p-1)/2} \mod p=1$.(否则为$p-1$)\\
	p 为奇素数, $x^b=a(\mod p)$,a 为 p 的 b 次剩余的必要充分条件为若 $a^{(p-1)/ (p-1, b)} \mod p=1$.


	\section{Language Reference}
	    \subsection{C++ Tips}
\begin{enumerate}
\item 开栈的命令 \#pragma comment(linker, "/STACK:16777216"), 交C++
\item ios::sync\_with\_stdio(false);
\item \%o 八进制 \%x 十六进制
\end{enumerate}

\subsection{Java Reference}
\lstset{language = Java}
\begin{lstlisting}
import java.io.*;
import java.math.*;
import java.util.*;

public class Main {
	final static int MOD = (int)1e9 + 7;

	public void run() {
		try {
			int n = reader.nextInt();
			String[] map = new String[n];
			for (int i = 0; i < n; ++ i) {
				map[i] = reader.next();
			}
			writer.println(10 % MOD);
		} catch (IOException ex) {
		}
		writer.close();
	}

	InputReader reader;
	PrintWriter writer;

	Main() {
		reader = new InputReader();
		writer = new PrintWriter(System.out);
	}

	public static void main(String[] args) {
		new Main().run();
	}

	void debug(Object...os) {
		System.err.println(Arrays.deepToString(os));
	}
}

class InputReader {
	BufferedReader reader;
	StringTokenizer tokenizer;

	InputReader() {
		reader = new BufferedReader(new InputStreamReader(System.in));
		tokenizer = new StringTokenizer("");
	}

	String next() throws IOException {
		while (!tokenizer.hasMoreTokens()) {
			tokenizer = new StringTokenizer(reader.readLine());
		}
		return tokenizer.nextToken();
	}

	Integer nextInt() throws IOException {
		return Integer.parseInt(next());
	}
}
//-------------------------------------------------------
import java.util.*;
import java.math.*;
import java.io.*;

public class Main {

	Scanner cin;

	void solve() {
		BigInteger a, b, c;
		a = cin.nextBigInteger();
		b = cin.nextBigInteger();
		c = a.add(b);
		System.out.println(a + " + " + b + " = " + c);
	}

	void run() {
		cin = new Scanner(new BufferedInputStream(System.in));
		int tmp = cin.nextInt();
		int testcase = 0;
		while(cin.hasNextBigInteger()) {
			++ testcase;
			if (testcase > 1)
				System.out.println();
			System.out.println("Case " + testcase + ":");
			solve();
		}
	}

	public static void main(String[] args) {
		new Main().run();
	}
}
//Arrays
int a[]=new int[10];
Arrays.fill(a,0);
Arrays.sort(a);
//String
String s;
s.charAt(int i);
s.compareTo(String b);
s.compareToIgnoreCase();
s.contains(String b);
s.length();
s.substring(int l,int len);
//BigInteger
BigInteger a;
a.abs();
a.add(b);
a.bitLength();
a.subtract(b);
a.divide(b);
a.remainder(b);
a.divideAndRemainder(b);
a.modPow(b,c);//a^b mod c;
a.pow(int);
a.multiply(b);
a.compareTo(b);
a.gcd(b);
a.intValue();
a.longValue();
a.isProbablePrime(int certainty);//(1 - 1/2^certainty).
a.nextProbablePrime();
a.shiftLeft(int);
a.valueOf();
//BigDecimal
static int ROUND_CEILING,ROUND_DOWN,ROUND_FLOOR,
		   ROUND_HALF_DOWN,ROUND_HALF_EVEN,ROUND_HALF_UP,ROUND_UP;
a.divide(BigDecimal b,int scale,int round_mode);
a.doubleValue();
a.movePointLeft(int i);
a.pow(int);
a.setScale(int scale,int round_mode);
a.stripTrailingZeros();
//StringBuilder
StringBuilder sb=new StringBuilder();
sb.append(elem);
out.println(sb);
//StringTokenizer
StringTokenizer st=new StringTokenizer(in.readLine());
st.countTokens();
st.hasMoreTokens();
st.nextToken();
//Vector
a.add(elem);
a.add(index,elem);
a.clear();
a.elementAt(index);
a.isEmpty();
a.remove(index);
a.set(index,elem);
a.size();
//Queue
a.add(elem);
a.peek();//front
a.poll();//pop
//Integer Double Long
\end{lstlisting}



	\section{vimrc}
		\begin{lstlisting}
set cin nu mouse=a nobk hls si go= ts=4 sts=4 sw=4

nmap <C-A> ggVG
vmap <C-C> "+y

syntax on

nmap<F4> : !gedit % <CR>
nmap<F3> : vnew %<.in <CR>
nmap<F5> : !./%< <CR>
nmap<F9> : !g++ % -o %< -Wall <CR>
nmap<F8> : !time ./%< < %<.in <CR>
nmap<F10> : !javac % <CR>
nmap<F6> : !java %< < %<.in <CR>
\end{lstlisting}

\end{document}
