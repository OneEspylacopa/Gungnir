\begin{enumerate}\setlength{\itemsep}{-\itemsep}
	\item 题意有毒时,要耐心仔细读题
	\item 对于模拟题,要注意可能会出现的细节case
	\item 当做到题号为G的题目时,be careful
	\item 对于题意/算法发生变动修改代码,必须小心谨慎
	\item 可能多解时看清输出哪一个解
	\item 注意Case格式在一场中可能不同
	\item 看样例解释
	\item 手滑:循环的终止条件/nmij打混/函数重载默认参数/多层数组嵌套/数据范围/复制的代码/struct成员初始化
	\item 要define的常见名:left,right,next,hash,log
	\item 对bitset的常数认识不够。
	\item 对于抠过常数还TLE的题目,没有注意到是做法不够优越。
	\item 对于讨论题目,陷入打补丁的死回圈。
	\item 没有注意到不合理的数据范围而导致得出错误的算法
	\item 当意识到程序的逻辑问题时(比如大小于号打反),注意其他位置是不是也犯了类似的错误(也打反了)。
	\item 分块的大小要考虑常数谨慎估计。
	\item 网络流的数组开成V不要开成N
	\item 看机时空的时候要冷静,否则容易导致不优的写法算法上位。
	\item 欧拉路注意判断连通性。
	\item 常识缺乏,一个空的vector空间约为10个int。
	\item 写splay/LCT时,当需要自顶向下访问时(如求前驱/后继/K值)忘记一边走一边relax标记。
	\item 忘记了变量已经修改,试图访问其原始值。
	\item Farmland 为了偷懒每次直接暴力找后继,挂在star上。
	\item Farmland 为了判断挖掘完毕的条件要使用边而不是点。
	\item 几何旋转(比如为了使得x distinct)之前先看清楚题目到底限定了哪些点无重点
	\item 几何整数旋转时估计好数值范围。
	\item 判断直线与圆交点,当两点都在圆外时不一定无交点。
	\item 注意题目N,M的读入顺序
	\item 注意题目条件可能会隐藏在样例解释之中
	\item 大代码查错时先检查下标,取模等错误
	\item 注意检查同种错误重复发生
	\item 要考虑输入数据或许不合理 要是程序有足够的容错性
	\item 弄混了局部变量和全局变量 这个写程序的时候要注意,局部变量和全局变量不要重名
	\item 爆long long 处理比较大的数的时候要注意
	\item 树hash 这个以后不会再写错了
	\item MLE 交代码之前要算一算内存
	\item 板子错 要验板子
	\item 上界设小 上界要小心确定
	\item 没有必要的操作导致TLE 写题之前要想一想是否有无谓的操作
	\item 数组下标越界 写代码的时候要集中注意力
	\item 我没法一开始就给出完整的算法,或许要边写变改进 我尽量想清楚再说算法
	\item 函数没有返回值 这个犯了一次以后应该都会注意到的
	\item 没有想到题目会卡SPFA,以为算法错 这个以后也不会再犯
	\item 没有大胆写暴力有队过了,看上去又没有别的方法,就可以试一试
	\item 没有考虑重边和自环 这个以后不会再犯
	\item 题目读错 读题应该一边读一边划,特别要注意转述题意很容易出错。每个人写题之前必须读题目的输入格式和输出格式
\end{enumerate}
