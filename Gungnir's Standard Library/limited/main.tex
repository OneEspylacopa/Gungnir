%!TEX program = xelatex
\documentclass[landscape, oneside, a4paper, cs4size]{book}

\def\marginset#1#2{                      % 页边设置 \marginset{left}{top}
\setlength{\oddsidemargin}{#1}         % 左边(书内侧)装订预留空白距离
\iffalse                   % 如果考虑左侧(书内侧)的边注区则改为\iftrue
\reversemarginpar
\addtolength{\oddsidemargin}{\marginparsep}
\addtolength{\oddsidemargin}{\marginparwidth}
\fi
\setlength{\evensidemargin}{0mm}       % 置0
\iffalse                   % 如果考虑右侧(书外侧)的边注区则改为\iftrue
\addtolength{\evensidemargin}{\marginparsep}
\addtolength{\evensidemargin}{\marginparwidth}
\fi
% \paperwidth = h + \oddsidemargin+\textwidth+\evensidemargin + h
\setlength{\hoffset}{\paperwidth}
\addtolength{\hoffset}{-\oddsidemargin}
\addtolength{\hoffset}{-\textwidth}
\addtolength{\hoffset}{-\evensidemargin}
\setlength{\hoffset}{0.5\hoffset}
\addtolength{\hoffset}{-1in}           % h = \hoffset + 1in
%\setlength{\voffset}{-1in}             % 0 = \voffset + 1in
\setlength{\topmargin}{\paperheight}
\addtolength{\topmargin}{-\headheight}
\addtolength{\topmargin}{-\headsep}
\addtolength{\topmargin}{-\textheight}
\addtolength{\topmargin}{-\footskip}
\addtolength{\topmargin}{#2}           % 上边预留装订空白距离
\setlength{\topmargin}{0.5\topmargin}
}
% 调整页边空白使内容居中,两参数分别为纸的左边和上边预留装订空白距离
\marginset{125mm}{200mm}


%\usepackage{ctex}
\usepackage{bm}
%\usepackage[fleqn]{amsmath}
\usepackage{harpoon}
\usepackage{fontspec}
\usepackage{listings}
\usepackage[left=1cm,right=1cm,top=1cm,bottom=1cm,columnsep=1cm,dvipdfm]{geometry}
\usepackage{setspace}
\usepackage{bm}
\usepackage{cmap}
\usepackage{cite}
\usepackage{float}
\usepackage{xeCJK}
\usepackage{amsthm}
\usepackage{amsmath}
\usepackage{amssymb}
\usepackage{multirow}
\usepackage{multicol}
\usepackage{setspace}
\usepackage{enumerate}
\usepackage{indentfirst}
\usepackage{adjmulticol}
\usepackage{titlesec}
\usepackage{color,minted}
\usepackage[Chinese]{ucharclasses}
\allowdisplaybreaks
%\setlength{\parindent}{0em}
%\setlength{\mathindent}{0pt}
\lstset{breaklines}
\let\cleardoublepage\relax
\titleformat{\chapter}{\normalfont\large\bfseries}{Chapter \,\thechapter}{10pt}{\large}
\titleformat{\section}{\normalfont\normalsize\bfseries}{\thesection}{1em}{}
\titleformat{\subsection}{\normalfont\small\bfseries}{\thesubsection}{1em}{}
\titleformat{\subsubsection}{\normalfont\footnotesize\bfseries}{\thesubsubsection}{1em}{}
\titlespacing*{\chapter} {0pt}{0pt}{0pt}
\titlespacing*{\section} {0pt}{0pt}{0pt}
\titlespacing*{\subsection} {0pt}{-1pt}{-1pt}
\titlespacing*{\subsubsection}{0pt}{-1pt}{-1pt}
%configure fonts
\setmonofont{Menlo}[Scale=0.8]
\newfontfamily\substitutefont{STHeiti}[Scale=0.8]
\setTransitionsForChinese{\begingroup\substitutefont}{\endgroup}

\renewcommand{\theFancyVerbLine}{\sffamily \textcolor[rgb]{0.5,0.5,0.5}{\scriptsize {\arabic{FancyVerbLine}}}}

\setminted[cpp]{
	style=xcode,
	mathescape,
	linenos,
	autogobble,
	baselinestretch=0.5,
	tabsize=4,
	fontsize=\small,
	%bgcolor=Gray,
	frame=single,
	framesep=1mm,
	framerule=0.3pt,
	numbersep=1mm,
	breaklines=true,
	breaksymbolsepleft=2pt,
	%breaksymbolleft=\raisebox{0.8ex}{ \small\reflectbox{\carriagereturn}}, %not moe!
	%breaksymbolright=\small\carriagereturn,
	breakbytoken=false,
}
\setminted[java]{
	style=xcode,
	mathescape,
	linenos,
	autogobble,
	baselinestretch=1.0,
	tabsize=4,
	%bgcolor=Gray,
	frame=single,
	framesep=1mm,
	framerule=0.3pt,
	numbersep=1mm,
	breaklines=true,
	breaksymbolsepleft=2pt,
	%breaksymbolleft=\raisebox{0.8ex}{ \small\reflectbox{\carriagereturn}}, %not moe!
	%breaksymbolright=\small\carriagereturn,
	breakbytoken=false,
}
\setminted[text]{
	style=xcode,
	mathescape,
	linenos,
	autogobble,
	baselinestretch=1.0,
	tabsize=4,
	%bgcolor=Gray,
	frame=single,
	framesep=1mm,
	framerule=0.3pt,
	numbersep=1mm,
	breaklines=true,
	breaksymbolsepleft=2pt,
	%breaksymbolleft=\raisebox{0.8ex}{ \small\reflectbox{\carriagereturn}}, %not moe!
	%breaksymbolright=\small\carriagereturn,
	breakbytoken=false,
}
\begin{document}\scriptsize
	\title{\textbf{\LARGE{Gungnir's Standard Code Library}}}
	\author{\emph{Shanghai Jiao Tong University}}
	\date{Dated: \today}
	\maketitle
	\clearpage
	\begin{multicols}{2}
		\tableofcontents
		\clearpage
		\begin{spacing}{0.8}
			\def \source {../source}
\chapter{Computational Geometry}
\section{2D}
\subsection{Basis}
\lstinputlisting{\source/computational-geometry/2d/basis.cpp}

\chapter{数论}
\section{$O(m^2\log n)$求线性递推数列第n项}
Given $a_0, a_1, \ldots, a_{m - 1}$\\
	$a_n = c_0 \times a_{n - m} + \cdots + c_{m - 1} \times a_{n - 1}$\\
	Solve for $a_n = v_0 \times a_0 + v_1 \times a_1 + \cdots + v_{m - 1} \times a_{m - 1}$\\
\inputminted{cpp}{\source/number-theory/linear-recurrence.cpp}
\section{求逆元}
\inputminted{cpp}{\source/number-theory/get-inversion.cpp}
\section{中国剩余定理}
\inputminted{cpp}{\source/number-theory/chinese-remainder-theorem.cpp}
\section{素性测试}
\inputminted{cpp}{\source/number-theory/primality-test.cpp}
\section{质因数分解}
\inputminted{cpp}{\source/number-theory/pollards-rho-algorithm.cpp}
\section{线下整点}
\inputminted{cpp}{\source/number-theory/integer-lattice-under-segment.cpp}

\chapter{代数}
\section{快速傅里叶变换}
\inputminted{cpp}{\source/algebra/fast-fourier-transform.cpp}
\section{自适应辛普森积分}
\inputminted{cpp}{\source/algebra/adaptive-simpsons-method.cpp}
\section{单纯形}
\inputminted{cpp}{\source/algebra/simplex.cpp}

\chapter{字符串}
\section{后缀数组}
\inputminted{cpp}{\source/string/suffix-array.cpp}
\section{后缀自动机}
\inputminted{cpp}{\source/string/suffix-automaton.cpp}
\section{EX后缀自动机}
\inputminted{cpp}{\source/string/ex-suffix-automaton.cpp}
\section{回文自动机}
\inputminted{cpp}{\source/string/palindromic-tree.cpp}

\chapter{Graph Theory}
\section{Basis}
\lstinputlisting{\source/graph-theory/basis.cpp}
\section{Double Connected Graph (vertex)}
\texttt{dcc.forest} is a set of connected tree whose vertices are chequered with cut-vertex and DCC.
\lstinputlisting{\source/graph-theory/double-connected-graph-vertex.cpp}

\chapter{技巧}
\section{释放STL容器内存空间}
\inputminted{cpp}{\source/tricks/truly-release-container-space.cpp}
\section{大整数相乘取模}
Time complexity $O(1)$.
\inputminted{cpp}{\source/tricks/O1-multiply-mod.cpp}

\chapter{提示}

\section{控制cout输出实数精度}
\inputminted{cpp}{\source/hints/control-cout-precision.cpp}
\section{vimrc}
\inputminted{text}{\source/hints/vimrc}
\section{让make支持c艹11}
In .bashrc or whatever:
\begin{verbatim}
export CXXFLAGS='-std=c++11 -Wall'
\end{verbatim}

\section{线性规划转对偶}

\begin{equation*}
\begin{aligned}
&\text{maximize }\mathbf{c}^{T}\mathbf{x}\\
&\text{subject to }\mathbf{A}\mathbf{x} \leq \mathbf{b}, \mathbf{x} \geq 0
\end{aligned}
\Longleftrightarrow
\begin{aligned}
&\text{minimize }\mathbf{y}^{T}\mathbf{b}\\
&\text{subject to }\mathbf{y}^{T}\mathbf{A} \geq \mathbf{c}^{T}, \mathbf{y} \geq 0
\end{aligned}
\end{equation*}

\section{32-bit/64-bit随机素数}
\begin{tabular}{|l|l|}
\hline
\texttt{32-bit} & \texttt{64-bit} \\
\hline
73550053 & 1249292846855685773 \\
\hline
148898719 & 1701750434419805569 \\
\hline
189560747 & 3605499878424114901 \\
\hline
459874703 & 5648316673387803781 \\
\hline
1202316001 & 6125342570814357977 \\
\hline
1431183547 & 6215155308775851301 \\
\hline
1438011109 & 6294606778040623451 \\
\hline
1538762023 & 6347330550446020547 \\
\hline
1557944263 & 7429632924303725207 \\
\hline
1981315913 & 8524720079480389849 \\
\hline
\end{tabular}

\section{NTT 素数及其原根}
\begin{tabular}{|l|l|}
\hline
\texttt{Prime} & \texttt{Primitive root} \\
\hline
1053818881 & 7 \\
\hline
1051721729 & 6 \\
\hline
1045430273 & 3 \\
\hline
1012924417 & 5 \\
\hline
1007681537 & 3 \\
\hline
\end{tabular}


		\end{spacing}
	\end{multicols}
\end{document}
